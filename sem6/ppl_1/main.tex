\documentclass{article}
\usepackage{tikz}
\usetikzlibrary{shapes.geometric, arrows.meta, positioning, calc}

\begin{document}

\begin{tikzpicture}[
    node distance=2cm,
    block/.style={
        rectangle, 
        rounded corners, 
        draw, 
        minimum width=3cm, 
        minimum height=1cm, 
        align=center
    },
    decision/.style={
        diamond, 
        draw, 
        minimum width=2cm, 
        minimum height=1cm, 
        aspect ratio=1,
        align=center
    },
    arrow/.style={
        -Latex, thick
    }
]

% Start node
\node[block] (start) {Start};

% Input process
\node[block, below=of start] (input) {Get Input (e.g., variables)};

% Process 1
\node[block, below=of input] (process1) {Process Data (e.g., calculations)};

% Decision
\node[decision, below=of process1] (decision1) {Condition?};

% Process 2 (Yes branch)
\node[block, below left=of decision1] (process2) {Process if True};

% Process 3 (No branch)
\node[block, below right=of decision1] (process3) {Process if False};

% Merge point
\node[block, below=of process2, xshift=-2cm] (merge) {}; % Invisible node for merging

% Output process
\node[block, below=of process3, xshift=2cm] (output) {Output Result};

% End node
\node[block, below=of merge, xshift=0cm] (end) {End};



% Connections
\draw[arrow] (start) -- (input);
\draw[arrow] (input) -- (process1);
\draw[arrow] (process1) -- (decision1);
\draw[arrow] (decision1) -- (process2) node[midway, left] {Yes};
\draw[arrow] (decision1) -- (process3) node[midway, right] {No};
\draw[arrow] (process2) -- (merge);
\draw[arrow] (process3) -- (output); % Connect directly to output
\draw[arrow] (merge) -- (output);      % Connect merge to output
\draw[arrow] (output) -- (end);


\end{tikzpicture}

\end{document}
