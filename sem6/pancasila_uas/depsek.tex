\documentclass[a4paper,12pt]{article}
\usepackage{times}
\usepackage[bahasa]{babel}
\usepackage{graphicx}
\usepackage{geometry}
\usepackage{titlesec}
\usepackage{setspace}
\usepackage{hangparas}
\usepackage{apacite}

\geometry{margin=2.5cm}
\setstretch{1.5}

% Format section headings
\titleformat{\section}[block]
  {\fontsize{12}{14}\bfseries\filcenter}{\thesection}{1em}{}
\titleformat{\subsection}[block]
  {\fontsize{12}{14}\bfseries}{\thesubsection}{1em}{}

\title{\fontsize{16}{19}\selectfont\bfseries WAWASAN NUSANTARA: DINAMIKA HISTORIS DAN URGENSI DALAM PERGAULAN GLOBAL}
\author{
  \fontsize{11}{13}\selectfont\textbf{Penulis Pertama}^1, 
  \fontsize{11}{13}\selectfont\textbf{Penulis Kedua}^2, 
  \fontsize{11}{13}\selectfont\textbf{Penulis Terakhir}^3 \\
  ^1\fontsize{10}{12}\selectfont Institusi Penulis Pertama \\
  ^2\fontsize{10}{12}\selectfont Institusi Penulis Kedua \\
  ^3\fontsize{10}{12}\selectfont Institusi Penulis Terakhir \\
  \fontsize{10}{12}\selectfont Penulis Korespondensi: email@institusi.ac.id
}

\begin{document}

\maketitle

\begin{abstract}
\fontsize{10}{12}\selectfont\itshape
Penelitian ini menganalisis evolusi konsep Wawasan Nusantara sebagai paradigma geopolitik Indonesia. Melalui pendekatan historis dan analisis kebijakan, studi ini mengungkap transformasi konsep dari masa pra-kolonial hingga implementasi kontemporer...

\textit{Keywords}: geopolitik, integrasi nasional, nusantara, sejarah, wawasan kebangsaan
\end{abstract}

\begin{abstract}
\fontsize{10}{12}\selectfont\itshape
Abstrak harus faktual dan ringkas. Abstrak ditulis dalam bahasa Inggris dan Bahasa atau bahasa Arab dan Bahasa...

\textit{Kata Kunci}: kebangsaan, nusantara, sejarah, wawasan, geopolitik
\end{abstract}

\section{INTRODUCTION - PENDAHULUAN}
Konsep Wawasan Nusantara telah menjadi pilar utama... \cite{Soemarwoto2005}

\section{METHODS - METODE PENELITIAN}
Penelitian ini menggunakan metode kualitatif... \cite{Bappenas2023}

% Gambar dan tabel tetap sama

\section{CONCLUSION - KESIMPULAN}
Wawasan Nusantara terbukti menjadi konsep dinamis...

\bibliographystyle{apacite}
\bibliography{references}

\end{document}
