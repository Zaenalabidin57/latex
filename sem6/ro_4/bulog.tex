\documentclass[12pt,a4paper]{article}
\usepackage{amsmath}
\usepackage{amsfonts}
\usepackage{amssymb}
\usepackage[left=2.00cm, right=1.50cm, top=1.50cm, bottom=1.50cm]{geometry}

\title{Penyelesaian Masalah Produksi dengan Metode Simpleks}
\author{}
\date{}

\begin{document}

\maketitle

\section*{Pendahuluan}
Metode simpleks adalah teknik penyelesaian dalam program linier yang digunakan untuk menentukan alokasi sumber daya secara optimal. Dalam kasus ini, kita akan menganalisis produksi dua jenis produk di PT Yummy Food, yaitu Vanilla dan Violette.

\section*{Formulasi Masalah}
Diberikan data sebagai berikut:
\begin{center}
\begin{tabular}{|c|c|c|c|}
\hline
Jenis Bahan Baku / Tenaga Kerja & Vanilla ($X_1$) & Violette ($X_2$) & Maksimum Penyediaan \\
\hline
Bahan Baku A & 2 kg & 3 kg & 60 kg/hari \\
Bahan Baku B & - & 2 kg & 30 kg/hari \\
Tenaga Kerja & 2 jam & 1 jam & 40 jam/hari \\
Sumbangan Keuntungan & Rp40,00 & Rp30,00 & \\
\hline
\end{tabular}
\end{center}

Fungsi tujuan: 
\[
\text{Maksimalkan } Z = 40X_1 + 30X_2
\]

Kendala:
\[
\begin{aligned}
2X_1 + 3X_2 &\leq 60 \\
2X_2 &\leq 30 \\
2X_1 + X_2 &\leq 40 \\
X_1, X_2 &\geq 0
\end{aligned}
\]

\section*{Penyelesaian dengan Metode Simpleks}

Ubah kendala menjadi bentuk baku dengan menambahkan variabel slack $S_1, S_2, S_3$:
\[
\begin{aligned}
2X_1 + 3X_2 + S_1 &= 60 \\
2X_2 + S_2 &= 30 \\
2X_1 + X_2 + S_3 &= 40 \\
\end{aligned}
\]

Fungsi tujuan:
\[
Z = 40X_1 + 30X_2 + 0S_1 + 0S_2 + 0S_3
\]

\subsection*{Tabel Simpleks Awal}
\begin{center}
\begin{tabular}{|c|c|c|c|c|c|c|c|}
\hline
 & $C_j$ & 40 & 30 & 0 & 0 & 0 & \\
\hline
Ci & BV & $X_1$ & $X_2$ & $S_1$ & $S_2$ & $S_3$ & $Bi$ \\
\hline
0 & $S_1$ & 2 & 3 & 1 & 0 & 0 & 60 \\
0 & $S_2$ & 0 & 2 & 0 & 1 & 0 & 30 \\
0 & $S_3$ & 2 & 1 & 0 & 0 & 1 & 40 \\
\hline
 & $Z_j$ & 0 & 0 & 0 & 0 & 0 & 0 \\
\hline
 & $C_j-Z_j$ & 40 & 30 & 0 & 0 & 0 & \\
\hline
\end{tabular}
\end{center}

\subsection*{Tabel Iterasi Pertama}
Setelah dilakukan iterasi pertama, tabel simpleks menjadi:
\begin{center}
\begin{tabular}{|c|c|c|c|c|c|c|c|}
\hline
 & $C_j$ & 40 & 30 & 0 & 0 & 0 & \\
\hline
Ci & BV & $X_1$ & $X_2$ & $S_1$ & $S_2$ & $S_3$ & $Bi$ \\
\hline
0 & $S_1$ & 0 & 2 & 1 & 0 & -1 & 20 \\
0 & $S_2$ & 0 & 2 & 0 & 1 & 0 & 30 \\
40 & $X_1$ & 1 & ½ & 0 & 0 & ½ & 20 \\
\hline
 & $Z_j$ & 40 & 20 & 0 & 0 & 20 & \\
\hline
 & $C_j-Z_j$ & 0 & 10 & 0 & 0 & -20 & \\
\hline
\end{tabular}
\end{center}

\subsection*{Tabel Iterasi Kedua (Optimal)}
Setelah iterasi kedua, tabel akhir adalah:
\begin{center}
\begin{tabular}{|c|c|c|c|c|c|c|c|}
\hline
 & $C_j$ & 40 & 30 & 0 & 0 & 0 & \\
\hline
Ci & BV & $X_1$ & $X_2$ & $S_1$ & $S_2$ & $S_3$ & $Bi$ \\
\hline
30 & $X_2$ & 0 & 1 & ½ & 0 & -½ & 10 \\
0 & $S_2$ & 0 & 0 & -1 & 1 & 1 & 10 \\
40 & $X_1$ & 1 & 0 & -¼ & 0 & ¾ & 15 \\
\hline
 & $Z_j$ & 40 & 30 & 5 & 0 & 15 & \\
\hline
 & $C_j-Z_j$ & 0 & 0 & -5 & 0 & -15 & 900 \\
\hline
\end{tabular}
\end{center}

\section*{Kesimpulan}
Solusi optimal diperoleh ketika:
\[
\begin{aligned}
X_1 &= 15 \text{ unit (Vanilla)} \\
X_2 &= 10 \text{ unit (Violette)} \\
Z &= \text{Rp 900,00 (Keuntungan maksimum)}
\end{aligned}
\]

\end{document}
