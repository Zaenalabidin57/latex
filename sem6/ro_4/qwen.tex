\documentclass[12pt,a4paper]{article}
\usepackage{amsmath}
\usepackage{amsfonts}
\usepackage{amssymb}
\usepackage{tikz}
\usepackage{float}
\usetikzlibrary{calc,intersections}
\usepackage[left=2.00cm, right=1.50cm, top=1.50cm, bottom=1.50cm]{geometry}

\title{Penyelesaian Masalah Produksi dengan Metode Grafik}
\author{}
\date{}

\begin{document}

\maketitle

\section*{Soal 2: Masalah Optimasi Produksi PT Yummy Food}
PT Yummy Food memiliki pabrik yang memproduksi dua jenis produk: vanilla ($X_1$) dan violette ($X_2$). Maksimum penyediaan sumber daya adalah:

    
- Bahan Baku A: 60 kg/hari
    
- Bahan Baku B: 30 kg/hari
    
- Tenaga Kerja: 40 jam/hari


\begin{table}[H]
\centering
\begin{tabular}{|c|c|c|c|}
\hline
\textbf{Sumber Daya} & \textbf{Vanilla} ($X_1$) & \textbf{Violette} ($X_2$) & \textbf{Maksimum} \\
\hline
Bahan Baku A & 2 kg & \textcolor{red}{3 kg} & 60 kg \\
\hline
Bahan Baku B & - & 2 kg & 30 kg \\
\hline
Tenaga Kerja & 2 jam & 1 jam & 40 jam \\
\hline
Sumbangan Keuntungan & Rp40,00 & Rp30,00 & \\
\hline
\end{tabular}
\caption{Data Kebutuhan Produksi (Data Koreksi)}
\end{table}

\section*{Penyelesaian dengan Metode Grafik}

\subsection*{Langkah 1: Variabel Keputusan}
\begin{align*}
X_1 &= \text{Jumlah unit vanilla/hari} \\
X_2 &= \text{Jumlah unit violette/hari}
\end{align*}

\subsection*{Langkah 2: Fungsi Tujuan}
$$
\text{Maksimalkan } Z = 40X_1 + 30X_2
$$

\subsection*{Langkah 3: Kendala}
\begin{align*}
\text{Bahan Baku A: } & 2X_1 + 3X_2 \leq 60 \\
\text{Bahan Baku B: } & 2X_2 \leq 30 \Rightarrow X_2 \leq 15 \\
\text{Tenaga Kerja: } & 2X_1 + X_2 \leq 40 \\
\text{Non-negatif: } & X_1, X_2 \geq 0
\end{align*}

\subsection*{Langkah 4: Menggambar Daerah Fisibel}
\begin{figure}[H]
\centering
\begin{tikzpicture}[scale=0.7]
% Sumbu koordinat
\draw[->] (-0.5,0) -- (25,0) node[right] {$X_1$};
\draw[->] (0,-0.5) -- (0,25) node[above] {$X_2$};

% Label sumbu
\foreach \x in {5,10,...,20}
    \draw (\x,0.1) -- (\x,-0.1) node[below] {\x};
\foreach \y in {5,10,...,20}
    \draw (0.1,\y) -- (-0.1,\y) node[left] {\y};

% Garis kendala Bahan Baku A: 2X1 + 3X2 = 60
\draw[name path=bbA,blue] (0,20) -- (15,10) -- (30,0) node[right] {BB A};

% Garis kendala Tenaga Kerja: 2X1 + X2 = 40
\draw[name path=kerja,green] (0,40) -- (20,0) node[right] {Tenaga Kerja};

% Garis kendala Bahan Baku B: X2 = 15
\draw[name path=bbB,red] (0,15) -- (25,15) node[right] {BB B};

% Titik potong BB A dan Tenaga Kerja
\path[name intersections={of=bbA and kerja,by=optimal}];
\fill[orange] (optimal) circle (3pt) node[above right] {(15,10)};

% Daerah fisibel
\fill[gray!30] 
    (0,0) 
    -- (0,15) 
    -- (intersection of bbB and bbA) 
    -- (optimal) 
    -- (20,0) -- cycle;

% Garis tujuan Z = 40X1 + 30X2 = 900
\draw[dashed,orange] (0,30) -- (22.5,0) node[right] {Z=900};

\end{tikzpicture}
\caption{Grafik Solusi Optimal - Masalah Produksi PT Yummy Food}
\end{figure}

\subsection*{Langkah 5: Menentukan Titik-Titik Sudut}
Dengan koreksi data, titik-titik sudut daerah fisibel adalah:
\begin{enumerate}
    \item $(0,0)$: Titik asal
    \item $(0,15)$: Titik potong kendala BB B dengan sumbu $X_2$
    \item $(12,15)$: Titik potong kendala BB B ($X_2=15$) dan BB A ($2X_1 + 3X_2=60$)
    \item $(15,10)$: Titik potong kendala BB A ($2X_1 + 3X_2=60$) dan Tenaga Kerja ($2X_1 + X_2=40$)
    \item $(20,0)$: Titik potong kendala Tenaga Kerja dengan sumbu $X_1$
\end{enumerate}

\subsection*{Langkah 6: Evaluasi Fungsi Tujuan}
$$
\begin{array}{l|l}
\text{Titik} & Z = 40X_1 + 30X_2 \\
\hline
(0,0) & 0 \\
(0,15) & 450 \\
(12,15) & 930 \\
\textcolor{red}{(15,10)} & \textcolor{red}{900} \\
(20,0) & 800 \\
\end{array}
$$

\subsection*{Kesimpulan}
Solusi optimal tercapai pada titik $(15,10)$ dengan keuntungan maksimum:
$$
Z = 40(15) + 30(10) = \text{Rp }900,00
$$
Hasil ini sesuai dengan penyelesaian metode simpleks sebelumnya.

\end{document}
