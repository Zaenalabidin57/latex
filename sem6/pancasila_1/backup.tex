% Preamble
\documentclass[12pt, a4paper]{article}
\usepackage[bahasa]{babel}
\usepackage{times}
\usepackage[utf8]{inputenc}
\usepackage[T1]{fontenc}
\usepackage{graphicx}
\usepackage{caption}
\usepackage[left=3cm, right=3cm, top=3cm, bottom=3cm]{geometry}
\usepackage{setspace}
\usepackage{enumitem}
\usepackage{ragged2e}
\usepackage{xurl}
\usepackage{apacite}
\usepackage{etoolbox}
\usepackage{titlesec}


% Global line spacing (1.5)
\onehalfspacing


\titleformat{\section}
  {\fontsize{12pt}{14pt}\bfseries\uppercase}
  {\thesection}
  {1em}
  {}
\titleformat{\subsection}
  {\fontsize{12pt}{14pt}\bfseries}
  {\thesection}
  {1em}
  {}
% Custom command for Abstract/Abstrak titles
  \newcommand{\makecustomtitle}[1]{\begin{center}\large\bfseries\fontsize{12pt}{14pt}\MakeUppercase{#1}\end{center}\vspace{0.5cm}}

\begin{document}

% Title Page
\begin{titlepage}
    \centering
    \vspace*{2cm}
    {\LARGE\bfseries Bagaimana dinamika historis dan urgensi wawasan nusantara sebagai konsepsi dan pandangan kolektif kebangsaan indonesia dalam konteks pergaulan dunia\par}
    \vspace{1cm}
    {\large\bfseries Rifqi Fadil Fahrial$^1$\par}
    \vspace{0.5cm}
    \normalsize
    $^1$Sekolah Tinggi Teknologi Bandung (STTB)\par
    \vspace{0.5cm}
    Penulis: rifqifadil57@gmail.com
    \vfill
\end{titlepage}

% ABSTRACT (English)
\makecustomtitle{ABSTRACT}
\begin{minipage}{\textwidth}
\singlespacing
\justifying
\textit{\small
Nusantara's insight in the way for the indonesian people to view their country living states. how the way of the nation's unity and diversity in the world politics. Based of \textit{lembaga ketahanan nasional} year 1999 the meaning of wawasan nusatara is the point of view of the indonesian people about their self and their community with varying degree of diversity with many strategic value with emphasis on unity and solidarity between nation and region to achieve national goal.}
\vspace{0.5em}
\par\noindent
\textit{Keywords:} wawasan nusantara, geopolitics, indonesia, unity, diversity
\end{minipage}
\vspace{1cm}

% ABSTRAK (Bahasa Indonesia)
\makecustomtitle{ABSTRAK}
\begin{minipage}{\textwidth}
\singlespacing
\justifying
\textit{\small
Wawasan nusantara merupakan cara pandang bangsa terhadap diri dan lingkungan tempat hidup bangsa yang bersangkutan. Cara bangsa memandang diri dan lingungannya sangat mempengaruhi keberlangsungan dan keberhasilan bangsa itu menuju tujuannya. Berdasarkan lembaga ketahanan nasional tahun 1999 pengertian wawasan nusantara adalah sebuah cara pandang dan sikap bangsa indonesia mengenai diri dan lingkungannya yang serba beragam dan bernilai strategis dengan mengutamakan persatuan dan kesatuan bangsa serta kesatuan wilayah dalam penyelenggaraan kehidupan bermasyarakat, berbangsa dan bernegara untuk mencapai tujuan nasional.}
\vspace{0.5em}
\par\noindent
\textit{Kata Kunci:} wawasan nusantara, geopolitik, indonesia, persatuan, kesatuan
\end{minipage}
\vspace{1cm}

% Main Content
\section{PENDAHULUAN}
\setlength{\parindent}{1em}
\justifying
Wawasan nusantara ini bisa dibagi menjadi dua pengertian yakni pengertian secara etimologis dan pengertian terminologi.

\subsection*{Etimologi}
Secara etimologi, kata wawasan nusantara berasal dari dua kata wawasan dan nusantara. Wawasan dari kata wawas (Bahasa Jawa) yang artinya pandangan. Sementara kata "nusantara" merupakan gabungan kata nusa yang artinya pulau dan antara. Kata "nusa" dalam bahasa Sansekerta berarti pulau atau kepulauan. Dan kata "Antara" memiliki padanan dalam bahasa Latin, \textit{in} dan \textit{terra} yang berarti antara atau dalam suatu kelompok, kata nusantara dapat diartikan sebagai kepulauan yang diantara laut atau bangsa-bangsa yang dihubungkan oleh laut. 

Kata nusantara sendiri bermula dari bunyi Sumpah Pala dari Patih Gajah Mada yang diucapkan dalam upacara pengangkatannya sebagai Mahapatih di Kerajaan Majapahit tahun 1336 M, Tertulis di dalam Kitab Paraton. Bunyi sumpah tersebut sebagai berikut:

\begin{quote}
\itshape
Gajah Mada Patih Amangkubumi tidak ingin melepaskan puasa. Ia Gajah Mada, "Jika telah mengalahkan nusantara, saya (baru akan) melepaskan puasa. Jika mengalahkan Gurun, Seran, Tanjung Pura, Haru, Pahang, Dompo, Bali, Sunda, Palembang, Tumasik, demikianlah saya (baru akan) melepaskan puasa".
\end{quote}

Penamaan Nusantara ini berdasarkan sudut pandang Majapahit (Jawa), mengingat pada waktu itu belum ada sebutan yang cocok untuk menyebut seluruh kepulauan yang sekarang bernama Indonesia dan juga Malaysia. Kemudian Kata Nusantara digunakan oleh Ki Hajar Dewantara untuk menggantikan sebutan Hindia Belanda (\textit{Nederlandsch-Indie}), pada acara Kongres Pemuda Indonesia II tahun 1928 (peristiwa Sumpah Pemuda), digunakan istilah Indonesia sebagai pengganti Nusantara. Nama Indonesia berasal dari dua kata bahasa Yunani, yaitu \textit{indo/indu} yang berarti Hindu/Hindia dan \textit{nesia/nesos} yang berarti pulau. Dengan demikian kata nusantara bisa dipakai sebagai sinonim kata indonesia, yang menunjuk pada wilayah (sebaran pulau-pulau) yang berada di antara dua samudra yakni Samudra Hindia dan Samudra Pasifik dan dua benua yakni Benua Asia dan Australia.

\subsection*{Terminologis}
Untuk Pengertian terminologis, wawasan nusantara merupakan pandangan bangsa Indonesia terhadap lingkungan tempat berada termasuk diri bangsa Indonesia itu sendiri. Rumusan Wawasan Nusantara dalam GBHN 1998\cite{GBHN1998}:

\begin{quote}
\itshape
"Wawasan Nusantara adalah cara pandang dan sikap bangsa Indonesia mengenai diri dan lingkungannya, dengan mengutamakan persatuan dan kesatuan bangsa serta kesatuan wilayah dalam penyelenggaraan kehidupan bermasyarakat, berbangsa dan bernegara".
\end{quote}

\section{Latar Belakang Historis Wawasan Nusantara}
Lahirnya Konsepsi wawasan nusantara bermula dari Perdana Menteri Ir. H. Djuanda Kartawidjaja yang pada tanggal 13 Desember 1957 \cite{hari_nusantara_upi} mengeluarkan deklarasi yang selanjutnya dikenal sebagai Deklarasi Djuanda. Isinya sebagai berikut:

\begin{quote}
\itshape
"Bahwa segala perairan di sekitar, di antara dan yang menghubungkan pulau-pulau yang termasuk Negara Indonesia dengan tidak memandang luas atau lebarnya adalah bagian-bagian yang wajar daripada wilayah daratan Negara Indonesia dan dengan demikian bagian daripada perairan pedalaman atau nasional yang berada di bawah kedaulatan mutlak Negara Indonesia. Lalu lintas yang damai di perairan pedalaman ini bagi kapal-kapal asing dijamin selama dan sekedar tidak bertentangan dengan/mengganggu kedaulatan dan keselamatan Negara Indonesia. Penentuan batas landas lautan teritorial (yang lebarnya 12 mil) diukur dari garis yang menghubungkan titik-titik ujung yang terluar pada pulau-pulau Negara Indonesia. Ketentuan-ketentuan tersebut di atas akan diatur selekas-lekasnya dengan Undang-Undang"
\end{quote}

\textit{yang intinya mendeklarasikan bahwa perairan di sekitar, di antara, dan yang menghubungkan pulau-pulau Indonesia adalah bagian dari wilayah daratan Indonesia. Ini mengubah pandangan sebelumnya yang berdasarkan Ordonansi 1939, di mana laut hanya diakui sejauh 3 mil dari garis pantai.}

Kemudian pada tanggal 30 April 1982, PBB mengeluarkan dokumen yang bernama \textit{"The United Nation Convention on the Law of the Sea"} \cite{UN1982} (UNCLOS) yang mengakui Indonesia sebagai negara kelautan yang memiliki wilayah laut sebesar 5.9 Juta km\textsuperscript{2}.


\section{Latar Belakang Sosiologis Wawasan Nusantara}
Pada awalnya Bangsa Indonesia adalah bangsa yang terpecah belah dan berfokus pada masing-masing suku dikarenakan diadu domba dalam masa penjajahan oleh Belanda yang disebut teknik politik \textit{Devide et Impera} yang mana kondisi sosial budaya Indonesia hanya berputar pada satu golongan saja dan tidak ada kesatuan, namun kemudian mulainya keluar bibit-bibit persatuan yang dimulai dengan peristiwa Kebangkitan Nasional pada Tanggal 20 Mei 1908 yang kemudian Ditegaskan kembali dalam Sumpah Pemuda pada 28 Oktober 1928 yang kemudian berhasil diwujudkan dengan proklamasi Kemerdekaan bangsa pada tanggal 17 Agustus 1945 yang berhasil menyatukan Indonesia, oleh karena itu sebelum Deklarasi Djuanda 1957 pun konsep semangat dan kesatuan kebangsaan sudah tumbuh dalam diri bangsa.


\section{Latar Belakang Politis Wawasan Nusantara}
Wawasan Nusantara menjadi konsep politik kenegaraan untuk mempertahankan kesatuan wilayah dan bangsa. Letak geografis Indonesia yang strategis, yaitu di antara dua benua (Asia dan Australia) dan dua samudra (Hindia dan Pasifik), juga memengaruhi pandangan geopolitik Indonesia.

\section{Esensi dan Urgensi Wawasan Nusantara}
\subsection*{Esensi}
\begin{itemize}
    \item Kesatuan Wilayah: Indonesia sebagai negara kepulauan dengan 17.508 pulau.
    \item Persatuan bangsa: Keragaman suku, agama dan budaya harus dipandang sebagai satu kesatuan.
\end{itemize}

\subsection*{Urgensi:}
\begin{itemize}
    \item Dampak globalisasi merupakan salah satu faktor utama. Arus informasi, budaya, dan nilai-nilai global yang masuk secara masif dan cepat melalui berbagai kanal berpotensi menggerus nilai-nilai luhur lokal, tradisi bangsa, dan semangat nasionalisme, terutama di kalangan generasi muda. Dominasi sistem ekonomi kapitalisme global juga dapat memengaruhi kedaulatan ekonomi nasional dan memperlebar kesenjangan.  
    \item potensi disintegrasi bangsa yang bersumber dari isu-isu Suku, Agama, Ras, dan Antargolongan (SARA) serta munculnya gerakan separatisme. Keberagaman Indonesia yang luar biasa, meskipun merupakan kekayaan, juga menyimpan potensi konflik jika tidak dikelola dengan bijaksana dan dilandasi semangat toleransi serta saling menghargai. Gerakan separatisme di beberapa daerah, meskipun intensitasnya bervariasi, tetap menjadi ancaman laten terhadap keutuhan wilayah NKRI.
    \item Meningkatkan kesejahteraan rakyat melalui pengelolaan sumber daya alam yang adil dan merata.
\end{itemize}


\section{Tantangan Kontemporer, Prospek Masa Depan, dan Strategi Penguatan Wawasan Nusantara}
\subsection*{Analisis Tantangan Internal}

Implementasi Wawasan Nusantara sebagai pandangan kolektif kebangsaan menghadapi berbagai tantangan internal yang memerlukan perhatian serius. Salah satu tantangan utama adalah kesenjangan pembangunan antarwilayah. Perbedaan tingkat kemajuan ekonomi dan akses terhadap layanan publik antara wilayah barat dan timur Indonesia, serta antara perkotaan dan perdesaan, masih menjadi persoalan nyata. Kesenjangan ini, jika tidak ditangani secara efektif, dapat memicu kecemburuan sosial, rasa ketidakadilan, dan pada akhirnya dapat mengancam semangat persatuan dan kesatuan bangsa yang menjadi inti Wawasan Nusantara. Padahal, Wawasan Nusantara secara eksplisit mengamanatkan terwujudnya pembangunan yang merata dan berkeadilan di seluruh pelosok tanah air.  

Tantangan internal lainnya adalah potensi disintegrasi bangsa yang bersumber dari isu-isu Suku, Agama, Ras, dan Antargolongan (SARA) serta munculnya gerakan separatisme. Keberagaman Indonesia yang luar biasa, meskipun merupakan kekayaan, juga menyimpan potensi konflik jika tidak dikelola dengan bijaksana dan dilandasi semangat toleransi serta saling menghargai. Gerakan separatisme di beberapa daerah, meskipun intensitasnya bervariasi, tetap menjadi ancaman laten terhadap keutuhan wilayah NKRI.  

Dalam konteks ini, penguatan identitas nasional yang inklusif menjadi sangat penting. Identitas nasional tidak boleh meniadakan atau meminggirkan identitas-identitas lokal yang beragam. Sebaliknya, Wawasan Nusantara mengajarkan bahwa budaya-budaya lokal adalah bagian integral dari kekayaan budaya nasional dan harus dilestarikan serta dikembangkan sebagai pilar-pilar yang memperkokoh identitas bangsa secara keseluruhan. Membangun narasi kebangsaan yang merangkul seluruh elemen masyarakat tanpa terkecuali adalah kunci untuk memperkuat kohesi sosial.  

Tantangan-tantangan internal terhadap Wawasan Nusantara ini seringkali berakar pada persepsi mendalam mengenai ketidakadilan, baik dalam distribusi sumber daya dan hasil pembangunan (keadilan distributif) maupun dalam pengakuan terhadap eksistensi dan kontribusi berbagai kelompok masyarakat (keadilan rekognitif). Apabila persepsi ketidakadilan ini terus menguat dan tidak mendapatkan respons kebijakan yang memadai, ia dapat dengan mudah dieksploitasi oleh pihak-pihak yang memiliki agenda untuk memecah belah persatuan bangsa. Oleh karena itu, implementasi Wawasan Nusantara yang sejati dan substantif menuntut komitmen yang kuat dari negara untuk mewujudkan keadilan dalam segala dimensinya. Upaya penguatan Wawasan Nusantara yang hanya bersifat top-down, seremonial, atau retoris, tanpa diiringi dengan kebijakan konkret yang mengatasi akar permasalahan ketidakadilan dan kurangnya representasi, akan cenderung kurang efektif dan tidak berkelanjutan.
\subsection*{Analisis Tantangan Eksternal}

Selain tantangan internal, implementasi Wawasan Nusantara juga dihadapkan pada berbagai tantangan eksternal yang signifikan di era kontemporer. Dampak globalisasi merupakan salah satu faktor utama. Arus informasi, budaya, dan nilai-nilai global yang masuk secara masif dan cepat melalui berbagai kanal berpotensi menggerus nilai-nilai luhur lokal, tradisi bangsa, dan semangat nasionalisme, terutama di kalangan generasi muda. Dominasi sistem ekonomi kapitalisme global juga dapat memengaruhi kedaulatan ekonomi nasional dan memperlebar kesenjangan.  

Perkembangan Teknologi Informasi dan Komunikasi (TIK) yang pesat menghadirkan tantangan sekaligus peluang. Di satu sisi, TIK mempermudah akses terhadap informasi, memperlancar komunikasi lintas batas, dan membuka peluang ekonomi baru. Namun, di sisi lain, TIK juga menjadi medium penyebaran berita bohong (hoaks), ujaran kebencian, konten ilegal (seperti pornografi dan perjudian daring), serta perundungan siber (cyberbullying), yang semuanya dapat merusak tatanan sosial dan mengancam persatuan bangsa. Era digital menuntut adanya adaptasi dalam cara pandang dan implementasi Wawasan Nusantara agar tetap relevan dan mampu menangkal dampak negatif TIK.  

Dinamika geopolitik internasional yang terus berubah juga menjadi tantangan tersendiri. Persaingan antara kekuatan-kekuatan global, munculnya sengketa wilayah di kawasan sekitar Indonesia, serta meningkatnya ancaman keamanan transnasional (seperti terorisme, penyelundupan narkoba, dan kejahatan siber) memerlukan respons strategis yang cerdas dan adaptif, yang tetap berlandaskan pada prinsip-prinsip Wawasan Nusantara.  

Menariknya, tantangan eksternal, khususnya yang berasal dari globalisasi dan kemajuan TIK, tidak secara otomatis berakibat pada pelemahan Wawasan Nusantara. Sebaliknya, tantangan-tantangan ini justru dapat menjadi katalisator untuk memperkuat relevansi dan urgensi Wawasan Nusantara. Jika dimanfaatkan secara cerdas dan strategis, globalisasi dan TIK dapat menjadi alat yang efektif untuk mempromosikan narasi persatuan Indonesia, kekayaan budayanya, dan kepentingan nasionalnya di kancah global. Misalnya, platform digital dapat digunakan untuk menyebarkan pesan-pesan positif tentang keberagaman dan harmoni di Indonesia, serta untuk membangun solidaritas nasional dalam menghadapi isu-isu bersama. Wawasan Nusantara, dalam konteks ini, perlu diadaptasi agar tidak hanya menjadi panduan internal, tetapi juga sebagai landasan untuk berinteraksi dan berkontribusi dalam komunitas global. Kerjasama internasional di era digital, misalnya dalam penanganan kejahatan siber atau promosi budaya, menjadi semakin penting. Dengan demikian, Indonesia berpotensi untuk "mengekspor" nilai-nilai positif yang terkandung dalam Wawasan Nusantara, seperti kemampuan untuk hidup rukun dalam keberagaman, sebagai kontribusi bagi upaya membangun perdamaian dan stabilitas global. Hal ini akan mengubah Wawasan Nusantara dari sekadar konsep yang berorientasi ke dalam (inward-looking) menjadi sebuah aset diplomasi publik (public diplomacy) yang berharga di panggung internasional.  

\subsection*{Strategi Komprehensif Penguatan Wawasan Nusantara}

Menghadapi berbagai tantangan internal dan eksternal, diperlukan strategi komprehensif untuk terus memperkuat dan mengaktualisasikan Wawasan Nusantara dalam kehidupan berbangsa dan bernegara. Strategi ini harus melibatkan berbagai sektor dan aktor, serta bersifat berkelanjutan.
\subsubsection*{Peran Pendidikan dan Kebudayaan}

Sektor pendidikan dan kebudayaan memegang peran sentral dalam menanamkan nilai-nilai Wawasan Nusantara sejak dini. Kementerian Pendidikan, Kebudayaan, Riset, dan Teknologi (Kemendikbudristek) memiliki berbagai program yang relevan. Salah satunya adalah Projek Penguatan Profil Pelajar Pancasila (P5), yang merupakan bagian integral dari Kurikulum Merdeka. P5 bertujuan untuk mengembangkan karakter dan kompetensi siswa agar sesuai dengan nilai-nilai luhur Pancasila melalui pendekatan pembelajaran berbasis proyek yang kontekstual dan relevan dengan lingkungan sekitar siswa. Salah satu tema dalam P5, yaitu "Kearifan Lokal," secara langsung dapat digunakan untuk menggali, memahami, dan mengapresiasi keragaman budaya Nusantara, yang merupakan esensi dari Wawasan Nusantara.  

Selain P5, program Pendidikan Karakter juga terus digalakkan untuk menanamkan nilai-nilai fundamental seperti religiusitas, kejujuran, toleransi, disiplin, kerja keras, cinta tanah air, semangat kebangsaan, dan penghargaan terhadap keberagaman budaya dan tradisi. Penanaman nilai nasionalisme dan cinta tanah air secara eksplisit menjadi bagian dari nilai-nilai utama karakter yang dikembangkan. Internalisasi Wawasan Kebangsaan secara lebih luas dilakukan melalui jalur pendidikan formal (integrasi dalam kurikulum mata pelajaran seperti Pendidikan Pancasila dan Kewarganegaraan) maupun jalur informal (kegiatan ekstrakurikuler, pembiasaan di lingkungan sekolah dan keluarga).  

Upaya pelestarian dan pengembangan budaya lokal juga merupakan bagian tak terpisahkan dari penguatan Wawasan Nusantara. Budaya lokal diakui sebagai bagian integral dari kekayaan dan identitas nasional, yang harus dijaga kelestariannya dan dikembangkan agar tetap relevan di tengah perubahan zaman.  

\subsubsection*{Peran Kementerian Agama (Kemenag)}

Kementerian Agama memiliki peran strategis dalam mempromosikan harmoni dan kerukunan di tengah masyarakat Indonesia yang majemuk secara agama, yang sejalan dengan semangat Wawasan Nusantara. Salah satu program utama Kemenag adalah Moderasi Beragama. Program ini bertujuan untuk menumbuhkan cara pandang, sikap, dan praktik beragama yang melindungi martabat kemanusiaan, membangun kemaslahatan umum, serta berlandaskan pada prinsip keadilan, keseimbangan, dan ketaatan terhadap konstitusi negara. Moderasi beragama bukanlah upaya untuk memoderasi ajaran agama itu sendiri, melainkan memoderasi pemahaman dan praktik umat beragama agar selaras dengan nilai-nilai kebangsaan.  

Selain itu, Kemenag juga aktif dalam Penguatan Kerukunan Umat Beragama. Upaya ini dilakukan melalui berbagai kegiatan, seperti optimalisasi dan sosialisasi peraturan perundang-undangan terkait kerukunan, peningkatan kapasitas para aktor kerukunan (termasuk Forum Kerukunan Umat Beragama - FKUB), pemberdayaan lembaga-lembaga keagamaan dan institusi media, pengembangan kesadaran akan pentingnya kerukunan, serta pembinaan pemahaman ajaran agama yang berwawasan moderat dan multikultural.  

\subsubsection*{Kebijakan Publik yang Berorientasi pada Pemerataan dan Kesejahteraan}

Untuk mengatasi tantangan kesenjangan pembangunan, diperlukan kebijakan publik yang secara nyata berorientasi pada pemerataan dan peningkatan kesejahteraan di seluruh wilayah Indonesia. Pembangunan yang merata dan berkeadilan adalah salah satu pilar penting dalam menjaga keutuhan bangsa, sebagaimana diamanatkan oleh Wawasan Nusantara. Implementasi otonomi daerah juga harus senantiasa diselaraskan dengan kerangka Wawasan Nusantara agar tidak mengarah pada disintegrasi, melainkan justru memperkuat integrasi nasional melalui pemberdayaan daerah yang bertanggung jawab.  

\subsubsection*{Doktrin Sistem Pertahanan dan Keamanan Rakyat Semestasebagai Penjabaran Aspek Pertahanan Wawasan Nusantara}

Dalam aspek pertahanan dan keamanan, Wawasan Nusantara dijabarkan lebih lanjut dalam doktrin Sistem Pertahanan dan Keamanan Rakyat Semesta \cite{SISHANKAMRATA}. Sishankamrata adalah sistem pertahanan dan keamanan negara yang melibatkan seluruh potensi, kemampuan, dan kekuatan nasional (militer dan non-militer, termasuk seluruh warga negara dan sumber daya nasional lainnya) yang bekerja secara total, integral, terpadu, terarah, dan berlanjut untuk menghadapi segala macam ancaman. Sifat kesemestaan ini mengandung makna pelibatan seluruh rakyat dan segenap sumber daya nasional, sarana dan prasarana nasional, serta seluruh wilayah negara sebagai satu kesatuan pertahanan.  

Wawasan Nusantara, yang memandang kepulauan Nusantara sebagai satu kesatuan pertahanan dan keamanan, menjadi landasan visional bagi penyelenggaraan Sishankamrata. Amanat Undang-Undang Dasar 1945 Pasal 30 mengenai hak dan kewajiban setiap warga negara untuk ikut serta dalam usaha pertahanan dan keamanan negara menjadi dasar konstitusional Sishankamrata. Tujuan utama Sishankamrata adalah untuk menjaga kedaulatan negara, keutuhan wilayah NKRI, dan keselamatan segenap bangsa dari berbagai ancaman, baik yang datang dari luar maupun dari dalam negeri. Doktrin ini lahir dan berkembang dari sejarah perjuangan bangsa Indonesia dan disesuaikan dengan kondisi geografis Indonesia sebagai negara kepulauan.  

\subsubsection*{Peran Aktif Masyarakat Sipil dan Media}

Penguatan Wawasan Nusantara tidak hanya menjadi tanggung jawab pemerintah, tetapi juga memerlukan peran aktif dari seluruh komponen masyarakat sipil. Setiap warga negara diharapkan dapat memahami, menghayati, dan mengamalkan nilai-nilai Wawasan Nusantara dalam kehidupan sehari-hari. Media massa, termasuk media sosial, memiliki peran strategis dalam proses ini. Media dapat digunakan sebagai sarana untuk menyebarkan informasi yang akurat, pesan-pesan positif tentang keberagaman budaya dan persatuan bangsa, serta membangun kesadaran kolektif akan pentingnya Wawasan Nusantara.  

Strategi penguatan Wawasan Nusantara yang paling efektif adalah strategi yang bersifat multidimensional, melibatkan berbagai aspek kehidupan; partisipatif, melibatkan seluruh elemen bangsa; dan adaptif, mampu merespons perubahan zaman dan tantangan baru. Pendekatan yang hanya bersifat top-down dari negara cenderung kurang optimal. Sebaliknya, upaya penguatan Wawasan Nusantara harus mampu memberdayakan inisiatif-inisiatif yang tumbuh dari bawah (bottom-up), serta memanfaatkan kemajuan teknologi informasi dan komunikasi secara positif dan konstruktif. Penguatan Wawasan Nusantara tidak bisa lagi bersifat monolitik atau dogmatis. Ia harus menjadi sebuah "gerakan" kebangsaan yang melibatkan beragam aktor (pemerintah, lembaga pendidikan, keluarga, komunitas lokal, organisasi masyarakat sipil, media, dan individu) dan menggunakan berbagai platform (kurikulum pendidikan, kegiatan budaya, konten digital kreatif, dialog publik, keteladanan) untuk menanamkan dan mengaktualisasikan nilai-nilai kesatuan dalam keberagaman secara relevan, menarik, dan berkelanjutan bagi seluruh generasi bangsa Indonesia.
\section{Kesimpulan dan Rekomendasi Strategis}
\subsection*{Sintesis Temuan Kunci mengenai Dinamika Historis dan Urgensi Wawasan Nusantara}

Analisis terhadap dinamika historis dan urgensi Wawasan Nusantara sebagai konsepsi dan pandangan kolektif kebangsaan Indonesia dalam konteks pergaulan dunia mengungkapkan beberapa temuan kunci. 
\subsubsection*{perubahan konsepsi wilayah maritim Indonesia} \label{perubahan-konsepsi-wilayah-maritim-indonesia} pertama Revolusi konsepsi wilayah maritim Indonesia menunjukkan transformasi fundamental dari rezim Territoriale Zeeën en Maritieme Kringen Ordonantie 1939 (TZMKO 1939)\cite{TZMKO} yang bersifat fragmentatif dan merupakan warisan kolonial, menuju proklamasi kesatuan melalui Deklarasi Djuanda 1957. Perjuangan diplomatik yang gigih, terutama di bawah kepemimpinan intelektual Prof. Mochtar Kusumaatmadja, berpuncak pada pengakuan internasional atas konsep negara kepulauan dalam UNCLOS 1982. Pengakuan ini kemudian diformalkan dan diintegrasikan ke dalam sistem hukum nasional melalui serangkaian peraturan perundang-undangan, yang menandai legitimasi penuh Wawasan Nusantara secara yuridis.

\subsubsection*{Wawasan Nusantara lebih dari sekadar konsep geopolitik}
Kedua, Wawasan Nusantara lebih dari sekadar konsep geopolitik; ia adalah pandangan hidup bangsa yang berakar kuat pada falsafah Pancasila dan semangat Bhinneka Tunggal Ika. Konsepsi ini bersifat multidimensional, mencakup kesatuan dalam aspek politik, ekonomi, sosial budaya, serta pertahanan dan keamanan (Poleksosbudhankam), yang memandang seluruh wilayah darat, laut, dan udara Indonesia sebagai satu kesatuan yang utuh dan tidak terpisahkan.

\subsubsection*{Urgensi kontemporer Wawasan Nusantara} 
Ketiga urgensi kontemporer Wawasan Nusantara terletak pada relevansinya sebagai kerangka panduan strategis bagi Indonesia dalam menghadapi berbagai tantangan kompleks, baik yang bersifat internal maupun eksternal. Tantangan internal meliputi kesenjangan pembangunan antarwilayah, potensi disintegrasi akibat isu SARA dan separatisme, serta kebutuhan untuk terus memperkuat identitas nasional yang inklusif. Tantangan eksternal mencakup dampak globalisasi, pesatnya perkembangan teknologi informasi dan komunikasi, dinamika geopolitik regional dan internasional yang fluktuatif, serta isu-isu global seperti perubahan iklim dan keamanan siber. Wawasan Nusantara menyediakan landasan bagi Indonesia untuk mengelola wilayah maritimnya yang luas, menegakkan kedaulatan, merumuskan kebijakan luar negeri yang bebas aktif, serta berkontribusi pada tatanan dunia yang lebih baik.

\subsection*{Rekomendasi Kebijakan dan Langkah Aplikatif untuk Memperkokoh Wawasan Nusantara dalam Menghadapi Kompleksitas Global}
Berdasarkan analisis yang telah dilakukan, berikut adalah beberapa rekomendasi kebijakan dan langkah aplikatif yang dapat dipertimbangkan untuk memperkokoh Wawasan Nusantara dalam menghadapi kompleksitas global:
\subsubsection*{Bidang Pendidikan dan Kebudayaan}
\begin{enumerate}
  \item Mengintegrasikan secara lebih mendalam dan kreatif nilai-nilai Wawasan Nusantara ke dalam seluruh jenjang dan jalur pendidikan, tidak hanya melalui mata pelajaran khusus tetapi juga melalui kegiatan kokurikuler dan ekstrakurikuler yang kontekstual, seperti penguatan program Projek Penguatan Profil Pelajar Pancasila (P5) dengan tema-tema yang relevan dengan kearifan lokal dan kesatuan maritim.
  \item Mendorong penelitian dan pengembangan konten-konten pendidikan Wawasan Nusantara yang inovatif dan menarik bagi generasi muda, memanfaatkan platform digital dan media sosial secara optimal.
  \item Memperkuat program pelestarian dan revitalisasi budaya lokal sebagai bagian integral dari kekayaan budaya nasional, serta mempromosikannya sebagai aset diplomasi budaya Indonesia di kancah internasional.
\end{enumerate}
\subsubsection*{Bidang Hukum dan Tata Kelola Pemerintahan}
\begin{enumerate}
  \item Melakukan kajian komprehensif terhadap sinkronisasi dan harmonisasi berbagai peraturan perundang-undangan sektoral (Kelautan, Pertahanan, Penataan Ruang, Pemerintahan Daerah, dll.) untuk memastikan implementasi Wawasan Nusantara yang lebih terpadu dan efektif, serta mengatasi potensi tumpang tindih kewenangan antar lembaga.
  \item Memperkuat kapasitas kelembagaan dan sumber daya manusia aparatur negara di semua tingkatan dalam memahami dan mengimplementasikan Wawasan Nusantara dalam setiap perumusan dan pelaksanaan kebijakan publik.
  \item Meningkatkan transparansi dan akuntabilitas dalam pengelolaan sumber daya alam, khususnya sumber daya maritim, untuk memastikan manfaatnya dirasakan secara adil oleh seluruh masyarakat, terutama di daerah pesisir dan pulau-pulau kecil.
\end{enumerate}
\subsubsection*{Bidang Ekonomi dan Pembangunan}
\begin{enumerate}
  \item Mempercepat upaya pemerataan pembangunan infrastruktur fisik dan digital di seluruh wilayah Indonesia, khususnya di daerah tertinggal, terdepan, dan terluar (3T), untuk mengurangi kesenjangan dan memperkuat konektivitas nasional sebagai perwujudan nyata Wawasan Nusantara.
  \item Mengoptimalkan program Tol Laut dan pengembangan sentra-sentra ekonomi maritim yang terintegrasi, dengan melibatkan partisipasi aktif masyarakat lokal dan pelaku usaha daerah.
  \item Mendorong pengembangan ekonomi biru (blue economy) yang berkelanjutan, yang menyeimbangkan antara pemanfaatan potensi ekonomi kelautan dengan pelestarian lingkungan laut.
\end{enumerate}
\subsubsection*{Bidang Pertahanan dan Keamanan}
\begin{enumerate}
  \item Terus memodernisasi dan memperkuat kapasitas pertahanan dan keamanan nasional, khususnya matra laut dan udara, untuk menjaga kedaulatan dan integritas wilayah NKRI, serta mengamankan Alur Laut Kepulauan Indonesia (ALKI) dan chokepoints maritim dari berbagai ancaman.
  \item Mengoptimalkan implementasi doktrin Sistem Pertahanan dan Keamanan Rakyat Semesta (Sishankamrata) dengan melibatkan partisipasi aktif seluruh komponen bangsa dalam upaya bela negara.
  \item Meningkatkan kemampuan keamanan siber nasional untuk melindungi infrastruktur kritis, kedaulatan data, dan privasi warga negara dari ancaman siber.
\end{enumerate}

\subsubsection*{Bidang Diplomasi dan Hubungan Internasional}
\begin{enumerate}
  \item Memanfaatkan Wawasan Nusantara sebagai landasan yang kokoh dalam menjalankan politik luar negeri bebas aktif, memperjuangkan kepentingan nasional Indonesia di forum-forum regional dan internasional, serta berkontribusi dalam penyelesaian isu-isu global.
  \item Mengintensifkan diplomasi maritim untuk memperkuat kerjasama internasional dalam pengelolaan sumber daya laut yang berkelanjutan, penanganan kejahatan lintas negara di laut, dan pemeliharaan keamanan serta stabilitas kawasan maritim.
  \item Mempromosikan nilai-nilai Wawasan Nusantara, seperti harmoni dalam keberagaman dan penyelesaian sengketa secara damai, sebagai bagian dari kontribusi Indonesia bagi perdamaian dunia.
\end{enumerate}
\subsubsection*{Bidang Pemberdayaan Masyarakat dan Peran Media}
\begin{enumerate}
  \item Mendorong partisipasi aktif masyarakat sipil, organisasi kemasyarakatan, dan komunitas lokal dalam berbagai program penguatan Wawasan Nusantara, termasuk dalam pelestarian budaya, penjagaan lingkungan, dan pembangunan daerah.
  \item Bekerjasama dengan media massa dan para kreator konten digital untuk memproduksi dan menyebarluaskan narasi-narasi positif tentang persatuan, keberagaman, dan potensi Indonesia sebagai negara kepulauan, serta menangkal disinformasi dan ujaran kebencian yang dapat memecah belah bangsa.
  \item Meningkatkan literasi digital masyarakat agar mampu memanfaatkan teknologi informasi secara bijak dan bertanggung jawab, serta kritis terhadap konten-konten negatif.
\end{enumerate}
% References
 \bibliographystyle{apacite}
 \bibliography{references}

\end{document}
