% Preamble
\documentclass[12pt, a4paper]{article}
\usepackage[bahasa]{babel}
\usepackage{times}
\usepackage[utf8]{inputenc}
\usepackage{float}
\usepackage[T1]{fontenc}
\usepackage{graphicx}
\usepackage{caption}
\usepackage[left=3cm, right=3cm, top=3cm, bottom=3cm]{geometry}
\usepackage{setspace}
\usepackage{enumitem}
\usepackage{ragged2e}
\usepackage{xurl}
\usepackage{apacite}
\usepackage{etoolbox}
\usepackage{titlesec}

\usepackage{array}   % Untuk >{\command} dalam spesifikasi kolom tabel
\usepackage{ragged2e} % Untuk \RaggedRight, \Centering, \Justifying
% Jika Anda ingin garis yang lebih profesional (opsional, ganti \hline dengan \toprule, \midrule, \bottomrule)
% \usepackage{booktabs}
% \usepackage{longtable} % Jika tabelnya sangat panjang dan mungkin melewati halaman

% Global line spacing (1.5)
\onehalfspacing


\titleformat{\section}
  {\fontsize{12pt}{14pt}\bfseries\uppercase}
  {\thesection}
  {1em}
  {}
\titleformat{\subsection}
  {\fontsize{12pt}{14pt}\bfseries}
  {\thesection}
  {1em}
  {}
% Custom command for Abstract/Abstrak titles
  \newcommand{\makecustomtitle}[1]{\begin{center}\large\bfseries\fontsize{12pt}{14pt}\MakeUppercase{#1}\end{center}\vspace{0.5cm}}

\begin{document}

% Title Page
\begin{titlepage}
    \centering
    \vspace*{2cm}
    {\LARGE\bfseries Bagaimana dinamika historis dan urgensi wawasan nusantara sebagai konsepsi dan pandangan kolektif kebangsaan indonesia dalam konteks pergaulan dunia\par}
    \vspace{1cm}
    {\large\bfseries Rifqi Fadil Fahrial$^1$\par}
    \vspace{0.5cm}
    \normalsize
    $^1$Sekolah Tinggi Teknologi Bandung (STTB)\par
    \vspace{0.5cm}
    Penulis: rifqifadil57@gmail.com
    \vfill
\end{titlepage}

% ABSTRACT (English)
\makecustomtitle{ABSTRACT}
\begin{minipage}{\textwidth}
\singlespacing
\justifying
\textit{\small
Wawasan Nusantara, or Nusantara's Insight, serves as the Indonesian nation's fundamental outlook in viewing their homeland and collective life, particularly concerning national unity amidst diversity within the global political landscape. This concept has evolved historically, notably since the Djuanda Declaration of 1957 which asserted Indonesia's archipelagic integrity, later gaining international recognition through UNCLOS 1982. According to the \textit{Lembaga Ketahanan Nasional} (National Resilience Institute) in 1999, Wawasan Nusantara is defined as the Indonesian people's perspective and attitude regarding themselves and their diverse, strategically valuable environment, emphasizing national unity and solidarity across regions to achieve national goals. Its urgency lies in safeguarding territorial integrity, fostering national resilience, and guiding Indonesia's engagement with the world. The implementation of Wawasan Nusantara spans political, economic, socio-cultural, and defense-security aspects, ensuring a cohesive approach to national development and sovereignty.}
\vspace{0.5em}
\par\noindent
\textit{Keywords:} wawasan nusantara, geopolitics, indonesia, unity, diversity, archipelagic state, national resilience
\end{minipage}
\vspace{1cm}

% ABSTRAK (Bahasa Indonesia)
\makecustomtitle{ABSTRAK}
\begin{minipage}{\textwidth}
\singlespacing
\justifying
\textit{\small
Wawasan Nusantara merupakan cara pandang fundamental bangsa Indonesia terhadap diri dan lingkungan tempat hidupnya, yang sangat mempengaruhi keberlangsungan dan keberhasilan bangsa mencapai tujuannya. Konsepsi ini memiliki dinamika historis yang penting, bermula dari Deklarasi Djuanda pada tahun 1957 yang memandang wilayah Indonesia sebagai satu kesatuan kepulauan, dan kemudian diperkuat dengan pengakuan internasional melalui UNCLOS 1982. Berdasarkan \textit{Lembaga Ketahanan Nasional} tahun 1999, pengertian Wawasan Nusantara adalah cara pandang dan sikap bangsa Indonesia mengenai diri dan lingkungannya yang serba beragam dan bernilai strategis, dengan mengutamakan persatuan dan kesatuan bangsa serta kesatuan wilayah dalam penyelenggaraan kehidupan bermasyarakat, berbangsa, dan bernegara untuk mencapai tujuan nasional. Urgensi Wawasan Nusantara terletak pada kemampuannya sebagai landasan untuk menjaga keutuhan wilayah, kedaulatan negara, mencegah disintegrasi, serta sebagai pedoman dalam menghadapi tantangan global. Implementasinya terwujud dalam berbagai aspek kehidupan nasional, meliputi politik, ekonomi, sosial budaya, serta pertahanan dan keamanan.}
\vspace{0.5em}
\par\noindent
\textit{Kata Kunci:} wawasan nusantara, geopolitik, indonesia, persatuan, kesatuan, negara kepulauan, ketahanan nasional
\end{minipage}
\vspace{1cm}

% Main Content
\section{PENDAHULUAN}
\setlength{\parindent}{1em}
\justifying

\subsection*{Konseptualisasi Ulang Wawasan Nusantara: Esensi dan Urgensi Kontemporer}

Wawasan Nusantara, sebagai sebuah konsepsi fundamental bagi bangsa Indonesia, merepresentasikan cara pandang bangsa terhadap diri dan lingkungannya yang kompleks dan dinamis. Konsepsi ini secara inheren mengutamakan persatuan dan kesatuan nasional di atas segala kepentingan individu, kelompok, maupun golongan tertentu. Jauh melampaui sekadar pemahaman geografis, Wawasan Nusantara menjelma menjadi sebuah filosofi yang berupaya mengintegrasikan seluruh spektrum kehidupan bangsa, mulai dari aspek ideologi, politik, ekonomi, sosial-budaya, hingga pertahanan dan keamanan (Poleksosbudhankam). Esensi utama dari Wawasan Nusantara adalah pengakuan dan perwujudan kesatuan yang utuh, mencakup kesatuan wilayah yang terdiri dari daratan, lautan, dan ruang udara; kesatuan bangsa yang terdiri dari beragam suku, agama, dan budaya; serta kesatuan dalam berbagai aspek kehidupan bernegara. Dalam kerangka pandang ini, laut tidak lagi dipersepsikan sebagai elemen pemisah antar pulau, melainkan sebagai jembatan penghubung yang menyatukan seluruh gugusan kepulauan Indonesia menjadi satu entitas yang koheren.  

Urgensi Wawasan Nusantara dalam konteks kontemporer menjadi semakin signifikan mengingat kompleksitas tantangan global dan dinamika internal bangsa. Konsepsi ini menyediakan landasan strategis bagi Indonesia dalam menghadapi berbagai perubahan global, menjaga kedaulatan negara secara utuh, serta mengarahkan segenap upaya bangsa untuk mewujudkan tujuan-tujuan nasional yang diamanatkan konstitusi. Dengan demikian, Wawasan Nusantara bukan hanya sekadar warisan historis, melainkan sebuah panduan visioner yang relevan untuk menavigasi masa kini dan masa depan Indonesia.  

\subsection*{Tantangan Global dan Relevansi Wawasan Nusantara bagi Indonesia}

Memasuki era globalisasi, Indonesia dihadapkan pada serangkaian tantangan yang kompleks dan multidimensional. Arus globalisasi membawa serta penetrasi budaya asing yang masif, perkembangan teknologi informasi dan komunikasi yang eksponensial, serta dinamika ekonomi kapitalisme global yang berpotensi menggerus nilai-nilai luhur lokal dan memperlemah sendi-sendi nasionalisme. Lebih lanjut, ancaman terhadap kedaulatan wilayah negara masih menjadi isu krusial, termanifestasi dalam bentuk pelanggaran batas-batas laut, potensi konflik perbatasan, serta berbagai bentuk kejahatan lintas negara di wilayah maritim seperti penangkapan ikan ilegal, tidak dilaporkan, dan tidak diatur (Illegal, Unreported, and Unregulated Fishing - IUU Fishing), perompakan, dan penyelundupan. Isu-isu global lainnya seperti perubahan iklim yang dampaknya sangat dirasakan oleh negara kepulauan seperti Indonesia, serta tantangan dalam menjaga kedaulatan data dan keamanan di ruang siber, turut menambah kompleksitas lanskap tantangan yang dihadapi.  

Dalam menghadapi konstelasi tantangan tersebut, Wawasan Nusantara menunjukkan relevansinya yang tinggi. Konsepsi ini menawarkan kerangka kerja yang komprehensif bagi Indonesia untuk merumuskan strategi nasional yang adaptif dan responsif. Dengan berpegang pada prinsip-prinsip Wawasan Nusantara, Indonesia dapat memperkuat posisi tawarnya dalam pergaulan internasional, melindungi kepentingan nasionalnya secara efektif, dan memberikan kontribusi positif bagi tatanan dunia yang lebih adil dan damai. Relevansi Wawasan Nusantara di era global tidak hanya bersifat defensif, yakni sebagai benteng untuk menjaga keutuhan dan kedaulatan negara, tetapi juga bersifat proaktif. Pemahaman mendalam akan Wawasan Nusantara, yang bertujuan menjaga persatuan dan kedaulatan , memungkinkan Indonesia untuk tidak hanya merespons ancaman seperti sengketa Laut Cina Selatan, IUU fishing, dan perubahan iklim , tetapi juga untuk secara aktif terlibat dalam diplomasi  dan pengelolaan sumber daya yang berkelanjutan. Hal ini menyiratkan bahwa Wawasan Nusantara bukan hanya sebuah konsep introspektif yang berfokus pada urusan internal, melainkan juga sebuah pandangan ekstrovert yang memandu Indonesia untuk memainkan peran aktif di panggung dunia. Kesadaran akan potensi dan posisi strategisnya sebagai negara kepulauan terbesar di dunia menjadi modal penting dalam konstelasi ini. Implikasi yang lebih luas dari pemahaman ini adalah bahwa Wawasan Nusantara dapat menjadi model bagi negara-negara kepulauan lainnya dalam menavigasi kompleksitas global, menunjukkan bahwa kesatuan internal dan pandangan dunia yang koheren merupakan aset strategis yang tak ternilai harganya.  

\section{Lintasan Sejarah dan Evolusi Konsepsi Wawasan Nusantara}
\subsection*{Kondisi Pra-Deklarasi Djuanda: Fragmentasi Wilayah berdasarkan Territoriale Zeeën en Maritieme Kringen Ordonantie 1939 (TZMKO 1939)}

Sebelum lahirnya Deklarasi Djuanda, konsepsi wilayah Negara Republik Indonesia masih mengacu pada peraturan warisan pemerintah kolonial Hindia Belanda, yaitu Territoriale Zeeën en Maritieme Kringen Ordonantie 1939 (TZMKO 1939). Ordonansi ini menetapkan batas laut teritorial Indonesia hanya sejauh 3 mil laut (sekitar 5,556 kilometer, dengan 1 mil laut setara 1,852 kilometer) dari garis pantai setiap pulau secara individual. Ketentuan ini memiliki dampak yang sangat signifikan dan merugikan bagi Indonesia sebagai negara yang baru merdeka dan bercirikan kepulauan.  

Secara eksternal, penetapan batas laut teritorial yang sempit ini menjadikan wilayah perairan Indonesia sangat rentan terhadap intervensi dan aktivitas kapal-kapal asing yang dapat mengancam kedaulatan dan keamanan negara. Secara internal, dampak yang lebih fundamental adalah terjadinya fragmentasi wilayah. Lautan luas yang membentang di antara ribuan pulau di Indonesia, berdasarkan TZMKO 1939, dianggap sebagai perairan bebas atau laut lepas. Akibatnya, laut justru berfungsi sebagai elemen pemisah antar pulau, bukan sebagai penghubung. Hal ini secara diametral bertentangan dengan pandangan hidup dan konsep "Tanah Air" bangsa Indonesia yang memandang daratan dan lautan sebagai satu kesatuan yang tak terpisahkan. Konsekuensi logis dari status laut antar pulau sebagai perairan bebas adalah kapal-kapal asing dapat dengan leluasa melayari wilayah tersebut, bahkan melakukan eksploitasi sumber daya laut tanpa kendali dari negara Indonesia.  

TZMKO 1939, dengan demikian, bukan hanya sekadar sebuah peraturan mengenai batas laut yang diwariskan dari masa kolonial. Lebih dari itu, ia merupakan instrumen geopolitik yang secara inheren berlawanan dengan konsep negara-bangsa Indonesia yang merdeka dan berdaulat atas keseluruhan wilayah nusantara. Dampak dari ordonansi ini melampaui sekadar fragmentasi fisik wilayah; ia juga menciptakan fragmentasi psikologis dan politis yang menghambat proses pembentukan identitas kolektif bangsa yang memandang laut sebagai unsur pemersatu. Warisan kolonial ini secara fundamental tidak mempertimbangkan karakteristik unik Indonesia sebagai negara kepulauan. Laut yang seharusnya menjadi urat nadi penghubung justru menjadi "laut bebas" yang memecah belah kesatuan bangsa. Pertentangan mendasar dengan konsep "Tanah Air" yang menyatukan daratan dan lautan  menjadi pemicu utama bagi para pemimpin bangsa untuk mencari terobosan hukum demi kedaulatan yang utuh. Oleh karena itu, perjuangan untuk menggantikan TZMKO 1939 melalui Deklarasi Djuanda tidak dapat dipandang semata-mata sebagai upaya ekspansi teritorial, melainkan sebuah revolusi konseptual yang mendasar tentang kedaulatan dan identitas nasional. Peristiwa ini menggarisbawahi bagaimana sebuah produk hukum warisan kolonial dapat menjadi penghalang signifikan bagi realisasi aspirasi nasional, dan bagaimana perubahan hukum yang berani menjadi langkah krusial dalam proses dekolonisasi dan pembangunan bangsa (nation-building).  

\subsection*{Deklarasi Djuanda 1957: Proklamasi Kesatuan Wilayah Maritim dan Implikasinya}

Menyadari dampak negatif TZMKO 1939 yang memecah belah kedaulatan dan tidak selaras dengan karakteristik Indonesia sebagai negara kepulauan, Pemerintah Indonesia mengambil langkah historis. Pada tanggal 13 Desember 1957, Perdana Menteri Djuanda Kartawidjaja mengumumkan sebuah deklarasi yang kemudian dikenal sebagai Deklarasi Djuanda. Latar belakang utama deklarasi ini adalah kesadaran bahwa TZMKO 1939 telah menyebabkan wilayah Indonesia terfragmentasi, dengan pulau-pulau yang dipisahkan oleh perairan internasional yang dapat dengan bebas dilayari oleh kapal-kapal asing. Indonesia, yang terdiri dari kurang lebih 17.000 pulau, memiliki kebutuhan mendesak untuk menjaga kesatuan wilayah dan sistem pertahanannya.  

Isi pokok Deklarasi Djuanda secara fundamental mengubah cara pandang terhadap wilayah perairan Indonesia. Pertama, Indonesia secara tegas menyatakan dirinya sebagai negara kepulauan (Archipelagic State) yang memiliki corak dan karakteristik tersendiri. Kedua, deklarasi ini menyatakan bahwa segala perairan di sekitar, di antara, dan yang menghubungkan pulau-pulau atau bagian pulau-pulau yang termasuk dalam daratan Negara Republik Indonesia, tanpa memandang luas atau lebarnya, adalah bagian yang wajar dari wilayah daratan Negara Republik Indonesia. Dengan demikian, perairan tersebut merupakan bagian dari perairan nasional yang berada di bawah kedaulatan mutlak Negara Republik Indonesia. Konsepsi ini mengubah paradigma bahwa laut bukan lagi sebagai pemisah, melainkan sebagai pemersatu bangsa dan wilayah. Ketiga, meskipun menegaskan kedaulatan mutlak, Deklarasi Djuanda tetap menjamin hak lalu lintas damai bagi kapal-kapal asing di perairan pedalaman (perairan kepulauan) Indonesia, selama lalu lintas tersebut tidak bertentangan dengan atau mengganggu kedaulatan dan keselamatan negara Indonesia.  

Tujuan utama dari Deklarasi Djuanda adalah, pertama, untuk mewujudkan bentuk wilayah Kesatuan Republik Indonesia yang utuh dan bulat, tidak lagi terpecah-belah oleh laut bebas. Kedua, untuk menentukan batas-batas wilayah Negara Kesatuan Republik Indonesia (NKRI) secara jelas dan sesuai dengan asas negara kepulauan. Ketiga, untuk mengatur lalu lintas damai pelayaran internasional yang melalui perairan Indonesia demi menjamin keamanan dan keselamatan NKRI.  

Implikasi dari Deklarasi Djuanda sangatlah luas. Secara teritorial, luas wilayah laut Indonesia mengalami peningkatan yang sangat signifikan. Berbagai sumber menyebutkan peningkatan dari sekitar 2.027.087 km² menjadi 5.193.250 km² , atau dari sekitar 2 juta km² (hanya daratan) menjadi sekitar 5,1 juta km² (meliputi daratan dan lautan), dengan penambahan luas laut teritorial sekitar 0,3 juta km² dan perairan nusantara (kepulauan) sekitar 2,8 juta km². Seluruh sumber daya alam yang terkandung di dalam wilayah perairan yang baru ini secara otomatis menjadi milik dan berada di bawah kedaulatan Indonesia. Namun, deklarasi ini pada awalnya tidak serta merta diterima oleh komunitas internasional. Beberapa negara, terutama negara-negara maritim besar seperti Amerika Serikat, Australia, Inggris, Belanda, dan Selandia Baru, menyatakan pertentangannya. Di sisi lain, Indonesia mendapatkan dukungan dari negara-negara seperti Uni Soviet dan Republik Rakyat Tiongkok.  

Deklarasi Djuanda pada hakikatnya adalah sebuah tindakan politik kedaulatan (political act of sovereignty) yang sangat berani dan visioner pada masanya. Deklarasi ini tidak hanya mengubah peta fisik Indonesia dengan memperluas wilayah kedaulatannya secara signifikan, tetapi juga, yang lebih penting, mengubah peta mental bangsa Indonesia. Ini merupakan penegasan identitas maritim bangsa yang kemudian menjadi landasan fundamental bagi seluruh perjuangan diplomatik Indonesia di kancah internasional untuk mendapatkan pengakuan atas konsep negara kepulauan. Meskipun menghadapi tantangan dan penolakan awal dari beberapa negara kuat , keteguhan Indonesia dalam mempertahankan prinsip yang terkandung dalam Deklarasi Djuanda menunjukkan bahwa sebuah negara berkembang dapat secara proaktif menantang norma-norma hukum internasional yang dianggap tidak adil atau tidak sesuai dengan kondisi geografis serta kepentingan nasionalnya. Deklarasi Djuanda menjadi preseden penting dalam evolusi hukum laut internasional, yang salah satunya dipicu oleh aspirasi dan perjuangan negara-negara Dunia Ketiga. Keberanian politik ini juga menjadi modal penting di dalam negeri untuk membangun dan memperkokoh nasionalisme yang berbasis pada kesatuan maritim.  
\subsection*{Perjuangan Diplomasi Indonesia dalam Konferensi Hukum Laut PBB (UNCLOS I, II, dan III): Argumen, Tantangan, dan Peran Tokoh Kunci (Mochtar Kusumaatmadja, Djuanda Kartawidjaja)}

Perjuangan untuk mendapatkan pengakuan internasional atas konsepsi negara kepulauan yang telah dicanangkan melalui Deklarasi Djuanda merupakan sebuah proses panjang dan penuh tantangan. Indonesia secara aktif membawa aspirasi ini ke forum-forum internasional, terutama dalam rangkaian Konferensi Perserikatan Bangsa-Bangsa tentang Hukum Laut (UNCLOS).

Pada UNCLOS I (Jenewa, 1958) dan UNCLOS II (Jenewa, 1960), Indonesia, seringkali bersinergi dengan negara-negara kepulauan lain seperti Filipina dan didukung oleh beberapa negara lain seperti Yugoslavia, secara konsisten memperjuangkan pengakuan atas konsep negara kepulauan. Argumen utama yang dikedepankan oleh delegasi Indonesia adalah pentingnya kesatuan geografis, politik, ekonomi, dan historis dari kepulauan nusantara. Laut, dalam pandangan Indonesia, bukanlah elemen pemisah, melainkan justru penghubung yang mengintegrasikan ribuan pulau menjadi satu kesatuan wilayah yang utuh. Prof. Mochtar Kusumaatmadja, salah satu tokoh sentral dalam perjuangan ini, bahkan mengemukakan argumen linguistik bahwa istilah "archipelago" (kepulauan) tidak hanya bermakna "gugusan pulau-pulau" tetapi juga menyiratkan "laut yang ditaburi oleh banyak pulau". Aspek keamanan nasional dan integritas teritorial juga menjadi argumen kunci yang disampaikan untuk meyakinkan komunitas internasional akan urgensi pengakuan konsep ini bagi Indonesia.  

Namun, perjuangan di UNCLOS I dan II menghadapi tantangan yang berat. Konsep negara kepulauan dianggap sebagai gagasan baru yang radikal, terutama oleh negara-negara maritim besar. Mereka mengkhawatirkan bahwa pengakuan konsep ini akan membatasi prinsip kebebasan pelayaran (freedom of navigation) di laut lepas dan mempersulit akses terhadap sumber daya alam di perairan yang sebelumnya dianggap bebas. Konvensi-konvensi Jenewa 1958 yang dihasilkan dari UNCLOS I dinilai tidak memuaskan bagi negara-negara kepulauan seperti Indonesia, karena jika diterapkan secara kaku, akan tetap menciptakan "kantong-kantong laut bebas" di antara pulau-pulau dalam satu negara kepulauan. Selain itu, kurangnya data survei teknis yang komprehensif juga menjadi salah satu kendala yang dihadapi Indonesia pada UNCLOS II. Akibatnya, konsep negara kepulauan belum berhasil mendapatkan pengakuan formal dalam kedua konferensi tersebut.  

Titik balik perjuangan Indonesia terjadi pada UNCLOS III (1973-1982). Dalam konferensi yang berlangsung selama hampir satu dekade ini, Indonesia, dengan kepemimpinan intelektual dan diplomatik Prof. Mochtar Kusumaatmadja, terus memperjuangkan konsep negara kepulauan dengan argumen yang lebih matang, komprehensif, dan didukung oleh data yang lebih kuat. Argumen Indonesia tidak hanya bertumpu pada aspek geografis semata, tetapi juga menggabungkan dimensi historis, ekonomis, dan keamanan secara terpadu. Prof. Mochtar Kusumaatmadja memainkan peran sentral sebagai arsitek utama di balik konsepsi negara kepulauan Indonesia. Beliau tidak hanya memimpin delegasi Indonesia dalam perundingan-perundingan yang alot, tetapi juga secara brilian mengembangkan argumen-argumen hukum dan politik yang meyakinkan untuk mendukung posisi Indonesia. Sementara itu, Perdana Menteri Djuanda Kartawidjaja, melalui Deklarasi Djuanda 1957, telah memberikan landasan politik tingkat tinggi yang sangat krusial bagi perjuangan diplomatik ini. Dukungan yang semakin solid dari negara-negara berkembang lainnya yang memiliki kepentingan serupa juga menjadi faktor penting yang turut mengantarkan keberhasilan Indonesia dalam UNCLOS III.  

Kegigihan diplomasi Indonesia, yang dilandasi oleh argumen multidimensional dan kepemimpinan yang visioner, pada akhirnya menjadi faktor penentu dalam mengubah norma hukum laut internasional yang telah mapan. Penerimaan konsep negara kepulauan dalam UNCLOS 1982 bukanlah sekadar kemenangan legalistik semata, melainkan sebuah kemenangan politik dan konseptual yang menunjukkan kapasitas negara berkembang untuk turut serta membentuk tatanan global. Perjuangan panjang Indonesia di forum UNCLOS ini menjadi bukti nyata bahwa hukum internasional bukanlah entitas yang statis dan didominasi secara eksklusif oleh kepentingan negara-negara besar. Negara-negara berkembang, dengan modal argumen yang kuat, data yang valid, dan diplomasi yang persisten, terbukti mampu memainkan peran signifikan dalam evolusi hukum internasional agar lebih mencerminkan prinsip keadilan dan mengakomodasi kebutuhan beragam negara di dunia. Ini adalah contoh klasik keberhasilan penggunaan soft power dan diplomasi multilateral dalam mencapai kepentingan nasional yang fundamental.
\subsection*{Pengakuan Internasional melalui UNCLOS 1982: Konsekuensi Yuridis bagi Indonesia sebagai Negara Kepulauan}
Puncak dari perjuangan panjang diplomasi Indonesia adalah diterimanya konsep negara kepulauan secara resmi dalam Konvensi Perserikatan Bangsa-Bangsa tentang Hukum Laut tahun 1982 (UNCLOS 1982), yang ditandatangani di Montego Bay, Jamaika. Pengakuan ini tertuang secara komprehensif dalam Bab IV UNCLOS 1982, yang secara khusus mengatur tentang Negara Kepulauan (Pasal 46-54).  

Pasal 46 UNCLOS 1982 memberikan definisi yuridis mengenai "negara kepulauan" (archipelagic state) dan "kepulauan" (archipelago). Berdasarkan ketentuan ini, negara kepulauan berhak menarik garis pangkal lurus kepulauan (archipelagic baselines) yang menghubungkan titik-titik terluar dari pulau-pulau terluar dan karang kering terluar dari kepulauan tersebut. Perairan yang terletak di dalam garis pangkal lurus kepulauan ini kemudian diakui statusnya sebagai perairan kepulauan (archipelagic waters), yang berada di bawah kedaulatan penuh negara kepulauan yang bersangkutan.  

Konsekuensi yuridis dari pengakuan ini bagi Indonesia sangatlah fundamental. Pertama, terjadi perluasan signifikan atas wilayah laut yang diakui kedaulatannya secara internasional. Laut teritorial Indonesia ditetapkan selebar 12 mil laut, diukur dari garis pangkal lurus kepulauan yang baru. Kedua, Indonesia berhak menetapkan dan memanfaatkan Zona Ekonomi Eksklusif (ZEE) sejauh 200 mil laut dari garis pangkal kepulauannya. Luas ZEE Indonesia diperkirakan mencapai sekitar 2,7 juta kilometer persegi. Ketiga, Indonesia juga memiliki hak atas landas kontinen, yang dalam kondisi geologis tertentu dapat melebihi batas 200 mil laut. Indonesia bahkan telah mengajukan submisi untuk perluasan landas kontinen di beberapa wilayah perairannya. Keempat, pengakuan ini menegaskan kedaulatan penuh Indonesia atas seluruh perairan kepulauan beserta sumber daya alam yang terkandung di dalamnya, baik hayati maupun non-hayati, di dasar laut maupun di kolom airnya.  

Namun demikian, status sebagai negara kepulauan juga membawa sejumlah kewajiban internasional. Salah satu kewajiban utama adalah menetapkan Alur Laut Kepulauan Indonesia (ALKI) bagi pelayaran internasional. Selain itu, Indonesia juga wajib menghormati hak lintas damai (innocent passage) kapal-kapal asing melalui perairan kepulauannya, serta hak lintas transit (transit passage) melalui selat-selat yang digunakan untuk pelayaran internasional yang terdapat di dalam wilayah perairan kepulauannya. UNCLOS 1982 sendiri mulai berlaku secara efektif sebagai hukum internasional positif sejak tanggal 16 November 1994.  

Pengakuan internasional melalui UNCLOS 1982 bukanlah merupakan titik akhir dari perjuangan Indonesia, melainkan justru menjadi awal dari babak baru yang penuh tantangan. Indonesia kini dihadapkan pada tugas besar untuk mampu mengelola dan mengamankan wilayah maritimnya yang sangat luas secara efektif dan bertanggung jawab. Lebih jauh, Indonesia harus mampu menyeimbangkan pelaksanaan hak-hak kedaulatannya sebagai negara kepulauan dengan pemenuhan kewajiban-kewajiban internasional yang melekat pada status tersebut. Ini adalah sebuah tantangan implementasi yang kompleks, yang menuntut kapasitas kelembagaan, sumber daya, teknologi, dan diplomasi yang mumpuni. Keberhasilan Indonesia dalam mengimplementasikan seluruh ketentuan UNCLOS 1982 tidak hanya akan menentukan masa depan kemaritiman bangsa, tetapi juga akan menjadi tolok ukur kapasitas negara Indonesia sebagai negara maritim yang bertanggung jawab di mata komunitas internasional. Pengakuan ini menempatkan Indonesia pada posisi sentral dalam geopolitik maritim global, namun sekaligus meningkatkan tanggung jawabnya dalam menjaga keamanan pelayaran, melakukan konservasi lingkungan laut, dan memastikan pengelolaan sumber daya alam yang berkelanjutan untuk sebesar-besar kemakmuran rakyat.
2.5. Formalisasi dalam Hukum Nasional: Dari PP No. 4/1960 hingga UU No. 17 Tahun 1985 dan Peraturan Terkait Lainnya

Sejalan dengan perjuangan di kancah internasional, Indonesia juga melakukan serangkaian langkah formalisasi konsepsi Wawasan Nusantara ke dalam sistem hukum nasionalnya. Langkah awal yang krusial adalah pengukuhan Deklarasi Djuanda 1957 melalui Peraturan Pemerintah Pengganti Undang-Undang (Perppu) Nomor 4 Tahun 1960 tentang Perairan Indonesia. Perppu ini kemudian ditetapkan menjadi Undang-Undang Nomor 4/Prp/1960. Undang-undang ini secara yuridis menetapkan prinsip-prinsip utama Deklarasi Djuanda, yaitu penarikan garis pangkal lurus kepulauan untuk menentukan batas wilayah perairan Indonesia, penetapan lebar laut teritorial sejauh 12 mil laut yang diukur dari garis pangkal lurus tersebut, dan penjaminan hak lintas damai bagi kapal-kapal asing melalui perairan kepulauan Indonesia.  

Setelah keberhasilan monumental dalam UNCLOS III yang menghasilkan Konvensi Hukum Laut PBB 1982, Indonesia segera mengambil langkah untuk meratifikasi konvensi tersebut ke dalam hukum nasionalnya. Hal ini diwujudkan melalui penerbitan Undang-Undang Nomor 17 Tahun 1985 tentang Pengesahan United Nations Convention on the Law of the Sea (Konvensi Perserikatan Bangsa-Bangsa tentang Hukum Laut). Dengan undang-undang ini, seluruh ketentuan UNCLOS 1982, termasuk pengakuan terhadap rezim negara kepulauan, secara resmi menjadi bagian integral dari sistem hukum positif Indonesia. Penjelasan Undang-Undang Nomor 17 Tahun 1985 secara rinci menguraikan berbagai rezim hukum laut yang diatur dalam UNCLOS 1982, seperti ketentuan mengenai Negara Kepulauan, Laut Teritorial, Zona Tambahan, Zona Ekonomi Eksklusif (ZEE), Landas Kontinen, Laut Lepas, serta mekanisme penyelesaian sengketa kelautan. Penting untuk dicatat bahwa sebagai negara kepulauan, Indonesia juga memiliki kewajiban untuk menghormati hak-hak tradisional penangkapan ikan yang dimiliki oleh negara-negara tetangga yang berbatasan langsung, serta melindungi keberadaan kabel-kabel laut yang telah ada sebelumnya di bagian-bagian tertentu perairan kepulauan yang dulunya merupakan laut lepas.  

Seiring dengan perkembangan dan dinamika pasca-UNCLOS 1982, Undang-Undang Nomor 4/Prp/1960 tentang Perairan Indonesia dinilai tidak lagi sepenuhnya sesuai dengan rezim hukum negara kepulauan yang telah diakui secara internasional. Oleh karena itu, pemerintah Indonesia kemudian menerbitkan Undang-Undang Nomor 6 Tahun 1996 tentang Perairan Indonesia, yang secara komprehensif menggantikan dan menyempurnakan ketentuan undang-undang sebelumnya agar selaras dengan UNCLOS 1982. Lebih lanjut, untuk mengatur implementasi hak lintas alur laut kepulauan, pemerintah juga mengeluarkan Peraturan Pemerintah Nomor 37 Tahun 2002 tentang Hak dan Kewajiban Kapal dan Pesawat Udara Asing dalam Melaksanakan Hak Lintas Alur Laut Kepulauan Melalui Alur Laut Kepulauan yang Ditetapkan.  

Rangkaian proses formalisasi Wawasan Nusantara ke dalam berbagai produk hukum nasional ini menunjukkan komitmen kuat Indonesia untuk tidak hanya memperjuangkan pengakuan konsep negara kepulauan di tingkat internasional, tetapi juga untuk mengimplementasikannya secara konsisten dan sistematis dalam tata kelola domestik. Proses ini mencerminkan sebuah dialektika yang dinamis antara perkembangan hukum internasional dan pembentukan hukum nasional. Serangkaian legislasi ini, mulai dari UU No. 4/Prp/1960 hingga peraturan-peraturan turunannya, menandai evolusi pemahaman dan penerapan Wawasan Nusantara oleh negara seiring dengan perkembangan rezim hukum laut internasional. Meskipun demikian, tantangan signifikan tetap ada, terutama dalam hal harmonisasi antar berbagai peraturan perundang-undangan yang ada, sinkronisasi kebijakan antar lembaga terkait, serta efektivitas implementasi dan penegakan hukum di lapangan untuk mewujudkan cita-cita Wawasan Nusantara secara paripurna.
% TODO:tabel 1 desuwa
\begin{table}[H]
\centering
\caption{Evolusi Yuridis Wilayah Laut Indonesia dan Konsepsi Wawasan Nusantara}
\label{tab:evolusi_wilayah_laut}
\begin{tabular}{|>{\RaggedRight}p{0.22\textwidth}|>{\RaggedRight}p{0.28\textwidth}|>{\RaggedRight}p{0.2\textwidth}|>{\RaggedRight}p{0.22\textwidth}|}
\hline
\textbf{Periode/Regulasi Kunci} & \textbf{Prinsip Utama Penentuan Wilayah Laut} & \textbf{Luas Perairan (Estimasi/ Perubahan Signifikan)} & \textbf{Implikasi Terhadap Konsepsi Wawasan Nusantara} \\
\hline
Territoriale Zeeën en Maritieme Kringen Ordonantie 1939 (TZMKO 1939) & Laut teritorial 3 mil laut dari garis pantai setiap pulau secara terpisah. & Wilayah laut terfragmentasi, laut antar pulau berstatus laut bebas. & Konsep kesatuan wilayah Nusantara belum terwujud secara yuridis; laut sebagai pemisah. \\
\hline
Deklarasi Djuanda 1957 \& UU No. 4/Prp/1960 & Penarikan garis pangkal lurus kepulauan; laut teritorial 12 mil dari garis pangkal; perairan di dalam garis pangkal menjadi perairan pedalaman/nasional. & Penambahan signifikan luas kedaulatan perairan Indonesia, menyatukan wilayah laut antar pulau. &Peletakan dasar yuridis awal bagi Wawasan Nusantara sebagai konsepsi kesatuan wilayah darat dan laut. \\
\hline
UNCLOS I (1958) \& UNCLOS II (1960) & Perjuangan Indonesia untuk pengakuan internasional atas konsep negara kepulauan. & Status quo, konsep negara kepulauan belum diterima secara formal. & Upaya diplomatik intensif untuk melegitimasi konsepsi Wawasan Nusantara di forum internasional. \\
\hline
UNCLOS 1982 \& UU No. 17 Tahun 1985 tentang Pengesahan UNCLOS & Pengakuan internasional atas konsep negara kepulauan; hak atas laut teritorial 12 mil, ZEE 200 mil, dan landas kontinen. & Penambahan luas ZEE sekitar 2,7 juta km$^2$; total luas perairan yurisdiksi Indonesia menjadi sekitar 5,8 juta km$^2$. & Legitimasi internasional penuh bagi Wawasan Nusantara sebagai negara kepulauan; landasan kuat untuk pengelolaan sumber daya laut. \\
\hline
UU No. 6 Tahun 1996 tentang Perairan Indonesia & Penyesuaian hukum nasional mengenai perairan Indonesia dengan ketentuan UNCLOS 1982, menggantikan UU No. 4/Prp/1960. & Penegasan dan perincian rezim-rezim perairan Indonesia sesuai UNCLOS 1982. & Penguatan kerangka hukum nasional untuk implementasi Wawasan Nusantara dalam tata kelola maritim. \\
\hline
\end{tabular}
\end{table}


Tabel \figurename~\ref{tab:evolusi_wilayah_laut} ini secara visual menggambarkan transisi fundamental dari rezim kolonial yang memecah-belah menuju rezim nasional yang menyatukan, yang kemudian mendapatkan pengakuan dan legitimasi internasional. Dengan menyandingkan regulasi kunci, prinsip penentuan wilayah laut, perubahan signifikan luas perairan, dan implikasinya terhadap Wawasan Nusantara, tabel ini berfungsi sebagai alat bantu yang efektif untuk memahami evolusi konsep tersebut secara yuridis dan teritorial. Hal ini memperkuat argumen mengenai dinamika historis Wawasan Nusantara dan menunjukkan bagaimana perjuangan di bidang hukum telah secara signifikan membentuk konsepsi kebangsaan Indonesia modern.
\section{Wawasan Nusantara sebagai Pandangan Hidup dan Konsepsi Kolektif Kebangsaan}
\subsection*{Landasan Filosofis: Keterkaitan dengan Pancasila dan Bhinneka Tunggal Ika}
Wawasan Nusantara menjadi konsep politik kenegaraan untuk mempertahankan kesatuan wilayah dan bangsa. Letak geografis Indonesia yang strategis, yaitu di antara dua benua (Asia dan Australia) dan dua samudra (Hindia dan Pasifik), juga memengaruhi pandangan geopolitik Indonesia.

Wawasan Nusantara tidak lahir dalam ruang hampa, melainkan berakar kuat pada falsafah dasar negara Indonesia, yaitu Pancasila, dan semboyan pemersatu bangsa, Bhinneka Tunggal Ika. Keterkaitan ini bersifat fundamental dan memberikan landasan nilai yang kokoh bagi Wawasan Nusantara sebagai pandangan hidup dan konsepsi kolektif kebangsaan.

Pancasila, sebagai ideologi dan falsafah negara, menyediakan kerangka nilai yang menjiwai Wawasan Nusantara. Prinsip-prinsip seperti Persatuan Indonesia, Kemanusiaan yang Adil dan Beradab, Kerakyatan yang Dipimpin oleh Hikmat Kebijaksanaan dalam Permusyawaratan/Perwakilan, dan Keadilan Sosial bagi Seluruh Rakyat Indonesia, secara inheren termanifestasi dalam konsepsi Wawasan Nusantara. Wawasan Nusantara, yang mengedepankan kesatuan wilayah, bangsa, politik, ekonomi, sosial budaya, dan pertahanan keamanan, merupakan pengejawantahan dari sila Persatuan Indonesia. Upaya untuk mewujudkan kesejahteraan yang merata di seluruh pelosok nusantara dan pengelolaan sumber daya alam untuk sebesar-besar kemakmuran rakyat, sebagaimana diamanatkan oleh Wawasan Nusantara, adalah cerminan dari sila Keadilan Sosial. Demikian pula, penghormatan terhadap hak asasi manusia dan pengambilan keputusan melalui musyawarah mufakat dalam pengelolaan negara kepulauan ini selaras dengan nilai-nilai Pancasila lainnya. Pancasila memberikan nilai-nilai dasar berupa toleransi, keadilan, persatuan, dan demokrasi yang esensial dalam menjaga kerukunan dan keutuhan bangsa yang majemuk, yang menjadi inti dari Wawasan Nusantara.  

Selanjutnya, semboyan Bhinneka Tunggal Ika, yang berarti "Berbeda-beda tetapi tetap satu jua," menjadi landasan sosiokultural bagi Wawasan Nusantara. Indonesia adalah bangsa yang luar biasa majemuk, terdiri dari ratusan suku bangsa, beragam agama dan kepercayaan, serta ribuan bahasa dan budaya lokal. Wawasan Nusantara secara sadar mengakui dan merangkul keberagaman ini bukan sebagai sumber perpecahan, melainkan sebagai kekayaan dan kekuatan bangsa yang disatukan dalam wadah Negara Kesatuan Republik Indonesia. Konsep kesatuan dalam Wawasan Nusantara tidak berarti penyeragaman, melainkan harmoni dalam keragaman, di mana setiap unsur budaya lokal memiliki ruang untuk tumbuh dan berkembang, serta berkontribusi pada pembentukan budaya nasional yang dinamis. Bhinneka Tunggal Ika menyediakan perekat filosofis yang memungkinkan Wawasan Nusantara untuk mengintegrasikan berbagai perbedaan tersebut menjadi satu pandangan kolektif kebangsaan.  

Dengan demikian, Wawasan Nusantara dapat dipahami bukan hanya sebagai sebuah konsep geopolitik atau strategi kewilayahan semata. Lebih dari itu, ia adalah manifestasi operasional dari nilai-nilai Pancasila dan semangat Bhinneka Tunggal Ika dalam konteks spesifik Indonesia sebagai negara kepulauan. Wawasan Nusantara menerjemahkan nilai-nilai filosofis yang abstrak tersebut menjadi panduan yang lebih konkret dalam penyelenggaraan kehidupan berbangsa dan bernegara. Ia memberikan "wadah" spasial dan konseptual bagi pengejawantahan nilai-nilai luhur Pancasila, seperti persatuan dan keadilan, serta semangat Bhinneka Tunggal Ika yang menghargai perbedaan dalam bingkai kesatuan. Kegagalan dalam memahami atau mengimplementasikan Wawasan Nusantara secara efektif berpotensi melemahkan penghayatan terhadap Pancasila dan Bhinneka Tunggal Ika itu sendiri. Hal ini disebabkan karena konsep kesatuan geografis, politik, ekonomi, sosial-budaya, dan pertahanan keamanan yang diusung Wawasan Nusantara merupakan arena nyata di mana nilai-nilai fundamental bangsa tersebut diuji, diaktualisasikan, dan diwujudkan dalam kehidupan sehari-hari.

\subsection*{Dimensi Multiaspek Wawasan Nusantara: Politik, Ekonomi, Sosial Budaya, Pertahanan dan Keamanan (Hankam)}

Wawasan Nusantara sebagai pandangan kolektif kebangsaan Indonesia mencakup seluruh aspek kehidupan nasional secara terpadu dan menyeluruh. Perwujudan kepulauan Nusantara sebagai satu kesatuan tidak hanya terbatas pada aspek geografis, tetapi juga meliputi kesatuan dalam bidang politik, ekonomi, sosial budaya, serta pertahanan dan keamanan (Poleksosbudhankam).  

Dalam dimensi politik, Wawasan Nusantara memandang seluruh wilayah Nusantara, beserta segenap kekayaan alam dan penduduknya, sebagai satu kesatuan politik yang utuh. Ini berarti bahwa Pancasila adalah satu-satunya falsafah serta ideologi bangsa dan negara yang melandasi, membimbing, dan mengarahkan bangsa menuju tujuannya. Seluruh kepulauan Nusantara merupakan satu kesatuan hukum nasional, di mana hanya ada satu hukum nasional yang mengabdi kepada kepentingan nasional. Kebijakan pemerintah, baik di tingkat pusat maupun daerah, harus terpadu dan diarahkan untuk kepentingan seluruh bangsa.  

Dalam dimensi ekonomi, Wawasan Nusantara menegaskan bahwa kekayaan wilayah Nusantara, baik yang potensial maupun yang efektif, adalah modal dan milik bersama bangsa. Keperluan hidup sehari-hari harus tersedia secara merata di seluruh wilayah tanah air. Tingkat perkembangan ekonomi harus serasi dan seimbang di seluruh daerah, tanpa meninggalkan ciri khas yang dimiliki oleh daerah-daerah dalam mengembangkan kehidupan ekonominya. Perekonomian yang diselenggarakan sebagai usaha bersama atas asas kekeluargaan ditujukan sebesar-besarnya untuk kemakmuran rakyat.  

Dalam dimensi sosial budaya, Wawasan Nusantara memandang masyarakat Indonesia sebagai satu kesatuan. Perikehidupan bangsa harus merupakan kehidupan yang serasi dengan terdapatnya tingkat kemajuan masyarakat yang sama, merata, dan seimbang, serta adanya keselarasan kehidupan yang sesuai dengan tingkat kemajuan bangsa. Budaya Indonesia pada hakikatnya adalah satu, sedangkan corak ragam budaya yang ada menggambarkan kekayaan budaya bangsa yang menjadi modal dan landasan pengembangan budaya bangsa seluruhnya, yang hasil-hasilnya dapat dinikmati oleh seluruh bangsa Indonesia.  

Dalam dimensi pertahanan dan keamanan (Hankam), Wawasan Nusantara menggariskan bahwa ancaman terhadap satu pulau atau satu daerah pada hakikatnya merupakan ancaman terhadap seluruh bangsa dan negara. Setiap warga negara mempunyai hak dan kewajiban yang sama dalam rangka pembelaan negara dan bangsa. Seluruh kepulauan Nusantara merupakan satu kesatuan pertahanan dan keamanan.  

Multidimensionalitas Wawasan Nusantara ini menunjukkan bahwa konsep tersebut bersifat holistik, di mana setiap aspek saling terkait dan memengaruhi satu sama lain (interdependen). Kelemahan atau gangguan pada satu aspek kehidupan nasional dapat berdampak negatif dan mengancam stabilitas aspek-aspek lainnya. Sebagai contoh, kesenjangan ekonomi yang tajam antar wilayah (kelemahan pada gatra ekonomi) dapat memicu ketidakpuasan sosial dan gejolak politik (mengganggu gatra sosial dan politik), yang pada gilirannya berpotensi mengancam persatuan bangsa dan stabilitas keamanan nasional (melemahkan gatra hankam). Oleh karena itu, implementasi Wawasan Nusantara menuntut adanya kebijakan nasional yang komprehensif dan terintegrasi. Pendekatan yang parsial atau sektoral, di mana kemajuan di satu bidang mengorbankan atau mengabaikan bidang lainnya, berisiko melemahkan esensi Wawasan Nusantara itu sendiri dan menghambat pencapaian tujuan nasional secara keseluruhan.  

\subsection*{integrasi Wawasan Nusantara dalam Sistem Hukum dan Kebijakan Nasional}

Upaya untuk menerjemahkan konsepsi Wawasan Nusantara ke dalam kerangka regulasi yang operasional dan kebijakan publik yang konkret tercermin dalam berbagai produk perundang-undangan nasional. Meskipun tidak semua undang-undang secara eksplisit menyebutkan frasa "Wawasan Nusantara", prinsip-prinsip fundamental yang terkandung di dalamnya, seperti kesatuan wilayah, kedaulatan negara, pengelolaan sumber daya untuk kemakmuran rakyat, dan pertahanan semesta, terintegrasi secara substansial.

Undang-Undang Nomor 3 Tahun 2002 tentang Pertahanan Negara secara jelas mengadopsi semangat Wawasan Nusantara. Pasal 3 ayat (2) undang-undang ini menyatakan bahwa pertahanan negara disusun dengan memperhatikan kondisi geografis Indonesia sebagai negara kepulauan. Tujuan pertahanan negara, sebagaimana diatur dalam Pasal 4, adalah untuk menjaga dan melindungi kedaulatan negara serta keutuhan wilayah Negara Kesatuan Republik Indonesia (NKRI). Lebih lanjut, Pasal 5 menegaskan bahwa pertahanan negara berfungsi untuk mewujudkan dan mempertahankan seluruh wilayah NKRI sebagai satu kesatuan pertahanan. Penjelasan Pasal 5 bahkan secara eksplisit menyatakan bahwa ancaman terhadap sebagian wilayah merupakan ancaman terhadap seluruh wilayah dan menjadi tanggung jawab segenap bangsa. Penjelasan Umum undang-undang ini juga menyebutkan bahwa pertahanan negara diselenggarakan dengan sistem pertahanan negara yang berketahanan nasional berdasarkan Wawasan Nusantara.  

Undang-Undang Nomor 32 Tahun 2014 tentang Kelautan menjadi salah satu pilar utama dalam implementasi Wawasan Nusantara di sektor maritim. Konsiderans Menimbang huruf b undang-undang ini mengakui bahwa wilayah laut sebagai bagian terbesar dari wilayah Indonesia memiliki posisi dan nilai strategis dari berbagai aspek kehidupan yang mencakup politik, ekonomi, sosial budaya, pertahanan, dan keamanan. Pasal 3 huruf a secara tegas menyatakan bahwa salah satu tujuan penyelenggaraan Kelautan adalah untuk "menegaskan Indonesia sebagai Negara Kepulauan berciri nusantara dan maritim". Pasal 5 ayat (1) memperkuat hal ini dengan menyatakan bahwa Indonesia merupakan negara kepulauan yang seluruhnya terdiri atas kepulauan-kepulauan dan mencakup pulau-pulau besar dan kecil yang merupakan satu kesatuan wilayah, politik, ekonomi, sosial budaya, dan historis. Kedaulatan Indonesia sebagai negara kepulauan meliputi wilayah daratan, perairan pedalaman, perairan kepulauan, dan laut teritorial, termasuk ruang udara di atasnya serta dasar laut dan tanah di bawahnya, beserta kekayaan alam yang terkandung di dalamnya (Pasal 5 ayat (2)). Pembangunan Kelautan, menurut Pasal 13, dilaksanakan sebagai bagian dari pembangunan nasional untuk mewujudkan Indonesia menjadi negara kepulauan yang mandiri, maju, kuat, dan berbasiskan kepentingan nasional.  

Undang-Undang Nomor 26 Tahun 2007 tentang Penataan Ruang juga secara signifikan mengintegrasikan prinsip Wawasan Nusantara. Konsiderans Menimbang huruf a menyatakan bahwa ruang wilayah NKRI yang merupakan negara kepulauan berciri Nusantara, meliputi ruang darat, ruang laut, dan ruang udara, termasuk ruang di dalam bumi, perlu ditingkatkan upaya pengelolaannya secara bijaksana. Konsiderans Menimbang huruf c lebih lanjut menegaskan bahwa penataan ruang bertujuan untuk memperkukuh Ketahanan Nasional berdasarkan Wawasan Nusantara. Pasal 3 undang-undang ini menetapkan bahwa penyelenggaraan penataan ruang bertujuan untuk mewujudkan ruang wilayah nasional yang aman, nyaman, produktif, dan berkelanjutan berlandaskan Wawasan Nusantara dan Ketahanan Nasional. Secara lebih operasional, Pasal 6 ayat (3) menyatakan bahwa penataan ruang wilayah nasional meliputi ruang wilayah yurisdiksi dan wilayah kedaulatan nasional yang mencakup ruang darat, ruang laut, dan ruang udara, termasuk ruang di dalam bumi sebagai satu kesatuan.  

Dalam konteks Undang-Undang Nomor 23 Tahun 2014 tentang Pemerintahan Daerah, meskipun Wawasan Nusantara tidak disebutkan secara eksplisit, prinsip-prinsipnya tercermin dalam pengaturan hubungan antara Pemerintah Pusat dan Daerah. Undang-undang ini diarahkan untuk mencapai tujuan nasional, termasuk melindungi segenap bangsa dan memajukan kesejahteraan umum. Pembagian urusan pemerintahan antara pusat dan daerah, serta peran gubernur sebagai wakil Pemerintah Pusat, menunjukkan upaya menyeimbangkan otonomi daerah dengan tetap menjaga kesatuan nasional. Penataan daerah dan pendekatan asimetris dalam otonomi juga mencerminkan penghargaan terhadap kekhasan daerah dalam bingkai NKRI, yang selaras dengan semangat Wawasan Nusantara yang mengakui keberagaman dalam kesatuan. Harmonisasi Peraturan Daerah dengan peraturan perundang-undangan yang lebih tinggi juga krusial untuk menjaga kesatuan sistem hukum nasional.  

Terakhir, Undang-Undang Nomor 1 Tahun 2014 tentang Perubahan Atas Undang-Undang Nomor 27 Tahun 2007 tentang Pengelolaan Wilayah Pesisir dan Pulau-Pulau Kecil juga mengintegrasikan nilai-nilai Wawasan Nusantara. Penegasan bahwa wilayah pesisir dan pulau-pulau kecil dikuasai oleh negara untuk sebesar-besar kemakmuran rakyat (Menimbang huruf a UU 1/2014)  dan pengelolaan yang dilakukan secara terpadu antarsektor serta antara ekosistem darat dan laut secara berkelanjutan (Pasal 1 angka 1 UU 1/2014)  adalah cerminan dari prinsip kesatuan dan pengelolaan sumber daya untuk kepentingan nasional. Pengakuan dan penghormatan terhadap Masyarakat Hukum Adat serta hak-hak tradisionalnya dalam kerangka NKRI , serta pemanfaatan pulau-pulau kecil berdasarkan kesatuan ekologis dan ekonomis dengan pulau besar di dekatnya , juga selaras dengan Wawasan Nusantara. Bahkan, UU No. 27 Tahun 2007 (sebelum diubah) telah menekankan pengelolaan terpadu antara ekosistem darat dan laut serta pemanfaatan pulau-pulau terluar untuk menjaga kedaulatan NKRI.  

Integrasi Wawasan Nusantara dalam berbagai undang-undang sektoral ini menunjukkan adanya kemauan politik (political will) dari negara untuk mengarusutamakan konsep dasar ini ke dalam kerangka regulasi yang lebih operasional. Namun, tantangan sesungguhnya tidak hanya terletak pada perumusan undang-undang, melainkan pada implementasi yang efektif dan sinkronisasi yang harmonis antar berbagai undang-undang tersebut serta antar lembaga pelaksana di lapangan. Potensi tumpang tindih kewenangan, perbedaan interpretasi, dan kurangnya koordinasi dapat menjadi penghambat dalam mencapai tujuan Wawasan Nusantara secara holistik, meskipun masing-masing undang-undang telah berupaya untuk mengintegrasikannya. Keberhasilan Wawasan Nusantara sebagai panduan bernegara sangat bergantung pada kapasitas tata kelola (governance capacity) negara untuk mengorkestrasi implementasi yang koheren dan sinergis di semua tingkatan pemerintahan dan di semua sektor pembangunan. Tanpa adanya kapasitas tata kelola yang kuat, Wawasan Nusantara berisiko menjadi sekadar retorika yang tertuang indah dalam naskah-naskah hukum, tanpa dampak nyata dalam kehidupan berbangsa dan bernegara.  

Wawasan Nusantara sebagai pandangan kolektif kebangsaan Indonesia mencakup seluruh aspek kehidupan nasional secara terpadu dan menyeluruh. Perwujudan kepulauan Nusantara sebagai satu kesatuan tidak hanya terbatas pada aspek geografis, tetapi juga meliputi kesatuan dalam bidang politik, ekonomi, sosial budaya, serta pertahanan dan keamanan (Poleksosbudhankam).  

Dalam dimensi politik, Wawasan Nusantara memandang seluruh wilayah Nusantara, beserta segenap kekayaan alam dan penduduknya, sebagai satu kesatuan politik yang utuh. Ini berarti bahwa Pancasila adalah satu-satunya falsafah serta ideologi bangsa dan negara yang melandasi, membimbing, dan mengarahkan bangsa menuju tujuannya. Seluruh kepulauan Nusantara merupakan satu kesatuan hukum nasional, di mana hanya ada satu hukum nasional yang mengabdi kepada kepentingan nasional. Kebijakan pemerintah, baik di tingkat pusat maupun daerah, harus terpadu dan diarahkan untuk kepentingan seluruh bangsa.  

Dalam dimensi ekonomi, Wawasan Nusantara menegaskan bahwa kekayaan wilayah Nusantara, baik yang potensial maupun yang efektif, adalah modal dan milik bersama bangsa. Keperluan hidup sehari-hari harus tersedia secara merata di seluruh wilayah tanah air. Tingkat perkembangan ekonomi harus serasi dan seimbang di seluruh daerah, tanpa meninggalkan ciri khas yang dimiliki oleh daerah-daerah dalam mengembangkan kehidupan ekonominya. Perekonomian yang diselenggarakan sebagai usaha bersama atas asas kekeluargaan ditujukan sebesar-besarnya untuk kemakmuran rakyat.  

Dalam dimensi sosial budaya, Wawasan Nusantara memandang masyarakat Indonesia sebagai satu kesatuan. Perikehidupan bangsa harus merupakan kehidupan yang serasi dengan terdapatnya tingkat kemajuan masyarakat yang sama, merata, dan seimbang, serta adanya keselarasan kehidupan yang sesuai dengan tingkat kemajuan bangsa. Budaya Indonesia pada hakikatnya adalah satu, sedangkan corak ragam budaya yang ada menggambarkan kekayaan budaya bangsa yang menjadi modal dan landasan pengembangan budaya bangsa seluruhnya, yang hasil-hasilnya dapat dinikmati oleh seluruh bangsa Indonesia.  

Dalam dimensi pertahanan dan keamanan (Hankam), Wawasan Nusantara menggariskan bahwa ancaman terhadap satu pulau atau satu daerah pada hakikatnya merupakan ancaman terhadap seluruh bangsa dan negara. Setiap warga negara mempunyai hak dan kewajiban yang sama dalam rangka pembelaan negara dan bangsa. Seluruh kepulauan Nusantara merupakan satu kesatuan pertahanan dan keamanan.  

Multidimensionalitas Wawasan Nusantara ini menunjukkan bahwa konsep tersebut bersifat holistik, di mana setiap aspek saling terkait dan memengaruhi satu sama lain (interdependen). Kelemahan atau gangguan pada satu aspek kehidupan nasional dapat berdampak negatif dan mengancam stabilitas aspek-aspek lainnya. Sebagai contoh, kesenjangan ekonomi yang tajam antar wilayah (kelemahan pada gatra ekonomi) dapat memicu ketidakpuasan sosial dan gejolak politik (mengganggu gatra sosial dan politik), yang pada gilirannya berpotensi mengancam persatuan bangsa dan stabilitas keamanan nasional (melemahkan gatra hankam). Oleh karena itu, implementasi Wawasan Nusantara menuntut adanya kebijakan nasional yang komprehensif dan terintegrasi. Pendekatan yang parsial atau sektoral, di mana kemajuan di satu bidang mengorbankan atau mengabaikan bidang lainnya, berisiko melemahkan esensi Wawasan Nusantara itu sendiri dan menghambat pencapaian tujuan nasional secara keseluruhan.  

%TODO: tabel 2 desuwa
\begin{table}[H]
\centering
\caption{Keterkaitan Prinsip Wawasan Nusantara dalam Berbagai Undang-Undang Sektoral}
\label{tab:Keterkaitan_prinsip_wawasan_nusantara}
\begin{tabular}{|>{\RaggedRight}p{0.20\textwidth}|>{\RaggedRight}p{0.25\textwidth}|>{\RaggedRight}p{0.5\textwidth}|}
\hline
\textbf{Periode/Regulasi Kunci} & \textbf{Prinsip Utama Penentuan Wilayah Laut} & \textbf{Luas Perairan (Estimasi/ Perubahan Signifikan)} \\
\hline
UU No. 3 Tahun 2002 tentang Pertahanan Negara &	Pasal 3 ayat (2), Pasal 4, Pasal 5; Penjelasan Umum; Penjelasan Pasal 5 & 	Kesatuan wilayah pertahanan, kondisi geografis negara kepulauan sebagai pertimbangan utama, pertahanan berketahanan nasional berdasarkan Wawasan Nusantara, ancaman terhadap sebagian wilayah adalah ancaman terhadap seluruh wilayah. \\
\hline
UU No. 32 Tahun 2014 tentang Kelautan &	Menimbang huruf b; Pasal 3 huruf a; Pasal 5 ayat (1) \& (2); Pasal 13 &	Penegasan Indonesia sebagai Negara Kepulauan berciri nusantara dan maritim, kesatuan wilayah Poleksosbudhankam, kedaulatan atas seluruh komponen wilayah (darat, laut, udara, dasar laut), pengelolaan sumber daya kelautan untuk kemakmuran bangsa dan kepentingan nasional. \\
\hline
UU No. 26 Tahun 2007 tentang Penataan Ruang &	Menimbang huruf a \& c; Pasal 3; Pasal 6 ayat (3) &	Ruang wilayah NKRI sebagai negara kepulauan berciri Nusantara (kesatuan darat, laut, udara, dalam bumi) dikelola secara bijaksana, penataan ruang untuk memperkukuh Ketahanan Nasional berdasarkan Wawasan Nusantara, mewujudkan ruang wilayah nasional yang berkelanjutan berlandaskan Wawasan Nusantara. \\
\hline
UU No. 23 Tahun 2014 tentang Pemerintahan Daerah &	Abstrak; Pengaturan Hubungan Pusat-Daerah; Pembagian Urusan Pemerintahan &	Keseimbangan antara otonomi daerah dengan kesatuan nasional, pelayanan publik yang merata untuk mencapai tujuan nasional, pengakuan kekhasan daerah dalam bingkai NKRI, harmonisasi peraturan untuk kesatuan hukum. \\ 
\hline
UU No. 1 Tahun 2014 tentang Perubahan Atas UU No. 27 Tahun 2007 tentang Pengelolaan Wilayah Pesisir dan Pulau-Pulau Kecil &	Menimbang huruf a (UU 1/2014); Pasal 1 angka 1 (UU 1/2014); Pengakuan Masyarakat Hukum Adat; Pemanfaatan pulau berdasarkan kesatuan ekologis &	Penguasaan negara atas wilayah pesisir dan pulau-pulau kecil untuk kemakmuran rakyat, pengelolaan terpadu antara ekosistem darat dan laut secara berkelanjutan, pengakuan hak-hak masyarakat adat dalam kerangka NKRI, pemanfaatan pulau terluar untuk menjaga kedaulatan. \\
\hline
\end{tabular}
\end{table}
Tabel ini secara sistematis menunjukkan bagaimana benang merah Wawasan Nusantara terjalin dalam berbagai legislasi penting di Indonesia. Hal ini memperkuat argumen bahwa Wawasan Nusantara bukan hanya sekadar konsep filosofis, tetapi juga menjadi landasan konkret bagi pembentukan hukum dan perumusan kebijakan negara. Visualisasi ini diharapkan dapat membantu pembaca memahami cakupan dan kedalaman pengaruh Wawasan Nusantara dalam sistem hukum Indonesia, serta mengapresiasi upaya negara dalam mengoperasionalkan konsep fundamental ini.
\section{Implementasi dan Relevansi Wawasan Nusantara dalam Pergaulan Dunia}
\subsection*{Pengelolaan Strategis Wilayah Maritim dan Sumber Daya Laut}

Implementasi Wawasan Nusantara dalam konteks pergaulan dunia sangat erat kaitannya dengan pengelolaan strategis wilayah maritim dan sumber daya laut Indonesia. Sebagai negara kepulauan terbesar di dunia, dengan garis pantai yang panjang dan posisi geografis yang vital, Indonesia memiliki tanggung jawab besar sekaligus peluang yang signifikan dalam memanfaatkan dan menjaga kedaulatan maritimnya.
\subsubsection*{Alur Laut Kepulauan Indonesia (ALKI) I, II, III dan Chokepoints Maritim: Signifikansi Geostrategis, Pengelolaan, Manfaat Ekonomi, dan Tantangan Keamanan}

Penetapan Alur Laut Kepulauan Indonesia (ALKI) merupakan salah satu manifestasi konkret dari implementasi Konvensi Hukum Laut PBB 1982 (UNCLOS 1982) oleh Indonesia sebagai negara kepulauan. Terdapat tiga ALKI utama, yaitu ALKI I, ALKI II, dan ALKI III, beserta beberapa cabang dan chokepoints (titik sempit strategis) maritim seperti Selat Malaka, Selat Sunda (bagian dari ALKI I), Selat Lombok, dan Selat Makassar (bagian dari ALKI II). Pengaturan mengenai hak dan kewajiban pelayaran melalui ALKI diatur lebih lanjut dalam Peraturan Pemerintah Nomor 37 Tahun 2002.  

Secara geostrategis, ALKI dan chokepoints ini memiliki signifikansi yang luar biasa. Indonesia berada di persimpangan jalur pelayaran internasional utama yang menghubungkan Samudra Hindia dengan Samudra Pasifik, serta Benua Asia dengan Benua Australia. Jalur-jalur ini vital tidak hanya bagi perdagangan komersial global tetapi juga bagi pergerakan armada militer negara-negara lain. Selat Lombok, misalnya, menjadi jalur penting bagi kapal-kapal supertanker yang tidak dapat melalui Selat Malaka yang lebih dangkal.  

Dari sisi manfaat ekonomi, keberadaan ALKI mendukung kelancaran arus perdagangan barang, baik domestik maupun internasional, membuka potensi pengembangan industri jasa maritim, dan meningkatkan konektivitas antar wilayah di Indonesia maupun dengan dunia luar. Namun, di sisi lain, pengelolaan ALKI dan chokepoints ini menghadirkan tantangan keamanan yang kompleks. Ancaman-ancaman tersebut meliputi potensi pelanggaran kedaulatan, kegiatan spionase oleh pihak asing, praktik IUU fishing, penyelundupan barang ilegal (termasuk narkoba dan senjata), perompakan dan pembajakan di laut, terorisme maritim, pencemaran lingkungan laut akibat kecelakaan kapal atau pembuangan limbah ilegal, hingga potensi terseret dalam konflik regional yang terjadi di sekitar wilayah perairan Indonesia. Pengawasan dan penegakan hukum di sepanjang ALKI yang sangat luas masih menjadi tantangan besar, diperparah oleh keterbatasan sarana dan prasarana patroli serta teknologi pengawasan. Peran Tentara Nasional Indonesia Angkatan Laut (TNI AL) dalam mengamankan ALKI, termasuk melalui skema Operasi Militer Selain Perang (OMSP), menjadi sangat krusial dalam menghadapi berbagai tantangan ini.  

\subsubsection*{Kebijakan Pembangunan Maritim Nasional: Program Tol Laut, Pembangunan Infrastruktur, dan Konektivitas Antarwilayah}

Sebagai perwujudan Wawasan Nusantara dalam pembangunan ekonomi dan konektivitas, pemerintah Indonesia meluncurkan berbagai kebijakan pembangunan maritim nasional. Salah satu program unggulan adalah Tol Laut, yang dirancang untuk meningkatkan konektivitas antar pulau di seluruh nusantara, mengurangi disparitas harga barang antara wilayah barat dan timur Indonesia, serta mendukung pemerataan pembangunan ekonomi. Implementasi program Tol Laut melibatkan penetapan rute-rute pelayaran yang tetap dan teratur, pemberian subsidi untuk angkutan barang kebutuhan pokok dan penting, serta pengembangan program "Rumah Kita" sebagai sentra-sentra logistik di daerah-daerah tujuan untuk menstabilkan harga dan ketersediaan barang.  

Selain Tol Laut, fokus pembangunan maritim juga diarahkan pada pengembangan infrastruktur pendukung, seperti modernisasi dan pembangunan pelabuhan-pelabuhan baru, termasuk deep seaport yang mampu melayani kapal-kapal berukuran besar. Pengembangan industri perkapalan nasional dan sistem logistik maritim yang efisien juga menjadi prioritas untuk mendukung visi Indonesia sebagai poros maritim dunia. Upaya-upaya ini secara keseluruhan bertujuan untuk meningkatkan konektivitas antarwilayah, yang pada gilirannya diharapkan dapat mengurangi kesenjangan pembangunan regional dan memperkuat integrasi nasional, baik secara fisik maupun ekonomi.  

\subsubsection*{Penegakan Kedaulatan dan Hukum di Laut: Penanganan Illegal, Unreported, and Unregulated (IUU) Fishing, Perompakan, dan Penyelundupan}

Penegakan kedaulatan dan hukum di laut merupakan aspek kritikal dalam implementasi Wawasan Nusantara. Praktik Illegal, Unreported, and Unregulated (IUU) Fishing menjadi salah satu ancaman paling serius terhadap kelestarian sumber daya laut dan perekonomian nasional Indonesia. Kerugian negara akibat IUU fishing ditaksir sangat besar setiap tahunnya. Wawasan Nusantara, yang menekankan kedaulatan atas seluruh wilayah perairan dan pengelolaan sumber daya alam untuk sebesar-besar kemakmuran rakyat, menjadi landasan bagi pemerintah Indonesia untuk mengambil tindakan tegas terhadap para pelaku IUU fishing, baik domestik maupun asing.  

Selain IUU fishing, ancaman keamanan maritim lainnya seperti perompakan, pembajakan, dan berbagai bentuk penyelundupan (narkoba, senjata, manusia, barang ilegal lainnya) juga memerlukan upaya penegakan hukum yang kuat, sistematis, dan berkelanjutan. Pembentukan dan penguatan peran Badan Keamanan Laut (Bakamla) sebagai koordinator penegakan hukum di laut diharapkan dapat meningkatkan efektivitas pemberantasan berbagai kejahatan maritim tersebut. Upaya ini juga memerlukan kerjasama internasional yang erat dengan negara-negara tetangga dan komunitas maritim global.  

Pengelolaan wilayah maritim Indonesia yang sangat luas dan strategis, sebagaimana diamanatkan oleh Wawasan Nusantara, pada hakikatnya adalah sebuah "pedang bermata dua". Di satu sisi, ia menawarkan potensi ekonomi dan geopolitik yang luar biasa besar bagi kemajuan bangsa. Namun, di sisi lain, ia juga menghadirkan tantangan keamanan dan pengelolaan yang sangat kompleks. Keberhasilan Indonesia dalam mengelola potensi dan mengatasi ancaman di wilayah lautnya tidak hanya bergantung pada aspek teknis atau ketersediaan anggaran semata. Lebih dari itu, ia menuntut adanya komitmen politik yang kuat dan berkelanjutan, koordinasi yang sinergis antar berbagai kementerian dan lembaga terkait, penegakan hukum yang konsisten tanpa pandang bulu, serta partisipasi aktif dari masyarakat maritim. Semua upaya ini harus senantiasa dijiwai oleh semangat dan prinsip-prinsip Wawasan Nusantara. Kegagalan dalam mengelola aset maritim ini secara efektif tidak hanya akan menimbulkan kerugian ekonomi yang besar, tetapi juga berpotensi menggerogoti kedaulatan dan integritas wilayah NKRI.
%TODO:tabel 3 desuwa
\begin{table}[H]
\centering
\caption{Analisis Strategis Alur Laut Kepulauan Indonesia (ALKI)}
\label{tab:strategis_alur_laut}
\begin{tabular}{|>{\RaggedRight}p{0.07\textwidth}|>{\RaggedRight}p{0.15\textwidth}|>{\RaggedRight}p{0.2\textwidth}|>{\RaggedRight}p{0.2\textwidth}|>{\RaggedRight}p{0.25\textwidth}|}
\hline
\textbf{ALKI} & \textbf{Deskripsi Jalur Utama} & \textbf{Kepentingan Ekonomi Nasional} &	\textbf{Kepentingan Pertahanan dan Keamanan Nasional} & \textbf{Tantangan Utama Pengelolaan}\\
\hline
ALKI I & Selat Sunda, Laut Jawa, Selat Karimata, Laut Natuna, Laut Cina Selatan &	Jalur utama perdagangan domestik (antar pulau Jawa-Sumatera-Kalimantan) dan internasional (dari/ke Samudra Hindia - Laut Cina Selatan), potensi pariwisata bahari,akses ke sumber daya migas di Laut Natuna, pendukung utama program Tol Laut.&	Jalur patroli strategis TNI AL, pengawasan kedaulatan di Laut Natuna Utara yang berbatasan dengan area sengketa LCS, potensi ancaman militer dari kekuatan asing, kepentingan pengamanan objek vital nasional lepas pantai. &	IUU fishing (terutama oleh kapal ikan asing), penyelundupan, pelanggaran lintas oleh kapal perang asing, potensi perompakan di area tertentu, pencemaran dari lalu lintas kapal yang padat, keterbatasan kemampuan pengawasan di area yang luas, potensi dampak konflik LCS. \\
\hline
ALKI II &	Selat Lombok, Selat Makassar, Laut Sulawesi &	Jalur alternatif penting bagi kapal-kapal besar (termasuk supertanker) yang tidak bisa melalui Selat Malaka, menghubungkan Australia dengan Asia Timur, potensi pengembangan pelabuhan hub internasional di Indonesia Timur, jalur distribusi logistik ke Indonesia bagian tengah dan timur. &Jalur strategis bagi pergerakan armada laut internasional, pengawasan aktivitas militer asing, pengamanan wilayah perbatasan laut dengan Malaysia dan Filipina, pencegahan infiltrasi dan separatisme. &IUU fishing, penyelundupan (khususnya narkoba dan senjata dari/ke Filipina Selatan), perompakan di Laut Sulawesi, ancaman terorisme lintas batas, potensi sengketa batas maritim, tantangan pengawasan di selat dalam dan panjang. \\
\hline
\end{tabular}
\end{table}

\begin{table}[H]
\centering
\begin{tabular}{|>{\RaggedRight}p{0.07\textwidth}|>{\RaggedRight}p{0.15\textwidth}|>{\RaggedRight}p{0.2\textwidth}|>{\RaggedRight}p{0.2\textwidth}|>{\RaggedRight}p{0.25\textwidth}|}
\hline
\textbf{ALKI} & \textbf{Deskripsi Jalur Utama} & \textbf{Kepentingan Ekonomi Nasional} &	\textbf{Kepentingan Pertahanan dan Keamanan Nasional} & \textbf{Tantangan Utama Pengelolaan}\\
\hline
ALKI III (A, B, C, E) &	ALKI III A: Laut Sawu, Selat Ombai, Laut Banda, Laut Seram, Laut Maluku, Samudra Pasifik. ALKI III B: Laut Timor, Selat Leti, Laut Banda (ke ALKI III A). ALKI III C: Laut Arafura, Laut Banda (ke ALKI III A). ALKI III E: (Rute sebaliknya dari ALKI III C). &	Akses ke sumber daya perikanan laut dalam di Indonesia Timur, potensi eksplorasi migas di Laut Arafura dan Laut Banda, jalur pelayaran menuju/dari Australia dan negara-negara Pasifik Selatan, mendukung konektivitas ke pulau-pulau terluar di wilayah timur. &	Pengamanan wilayah perbatasan laut dengan Timor Leste, Australia, dan Palau, pencegahan penyelundupan dan IUU fishing di wilayah terpencil, pengawasan aktivitas kapal asing di perairan yang jarang terjamah, mendukung kehadiran negara di wilayah perbatasan.&	IUU fishing skala besar oleh kapal asing, penyelundupan hasil hutan dan mineral, kerentanan pulau-pulau kecil dan terluar, keterbatasan infrastruktur pengawasan dan penegakan hukum, jarak yang jauh dari pusat komando utama, potensi dampak perubahan iklim terhadap ekosistem laut. \\
\hline
\end{tabular}
\end{table}
Tabel ini menyajikan perbandingan langsung antara ketiga ALKI utama dari berbagai aspek (jalur, ekonomi, pertahanan-keamanan, dan tantangan). Pemahaman ini penting untuk mengapresiasi kompleksitas pengelolaan ALKI dan relevansinya dengan Wawasan Nusantara dalam konteks kepentingan nasional Indonesia di kancah internasional.
\subsection*{Wawasan Nusantara sebagai Dasar Kebijakan Luar Negeri dan Diplomasi Maritim}

Wawasan Nusantara tidak hanya berfungsi sebagai panduan internal bagi bangsa Indonesia, tetapi juga menjadi landasan fundamental dalam perumusan kebijakan luar negeri dan pelaksanaan diplomasi, khususnya diplomasi maritim. Prinsip politik luar negeri Indonesia yang bebas aktif – bebas menentukan sikap dan kebijakan sendiri terhadap permasalahan internasional tanpa mengikatkan diri pada salah satu blok kekuatan dunia, dan aktif memberikan sumbangan dalam menyelesaikan konflik internasional demi terwujudnya ketertiban dunia – menemukan justifikasi dan arahnya dalam kerangka Wawasan Nusantara.  

Dalam konteks ini, Wawasan Nusantara menjadi pedoman utama dalam upaya menjaga kedaulatan dan integritas wilayah Indonesia dari segala bentuk ancaman eksternal. Lebih jauh, konsepsi ini mendorong Indonesia untuk proaktif dalam menjalankan diplomasi maritim yang bertujuan untuk menghilangkan atau memitigasi sumber-sumber konflik di laut, serta membangun kerjasama internasional yang saling menguntungkan di bidang kelautan, baik secara bilateral, regional, maupun multilateral.  

\subsubsection*{Posisi Indonesia dalam Dinamika Geopolitik Regional (Contoh Kasus: Sengketa Laut Cina Selatan)}

Salah satu contoh paling relevan dari implementasi Wawasan Nusantara dalam kebijakan luar negeri adalah sikap Indonesia dalam menghadapi dinamika geopolitik di kawasan, khususnya terkait sengketa Laut Cina Selatan (LCS). Meskipun Indonesia secara resmi menyatakan bukan merupakan negara pihak yang bersengketa (claimant state) atas kepulauan atau fitur maritim di LCS, Indonesia memiliki kepentingan nasional yang sangat besar terhadap stabilitas keamanan di kawasan tersebut dan penegakan prinsip-prinsip hukum internasional, terutama UNCLOS 1982.  

Permasalahan muncul ketika klaim sepihak Tiongkok atas sebagian besar wilayah LCS melalui konsep Nine-Dash Line tumpang tindih dengan Zona Ekonomi Eksklusif (ZEE) Indonesia di perairan sekitar Kepulauan Natuna, yang kini dikenal sebagai Laut Natuna Utara. Menghadapi situasi ini, respons Indonesia konsisten berlandaskan pada Wawasan Nusantara dan hukum internasional. Di satu sisi, Indonesia mengedepankan upaya-upaya diplomatik, baik secara bilateral dengan Tiongkok maupun dalam kerangka ASEAN, untuk mendorong penyelesaian sengketa secara damai dan perumusan Code of Conduct (CoC) di LCS yang mengikat secara hukum antara ASEAN dan Tiongkok. Di sisi lain, Indonesia juga mengambil langkah-langkah penegasan kedaulatan dan hak berdaulatnya di Laut Natuna Utara, termasuk dengan meningkatkan kehadiran patroli militer dan penegakan hukum terhadap kapal-kapal asing yang melakukan pelanggaran di wilayah yurisdiksinya. Sikap tegas Indonesia ini selalu didasarkan pada ketentuan UNCLOS 1982, yang juga merupakan produk hukum internasional yang turut diperjuangkan Indonesia.  

\subsubsection*{Peran Aktif dalam Organisasi Internasional dan Forum Maritim}

Sebagai negara kepulauan terbesar dan pemain penting di kawasan, Indonesia secara aktif berpartisipasi dalam berbagai organisasi internasional dan forum maritim. Keterlibatan ini merupakan bagian dari implementasi Wawasan Nusantara dalam konteks pergaulan global, di mana Indonesia berupaya memperjuangkan kepentingan nasionalnya sekaligus memberikan kontribusi bagi stabilitas dan kerjasama regional maupun internasional.  

Indonesia memainkan peran sentral di Perhimpunan Bangsa-Bangsa Asia Tenggara (ASEAN), dan turut aktif dalam forum-forum yang lebih luas seperti Perserikatan Bangsa-Bangsa (PBB) dan Kelompok Duapuluh (G20). Dalam isu-isu maritim, Indonesia terlibat aktif dalam berbagai platform seperti ASEAN Regional Forum (ARF), ASEAN Maritime Forum (AMF), dan Expanded ASEAN Maritime Forum (EAMF), yang bertujuan untuk membangun kepercayaan, kerjasama keamanan maritim, dan penanganan bersama terhadap tantangan-tantangan maritim di kawasan.  

Dalam arena diplomasi, Wawasan Nusantara memberikan Indonesia sebuah leverage atau daya tawar yang signifikan sebagai negara kepulauan terbesar di dunia dengan posisi geostrategis yang tidak tergantikan. Namun, posisi ini juga menuntut adanya konsistensi antara retorika kesatuan dan kedaulatan yang digaungkan di dalam negeri dengan tindakan nyata dalam menjaga stabilitas kawasan dan menghormati hukum internasional. Sikap Indonesia dalam isu LCS, yang berpegang teguh pada Wawasan Nusantara dan UNCLOS 1982, menjadi ujian penting bagi kredibilitasnya sebagai aktor regional yang bertanggung jawab dan matang. Kemampuan Indonesia untuk secara konsisten menolak klaim-klaim yang tidak memiliki dasar hukum internasional yang kuat, sambil tetap mengedepankan dialog dan solusi damai, adalah manifestasi dari penerapan Wawasan Nusantara yang dewasa dalam dinamika hubungan internasional kontemporer.
\subsection*{Respons terhadap Isu-Isu Global Kontemporer}

Relevansi Wawasan Nusantara juga teruji dalam kemampuan Indonesia merespons berbagai isu global kontemporer yang dampaknya melintasi batas-batas negara. Dua isu utama yang sangat relevan bagi Indonesia sebagai negara kepulauan adalah perubahan iklim dan tantangan kedaulatan data serta keamanan siber.
\subsubsection*{Strategi Adaptasi dan Mitigasi Perubahan Iklim bagi Negara Kepulauan}

Wawasan Nusantara, yang menekankan kesatuan lingkungan hidup dan pengelolaan sumber daya alam secara berkelanjutan untuk kesejahteraan bangsa, menyediakan landasan filosofis bagi Indonesia dalam menghadapi perubahan iklim. Sebagai negara kepulauan dengan ribuan pulau dan garis pantai yang sangat panjang, Indonesia termasuk negara yang paling rentan terhadap dampak negatif perubahan iklim. Ancaman nyata meliputi kenaikan permukaan air laut yang dapat menenggelamkan pulau-pulau kecil dan wilayah pesisir, meningkatnya frekuensi dan intensitas bencana alam hidrometeorologi (banjir, kekeringan, badai), serta kerusakan ekosistem laut yang vital bagi ketahanan pangan dan ekonomi masyarakat.  

Menghadapi tantangan ini, strategi Indonesia yang berlandaskan Wawasan Nusantara mencakup upaya mitigasi dan adaptasi. Upaya mitigasi difokuskan pada pengurangan emisi gas rumah kaca, antara lain melalui pengembangan energi terbarukan yang ramah lingkungan, pengelolaan hutan dan lahan gambut yang berkelanjutan, serta peningkatan efisiensi energi. Sementara itu, upaya adaptasi bertujuan untuk meningkatkan ketahanan masyarakat dan ekosistem terhadap dampak perubahan iklim yang sudah tidak dapat dihindari, misalnya melalui pembangunan infrastruktur pesisir yang tangguh, diversifikasi mata pencaharian masyarakat pesisir, dan pengembangan sistem peringatan dini bencana. Peningkatan kesadaran masyarakat akan pentingnya isu perubahan iklim dan pelestarian lingkungan juga menjadi bagian integral dari strategi ini.  

\subsubsection{Menjaga Kedaulatan Data dan Keamanan Siber Nasional}

Memasuki era digital, kedaulatan tidak lagi hanya dimaknai secara teritorial fisik, tetapi juga mencakup kedaulatan atas data dan keamanan di ruang siber. Perkembangan teknologi informasi dan komunikasi yang pesat membawa berbagai kemudahan, namun sekaligus memunculkan ancaman baru terhadap keamanan nasional, privasi warga negara, dan integritas data. Wawasan Nusantara, dengan penekanannya pada kesatuan dan kedaulatan negara, menjadi relevan dalam memandu Indonesia membangun sistem ketahanan siber nasional yang komprehensif.  

Pemerintah Indonesia, melalui lembaga seperti Badan Siber dan Sandi Negara (BSSN), telah merumuskan dan mengimplementasikan strategi keamanan siber nasional. Strategi ini umumnya mengacu pada lima pilar utama: penguatan kerangka hukum terkait kejahatan siber dan perlindungan data pribadi; peningkatan kapabilitas teknis untuk deteksi, pencegahan, dan respons terhadap serangan siber; pembentukan dan penguatan organisasi keamanan siber nasional dan sektoral; pengembangan kapasitas sumber daya manusia di bidang keamanan siber; serta peningkatan kerjasama internasional dalam penanganan ancaman siber lintas negara. Dalam perspektif Wawasan Nusantara, upaya ini juga harus didukung oleh penekanan pada pentingnya kedaulatan digital, kemandirian teknologi, dan perlindungan terhadap data strategis nasional serta data pribadi seluruh warga negara.  

Isu-isu global seperti perubahan iklim dan keamanan siber secara signifikan memperluas dimensi pemaknaan Wawasan Nusantara. Konsepsi ini tidak lagi hanya terbatas pada pemahaman akan kesatuan teritorial fisik darat, laut, dan udara. Lebih dari itu, Wawasan Nusantara kini dituntut untuk mampu merangkul dan mengintegrasikan dimensi "kesatuan ekologis" dalam menghadapi krisis iklim, serta "kesatuan digital" dalam menjaga kedaulatan di ruang maya. Adaptasi konseptual ini sangat krusial agar Wawasan Nusantara tetap relevan dan fungsional sebagai panduan strategis Indonesia dalam menjaga kepentingan nasionalnya di abad ke-21 yang semakin kompleks. Kegagalan dalam mengadaptasi Wawasan Nusantara terhadap isu-isu global kontemporer ini berisiko menjadikannya sekadar konsep yang usang dan kehilangan relevansinya. Sebaliknya, keberhasilan dalam proses adaptasi ini akan menunjukkan dinamisme dan ketahanan Wawasan Nusantara sebagai sebuah pandangan dunia yang hidup dan mampu memberikan solusi. Hal ini bahkan berpotensi menawarkan model bagi negara-negara lain yang menghadapi tantangan serupa dalam menjaga kedaulatan dan kesejahteraan di tengah arus perubahan global.
\section{Tantangan Kontemporer, Prospek Masa Depan, dan Strategi Penguatan Wawasan Nusantara}
\subsection*{Analisis Tantangan Internal}

Implementasi Wawasan Nusantara sebagai pandangan kolektif kebangsaan menghadapi berbagai tantangan internal yang memerlukan perhatian serius. Salah satu tantangan utama adalah kesenjangan pembangunan antarwilayah. Perbedaan tingkat kemajuan ekonomi dan akses terhadap layanan publik antara wilayah barat dan timur Indonesia, serta antara perkotaan dan perdesaan, masih menjadi persoalan nyata. Kesenjangan ini, jika tidak ditangani secara efektif, dapat memicu kecemburuan sosial, rasa ketidakadilan, dan pada akhirnya dapat mengancam semangat persatuan dan kesatuan bangsa yang menjadi inti Wawasan Nusantara. Padahal, Wawasan Nusantara secara eksplisit mengamanatkan terwujudnya pembangunan yang merata dan berkeadilan di seluruh pelosok tanah air.  

Tantangan internal lainnya adalah potensi disintegrasi bangsa yang bersumber dari isu-isu Suku, Agama, Ras, dan Antargolongan (SARA) serta munculnya gerakan separatisme. Keberagaman Indonesia yang luar biasa, meskipun merupakan kekayaan, juga menyimpan potensi konflik jika tidak dikelola dengan bijaksana dan dilandasi semangat toleransi serta saling menghargai. Gerakan separatisme di beberapa daerah, meskipun intensitasnya bervariasi, tetap menjadi ancaman laten terhadap keutuhan wilayah NKRI.  

Dalam konteks ini, penguatan identitas nasional yang inklusif menjadi sangat penting. Identitas nasional tidak boleh meniadakan atau meminggirkan identitas-identitas lokal yang beragam. Sebaliknya, Wawasan Nusantara mengajarkan bahwa budaya-budaya lokal adalah bagian integral dari kekayaan budaya nasional dan harus dilestarikan serta dikembangkan sebagai pilar-pilar yang memperkokoh identitas bangsa secara keseluruhan. Membangun narasi kebangsaan yang merangkul seluruh elemen masyarakat tanpa terkecuali adalah kunci untuk memperkuat kohesi sosial.  

Tantangan-tantangan internal terhadap Wawasan Nusantara ini seringkali berakar pada persepsi mendalam mengenai ketidakadilan, baik dalam distribusi sumber daya dan hasil pembangunan (keadilan distributif) maupun dalam pengakuan terhadap eksistensi dan kontribusi berbagai kelompok masyarakat (keadilan rekognitif). Apabila persepsi ketidakadilan ini terus menguat dan tidak mendapatkan respons kebijakan yang memadai, ia dapat dengan mudah dieksploitasi oleh pihak-pihak yang memiliki agenda untuk memecah belah persatuan bangsa. Oleh karena itu, implementasi Wawasan Nusantara yang sejati dan substantif menuntut komitmen yang kuat dari negara untuk mewujudkan keadilan dalam segala dimensinya. Upaya penguatan Wawasan Nusantara yang hanya bersifat top-down, seremonial, atau retoris, tanpa diiringi dengan kebijakan konkret yang mengatasi akar permasalahan ketidakadilan dan kurangnya representasi, akan cenderung kurang efektif dan tidak berkelanjutan.
\subsection*{Analisis Tantangan Eksternal}

Selain tantangan internal, implementasi Wawasan Nusantara juga dihadapkan pada berbagai tantangan eksternal yang signifikan di era kontemporer. Dampak globalisasi merupakan salah satu faktor utama. Arus informasi, budaya, dan nilai-nilai global yang masuk secara masif dan cepat melalui berbagai kanal berpotensi menggerus nilai-nilai luhur lokal, tradisi bangsa, dan semangat nasionalisme, terutama di kalangan generasi muda. Dominasi sistem ekonomi kapitalisme global juga dapat memengaruhi kedaulatan ekonomi nasional dan memperlebar kesenjangan.  

Perkembangan Teknologi Informasi dan Komunikasi (TIK) yang pesat menghadirkan tantangan sekaligus peluang. Di satu sisi, TIK mempermudah akses terhadap informasi, memperlancar komunikasi lintas batas, dan membuka peluang ekonomi baru. Namun, di sisi lain, TIK juga menjadi medium penyebaran berita bohong (hoaks), ujaran kebencian, konten ilegal (seperti pornografi dan perjudian daring), serta perundungan siber (cyberbullying), yang semuanya dapat merusak tatanan sosial dan mengancam persatuan bangsa. Era digital menuntut adanya adaptasi dalam cara pandang dan implementasi Wawasan Nusantara agar tetap relevan dan mampu menangkal dampak negatif TIK.  

Dinamika geopolitik internasional yang terus berubah juga menjadi tantangan tersendiri. Persaingan antara kekuatan-kekuatan global, munculnya sengketa wilayah di kawasan sekitar Indonesia, serta meningkatnya ancaman keamanan transnasional (seperti terorisme, penyelundupan narkoba, dan kejahatan siber) memerlukan respons strategis yang cerdas dan adaptif, yang tetap berlandaskan pada prinsip-prinsip Wawasan Nusantara.  

Menariknya, tantangan eksternal, khususnya yang berasal dari globalisasi dan kemajuan TIK, tidak secara otomatis berakibat pada pelemahan Wawasan Nusantara. Sebaliknya, tantangan-tantangan ini justru dapat menjadi katalisator untuk memperkuat relevansi dan urgensi Wawasan Nusantara. Jika dimanfaatkan secara cerdas dan strategis, globalisasi dan TIK dapat menjadi alat yang efektif untuk mempromosikan narasi persatuan Indonesia, kekayaan budayanya, dan kepentingan nasionalnya di kancah global. Misalnya, platform digital dapat digunakan untuk menyebarkan pesan-pesan positif tentang keberagaman dan harmoni di Indonesia, serta untuk membangun solidaritas nasional dalam menghadapi isu-isu bersama. Wawasan Nusantara, dalam konteks ini, perlu diadaptasi agar tidak hanya menjadi panduan internal, tetapi juga sebagai landasan untuk berinteraksi dan berkontribusi dalam komunitas global. Kerjasama internasional di era digital, misalnya dalam penanganan kejahatan siber atau promosi budaya, menjadi semakin penting. Dengan demikian, Indonesia berpotensi untuk "mengekspor" nilai-nilai positif yang terkandung dalam Wawasan Nusantara, seperti kemampuan untuk hidup rukun dalam keberagaman, sebagai kontribusi bagi upaya membangun perdamaian dan stabilitas global. Hal ini akan mengubah Wawasan Nusantara dari sekadar konsep yang berorientasi ke dalam (inward-looking) menjadi sebuah aset diplomasi publik (public diplomacy) yang berharga di panggung internasional.  

\subsection*{Strategi Komprehensif Penguatan Wawasan Nusantara}

Menghadapi berbagai tantangan internal dan eksternal, diperlukan strategi komprehensif untuk terus memperkuat dan mengaktualisasikan Wawasan Nusantara dalam kehidupan berbangsa dan bernegara. Strategi ini harus melibatkan berbagai sektor dan aktor, serta bersifat berkelanjutan.
\subsubsection*{Peran Pendidikan dan Kebudayaan}

Sektor pendidikan dan kebudayaan memegang peran sentral dalam menanamkan nilai-nilai Wawasan Nusantara sejak dini. Kementerian Pendidikan, Kebudayaan, Riset, dan Teknologi (Kemendikbudristek) memiliki berbagai program yang relevan. Salah satunya adalah Projek Penguatan Profil Pelajar Pancasila (P5), yang merupakan bagian integral dari Kurikulum Merdeka. P5 bertujuan untuk mengembangkan karakter dan kompetensi siswa agar sesuai dengan nilai-nilai luhur Pancasila melalui pendekatan pembelajaran berbasis proyek yang kontekstual dan relevan dengan lingkungan sekitar siswa. Salah satu tema dalam P5, yaitu "Kearifan Lokal," secara langsung dapat digunakan untuk menggali, memahami, dan mengapresiasi keragaman budaya Nusantara, yang merupakan esensi dari Wawasan Nusantara.  

Selain P5, program Pendidikan Karakter juga terus digalakkan untuk menanamkan nilai-nilai fundamental seperti religiusitas, kejujuran, toleransi, disiplin, kerja keras, cinta tanah air, semangat kebangsaan, dan penghargaan terhadap keberagaman budaya dan tradisi. Penanaman nilai nasionalisme dan cinta tanah air secara eksplisit menjadi bagian dari nilai-nilai utama karakter yang dikembangkan. Internalisasi Wawasan Kebangsaan secara lebih luas dilakukan melalui jalur pendidikan formal (integrasi dalam kurikulum mata pelajaran seperti Pendidikan Pancasila dan Kewarganegaraan) maupun jalur informal (kegiatan ekstrakurikuler, pembiasaan di lingkungan sekolah dan keluarga).  

Upaya pelestarian dan pengembangan budaya lokal juga merupakan bagian tak terpisahkan dari penguatan Wawasan Nusantara. Budaya lokal diakui sebagai bagian integral dari kekayaan dan identitas nasional, yang harus dijaga kelestariannya dan dikembangkan agar tetap relevan di tengah perubahan zaman.  

\subsubsection*{Peran Kementerian Agama (Kemenag)}

Kementerian Agama memiliki peran strategis dalam mempromosikan harmoni dan kerukunan di tengah masyarakat Indonesia yang majemuk secara agama, yang sejalan dengan semangat Wawasan Nusantara. Salah satu program utama Kemenag adalah Moderasi Beragama. Program ini bertujuan untuk menumbuhkan cara pandang, sikap, dan praktik beragama yang melindungi martabat kemanusiaan, membangun kemaslahatan umum, serta berlandaskan pada prinsip keadilan, keseimbangan, dan ketaatan terhadap konstitusi negara. Moderasi beragama bukanlah upaya untuk memoderasi ajaran agama itu sendiri, melainkan memoderasi pemahaman dan praktik umat beragama agar selaras dengan nilai-nilai kebangsaan.  

Selain itu, Kemenag juga aktif dalam Penguatan Kerukunan Umat Beragama. Upaya ini dilakukan melalui berbagai kegiatan, seperti optimalisasi dan sosialisasi peraturan perundang-undangan terkait kerukunan, peningkatan kapasitas para aktor kerukunan (termasuk Forum Kerukunan Umat Beragama - FKUB), pemberdayaan lembaga-lembaga keagamaan dan institusi media, pengembangan kesadaran akan pentingnya kerukunan, serta pembinaan pemahaman ajaran agama yang berwawasan moderat dan multikultural.  

\subsubsection*{Kebijakan Publik yang Berorientasi pada Pemerataan dan Kesejahteraan}

Untuk mengatasi tantangan kesenjangan pembangunan, diperlukan kebijakan publik yang secara nyata berorientasi pada pemerataan dan peningkatan kesejahteraan di seluruh wilayah Indonesia. Pembangunan yang merata dan berkeadilan adalah salah satu pilar penting dalam menjaga keutuhan bangsa, sebagaimana diamanatkan oleh Wawasan Nusantara. Implementasi otonomi daerah juga harus senantiasa diselaraskan dengan kerangka Wawasan Nusantara agar tidak mengarah pada disintegrasi, melainkan justru memperkuat integrasi nasional melalui pemberdayaan daerah yang bertanggung jawab.  

\subsubsection*{Doktrin Sistem Pertahanan dan Keamanan Rakyat Semestasebagai Penjabaran Aspek Pertahanan Wawasan Nusantara}

Dalam aspek pertahanan dan keamanan, Wawasan Nusantara dijabarkan lebih lanjut dalam doktrin Sistem Pertahanan dan Keamanan Rakyat Semesta (SISHANKAMRATA) \cite{SISHANKAMRATA}. Sishankamrata adalah sistem pertahanan dan keamanan negara yang melibatkan seluruh potensi, kemampuan, dan kekuatan nasional (militer dan non-militer, termasuk seluruh warga negara dan sumber daya nasional lainnya) yang bekerja secara total, integral, terpadu, terarah, dan berlanjut untuk menghadapi segala macam ancaman. Sifat kesemestaan ini mengandung makna pelibatan seluruh rakyat dan segenap sumber daya nasional, sarana dan prasarana nasional, serta seluruh wilayah negara sebagai satu kesatuan pertahanan.  

Wawasan Nusantara, yang memandang kepulauan Nusantara sebagai satu kesatuan pertahanan dan keamanan, menjadi landasan visional bagi penyelenggaraan Sishankamrata. Amanat Undang-Undang Dasar 1945 Pasal 30 mengenai hak dan kewajiban setiap warga negara untuk ikut serta dalam usaha pertahanan dan keamanan negara menjadi dasar konstitusional Sishankamrata \cite{sissu}. Tujuan utama Sishankamrata adalah untuk menjaga kedaulatan negara, keutuhan wilayah NKRI, dan keselamatan segenap bangsa dari berbagai ancaman, baik yang datang dari luar maupun dari dalam negeri. Doktrin ini lahir dan berkembang dari sejarah perjuangan bangsa Indonesia dan disesuaikan dengan kondisi geografis Indonesia sebagai negara kepulauan.  

\subsubsection*{Peran Aktif Masyarakat Sipil dan Media}

Penguatan Wawasan Nusantara tidak hanya menjadi tanggung jawab pemerintah, tetapi juga memerlukan peran aktif dari seluruh komponen masyarakat sipil. Setiap warga negara diharapkan dapat memahami, menghayati, dan mengamalkan nilai-nilai Wawasan Nusantara dalam kehidupan sehari-hari. Media massa, termasuk media sosial, memiliki peran strategis dalam proses ini. Media dapat digunakan sebagai sarana untuk menyebarkan informasi yang akurat, pesan-pesan positif tentang keberagaman budaya dan persatuan bangsa, serta membangun kesadaran kolektif akan pentingnya Wawasan Nusantara.  

Strategi penguatan Wawasan Nusantara yang paling efektif adalah strategi yang bersifat multidimensional, melibatkan berbagai aspek kehidupan; partisipatif, melibatkan seluruh elemen bangsa; dan adaptif, mampu merespons perubahan zaman dan tantangan baru. Pendekatan yang hanya bersifat top-down dari negara cenderung kurang optimal. Sebaliknya, upaya penguatan Wawasan Nusantara harus mampu memberdayakan inisiatif-inisiatif yang tumbuh dari bawah (bottom-up), serta memanfaatkan kemajuan teknologi informasi dan komunikasi secara positif dan konstruktif. Penguatan Wawasan Nusantara tidak bisa lagi bersifat monolitik atau dogmatis. Ia harus menjadi sebuah "gerakan" kebangsaan yang melibatkan beragam aktor (pemerintah, lembaga pendidikan, keluarga, komunitas lokal, organisasi masyarakat sipil, media, dan individu) dan menggunakan berbagai platform (kurikulum pendidikan, kegiatan budaya, konten digital kreatif, dialog publik, keteladanan) untuk menanamkan dan mengaktualisasikan nilai-nilai kesatuan dalam keberagaman secara relevan, menarik, dan berkelanjutan bagi seluruh generasi bangsa Indonesia.
\section{Kesimpulan dan Rekomendasi Strategis}
\subsection*{Sintesis Temuan Kunci mengenai Dinamika Historis dan Urgensi Wawasan Nusantara}

Analisis terhadap dinamika historis dan urgensi Wawasan Nusantara sebagai konsepsi dan pandangan kolektif kebangsaan Indonesia dalam konteks pergaulan dunia mengungkapkan beberapa temuan kunci. 
\subsubsection*{perubahan konsepsi wilayah maritim Indonesia} \label{perubahan-konsepsi-wilayah-maritim-indonesia} pertama Revolusi konsepsi wilayah maritim Indonesia menunjukkan transformasi fundamental dari rezim Territoriale Zeeën en Maritieme Kringen Ordonantie 1939 (TZMKO 1939)\cite{TZMKO} yang bersifat fragmentatif dan merupakan warisan kolonial, menuju proklamasi kesatuan melalui Deklarasi Djuanda 1957. Perjuangan diplomatik yang gigih, terutama di bawah kepemimpinan intelektual Prof. Mochtar Kusumaatmadja, berpuncak pada pengakuan internasional atas konsep negara kepulauan dalam UNCLOS 1982. Pengakuan ini kemudian diformalkan dan diintegrasikan ke dalam sistem hukum nasional melalui serangkaian peraturan perundang-undangan, yang menandai legitimasi penuh Wawasan Nusantara secara yuridis.
\cite{rustam2015tantangan}
\subsubsection*{Wawasan Nusantara lebih dari sekadar konsep geopolitik}
Kedua, Wawasan Nusantara lebih dari sekadar konsep geopolitik; ia adalah pandangan hidup bangsa yang berakar kuat pada falsafah Pancasila dan semangat Bhinneka Tunggal Ika. Konsepsi ini bersifat multidimensional, mencakup kesatuan dalam aspek politik, ekonomi, sosial budaya, serta pertahanan dan keamanan (Poleksosbudhankam), yang memandang seluruh wilayah darat, laut, dan udara Indonesia sebagai satu kesatuan yang utuh dan tidak terpisahkan.

\subsubsection*{Urgensi kontemporer Wawasan Nusantara} 
Ketiga urgensi kontemporer Wawasan Nusantara terletak pada relevansinya sebagai kerangka panduan strategis bagi Indonesia dalam menghadapi berbagai tantangan kompleks, baik yang bersifat internal maupun eksternal. Tantangan internal meliputi kesenjangan pembangunan antarwilayah, potensi disintegrasi akibat isu SARA dan separatisme, serta kebutuhan untuk terus memperkuat identitas nasional yang inklusif. Tantangan eksternal mencakup dampak globalisasi, pesatnya perkembangan teknologi informasi dan komunikasi, dinamika geopolitik regional dan internasional yang fluktuatif, serta isu-isu global seperti perubahan iklim dan keamanan siber. Wawasan Nusantara menyediakan landasan bagi Indonesia untuk mengelola wilayah maritimnya yang luas, menegakkan kedaulatan, merumuskan kebijakan luar negeri yang bebas aktif, serta berkontribusi pada tatanan dunia yang lebih baik.

\subsection*{Rekomendasi Kebijakan dan Langkah Aplikatif untuk Memperkokoh Wawasan Nusantara dalam Menghadapi Kompleksitas Global}
Berdasarkan analisis yang telah dilakukan, berikut adalah beberapa rekomendasi kebijakan dan langkah aplikatif yang dapat dipertimbangkan untuk memperkokoh Wawasan Nusantara dalam menghadapi kompleksitas global:
\subsubsection*{Bidang Pendidikan dan Kebudayaan}
\begin{enumerate}
  \item Mengintegrasikan secara lebih mendalam dan kreatif nilai-nilai Wawasan Nusantara ke dalam seluruh jenjang dan jalur pendidikan, tidak hanya melalui mata pelajaran khusus tetapi juga melalui kegiatan kokurikuler dan ekstrakurikuler yang kontekstual, seperti penguatan program Projek Penguatan Profil Pelajar Pancasila (P5) dengan tema-tema yang relevan dengan kearifan lokal dan kesatuan maritim.
  \item Mendorong penelitian dan pengembangan konten-konten pendidikan Wawasan Nusantara yang inovatif dan menarik bagi generasi muda, memanfaatkan platform digital dan media sosial secara optimal.
  \item Memperkuat program pelestarian dan revitalisasi budaya lokal sebagai bagian integral dari kekayaan budaya nasional, serta mempromosikannya sebagai aset diplomasi budaya Indonesia di kancah internasional.
\end{enumerate}
\subsubsection*{Bidang Hukum dan Tata Kelola Pemerintahan}
\begin{enumerate}
  \item Melakukan kajian komprehensif terhadap sinkronisasi dan harmonisasi berbagai peraturan perundang-undangan sektoral (Kelautan, Pertahanan, Penataan Ruang, Pemerintahan Daerah, dll.) untuk memastikan implementasi Wawasan Nusantara yang lebih terpadu dan efektif, serta mengatasi potensi tumpang tindih kewenangan antar lembaga.
  \item Memperkuat kapasitas kelembagaan dan sumber daya manusia aparatur negara di semua tingkatan dalam memahami dan mengimplementasikan Wawasan Nusantara dalam setiap perumusan dan pelaksanaan kebijakan publik.
  \item Meningkatkan transparansi dan akuntabilitas dalam pengelolaan sumber daya alam, khususnya sumber daya maritim, untuk memastikan manfaatnya dirasakan secara adil oleh seluruh masyarakat, terutama di daerah pesisir dan pulau-pulau kecil.
\end{enumerate}
\subsubsection*{Bidang Ekonomi dan Pembangunan}
\begin{enumerate}
  \item Mempercepat upaya pemerataan pembangunan infrastruktur fisik dan digital di seluruh wilayah Indonesia, khususnya di daerah tertinggal, terdepan, dan terluar (3T), untuk mengurangi kesenjangan dan memperkuat konektivitas nasional sebagai perwujudan nyata Wawasan Nusantara.
  \item Mengoptimalkan program Tol Laut dan pengembangan sentra-sentra ekonomi maritim yang terintegrasi, dengan melibatkan partisipasi aktif masyarakat lokal dan pelaku usaha daerah.
  \item Mendorong pengembangan ekonomi biru (blue economy) yang berkelanjutan, yang menyeimbangkan antara pemanfaatan potensi ekonomi kelautan dengan pelestarian lingkungan laut.
\end{enumerate}
\subsubsection*{Bidang Pertahanan dan Keamanan}
\begin{enumerate}
  \item Terus memodernisasi dan memperkuat kapasitas pertahanan dan keamanan nasional, khususnya matra laut dan udara, untuk menjaga kedaulatan dan integritas wilayah NKRI, serta mengamankan Alur Laut Kepulauan Indonesia (ALKI) dan chokepoints maritim dari berbagai ancaman.
  \item Mengoptimalkan implementasi doktrin Sistem Pertahanan dan Keamanan Rakyat Semesta (Sishankamrata) dengan melibatkan partisipasi aktif seluruh komponen bangsa dalam upaya bela negara.
  \item Meningkatkan kemampuan keamanan siber nasional untuk melindungi infrastruktur kritis, kedaulatan data, dan privasi warga negara dari ancaman siber.
\end{enumerate}

\subsubsection*{Bidang Diplomasi dan Hubungan Internasional}
\begin{enumerate}
  \item Memanfaatkan Wawasan Nusantara sebagai landasan yang kokoh dalam menjalankan politik luar negeri bebas aktif, memperjuangkan kepentingan nasional Indonesia di forum-forum regional dan internasional, serta berkontribusi dalam penyelesaian isu-isu global.
  \item Mengintensifkan diplomasi maritim untuk memperkuat kerjasama internasional dalam pengelolaan sumber daya laut yang berkelanjutan, penanganan kejahatan lintas negara di laut, dan pemeliharaan keamanan serta stabilitas kawasan maritim.
  \item Mempromosikan nilai-nilai Wawasan Nusantara, seperti harmoni dalam keberagaman dan penyelesaian sengketa secara damai, sebagai bagian dari kontribusi Indonesia bagi perdamaian dunia.
\end{enumerate}
\subsubsection*{Bidang Pemberdayaan Masyarakat dan Peran Media}
\begin{enumerate}
  \item Mendorong partisipasi aktif masyarakat sipil, organisasi kemasyarakatan, dan komunitas lokal dalam berbagai program penguatan Wawasan Nusantara, termasuk dalam pelestarian budaya, penjagaan lingkungan, dan pembangunan daerah.
  \item Bekerjasama dengan media massa dan para kreator konten digital untuk memproduksi dan menyebarluaskan narasi-narasi positif tentang persatuan, keberagaman, dan potensi Indonesia sebagai negara kepulauan, serta menangkal disinformasi dan ujaran kebencian yang dapat memecah belah bangsa.
  \item Meningkatkan literasi digital masyarakat agar mampu memanfaatkan teknologi informasi secara bijak dan bertanggung jawab, serta kritis terhadap konten-konten negatif.
\end{enumerate}
Rekomendasi-rekomendasi ini bertujuan untuk memastikan bahwa Wawasan Nusantara tidak hanya menjadi sebuah konsep ideal, tetapi benar-benar terwujud dalam tindakan nyata yang dirasakan manfaatnya oleh seluruh lapisan masyarakat Indonesia. Dengan demikian, Wawasan Nusantara dapat terus menjadi sumber inspirasi, perekat persatuan, dan panduan strategis bagi bangsa Indonesia dalam menavigasi kompleksitas tantangan domestik dan dinamika global, menuju pencapaian cita-cita nasional. Masa depan Wawasan Nusantara sangat bergantung pada kemampuannya untuk terus berevolusi, beradaptasi, dan memberikan solusi konkret bagi permasalahan bangsa, yang memerlukan kepemimpinan visioner, partisipasi publik yang luas, serta kemauan kolektif untuk terus belajar dan berinovasi.
% References
 \bibliographystyle{apacite}
 \bibliography{references}

\end{document}
