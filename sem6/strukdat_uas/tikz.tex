\documentclass[12pt,a4paper]{article}
\usepackage{geometry}
\usepackage{tikz}
\usetikzlibrary{positioning}

\geometry{margin=2.5cm}

\begin{document}

\section{Ilustrasi Queue Menggunakan Array}

Queue adalah struktur data yang bekerja berdasarkan prinsip FIFO (First In First Out). Berikut adalah ilustrasi cara kerja queue menggunakan array:

\subsection{Awal: Queue Kosong}

Pada awalnya, queue kosong, sehingga `front = 0` dan `rear = -1`.

\begin{center}
    \begin{tikzpicture}[node distance=1cm]
        % Define styles
        \tikzstyle{array} = [draw, minimum width=1.5cm, minimum height=1cm];
        \tikzstyle{label} = [above, font=\small];

        % Draw array cells
        \node[array] (cell0) {};
        \node[array, right=of cell0] (cell1) {};
        \node[array, right=of cell1] (cell2) {};
        \node[array, right=of cell2] (cell3) {};
        \node[array, right=of cell3] (cell4) {};

        % Add labels for indices
        \node[label] at (cell0.north) {0};
        \node[label] at (cell1.north) {1};
        \node[label] at (cell2.north) {2};
        \node[label] at (cell3.north) {3};
        \node[label] at (cell4.north) {4};

        % Add front and rear pointers
        \node[below=of cell0, font=\small] (front) {front = 0};
        \node[below=of cell4, font=\small] (rear) {rear = -1};

        % Arrows for front and rear
        \draw[->, thick] (front) -- (cell0.south);
        \draw[->, thick] (rear) -- ++(0,-0.5) node[below] {\textbf{(kosong)}};
    \end{tikzpicture}
\end{center}

\subsection{Setelah Enqueue Pertama (Masukkan 'A')}

Setelah memasukkan elemen pertama ('A'), `rear` menjadi 0.

\begin{center}
    \begin{tikzpicture}[node distance=1cm]
        % Define styles
        \tikzstyle{array} = [draw, minimum width=1.5cm, minimum height=1cm];
        \tikzstyle{label} = [above, font=\small];

        % Draw array cells
        \node[array] (cell0) {A};
        \node[array, right=of cell0] (cell1) {};
        \node[array, right=of cell1] (cell2) {};
        \node[array, right=of cell2] (cell3) {};
        \node[array, right=of cell3] (cell4) {};

        % Add labels for indices
        \node[label] at (cell0.north) {0};
        \node[label] at (cell1.north) {1};
        \node[label] at (cell2.north) {2};
        \node[label] at (cell3.north) {3};
        \node[label] at (cell4.north) {4};

        % Add front and rear pointers
        \node[below=of cell0, font=\small] (front) {front = 0};
        \node[below=of cell0, font=\small, xshift=2cm] (rear) {rear = 0};

        % Arrows for front and rear
        \draw[->, thick] (front) -- (cell0.south);
        \draw[->, thick] (rear) -- (cell0.south);
    \end{tikzpicture}
\end{center}

\subsection{Setelah Enqueue Kedua (Masukkan 'B')}

Setelah memasukkan elemen kedua ('B'), `rear` menjadi 1.

\begin{center}
    \begin{tikzpicture}[node distance=1cm]
        % Define styles
        \tikzstyle{array} = [draw, minimum width=1.5cm, minimum height=1cm];
        \tikzstyle{label} = [above, font=\small];

        % Draw array cells
        \node[array] (cell0) {A};
        \node[array, right=of cell0] (cell1) {B};
        \node[array, right=of cell1] (cell2) {};
        \node[array, right=of cell2] (cell3) {};
        \node[array, right=of cell3] (cell4) {};

        % Add labels for indices
        \node[label] at (cell0.north) {0};
        \node[label] at (cell1.north) {1};
        \node[label] at (cell2.north) {2};
        \node[label] at (cell3.north) {3};
        \node[label] at (cell4.north) {4};

        % Add front and rear pointers
        \node[below=of cell0, font=\small] (front) {front = 0};
        \node[below=of cell1, font=\small, xshift=2cm] (rear) {rear = 1};

        % Arrows for front and rear
        \draw[->, thick] (front) -- (cell0.south);
        \draw[->, thick] (rear) -- (cell1.south);
    \end{tikzpicture}
\end{center}

\subsection{Setelah Dequeue Pertama (Hapus 'A')}

Setelah menghapus elemen pertama ('A'), `front` menjadi 1.

\begin{center}
    \begin{tikzpicture}[node distance=1cm]
        % Define styles
        \tikzstyle{array} = [draw, minimum width=1.5cm, minimum height=1cm];
        \tikzstyle{label} = [above, font=\small];

        % Draw array cells
        \node[array] (cell0) {};
        \node[array, right=of cell0] (cell1) {B};
        \node[array, right=of cell1] (cell2) {};
        \node[array, right=of cell2] (cell3) {};
        \node[array, right=of cell3] (cell4) {};

        % Add labels for indices
        \node[label] at (cell0.north) {0};
        \node[label] at (cell1.north) {1};
        \node[label] at (cell2.north) {2};
        \node[label] at (cell3.north) {3};
        \node[label] at (cell4.north) {4};

        % Add front and rear pointers
        \node[below=of cell1, font=\small] (front) {front = 1};
        \node[below=of cell1, font=\small, xshift=2cm] (rear) {rear = 1};

        % Arrows for front and rear
        \draw[->, thick] (front) -- (cell1.south);
        \draw[->, thick] (rear) -- (cell1.south);
    \end{tikzpicture}
\end{center}

\subsection{Setelah Enqueue Ketiga (Masukkan 'C')}

Setelah memasukkan elemen ketiga ('C'), `rear` menjadi 2.

\begin{center}
    \begin{tikzpicture}[node distance=1cm]
        % Define styles
        \tikzstyle{array} = [draw, minimum width=1.5cm, minimum height=1cm];
        \tikzstyle{label} = [above, font=\small];

        % Draw array cells
        \node[array] (cell0) {};
        \node[array, right=of cell0] (cell1) {B};
        \node[array, right=of cell1] (cell2) {C};
        \node[array, right=of cell2] (cell3) {};
        \node[array, right=of cell3] (cell4) {};

        % Add labels for indices
        \node[label] at (cell0.north) {0};
        \node[label] at (cell1.north) {1};
        \node[label] at (cell2.north) {2};
        \node[label] at (cell3.north) {3};
        \node[label] at (cell4.north) {4};

        % Add front and rear pointers
        \node[below=of cell1, font=\small] (front) {front = 1};
        \node[below=of cell2, font=\small, xshift=2cm] (rear) {rear = 2};

        % Arrows for front and rear
        \draw[->, thick] (front) -- (cell1.south);
        \draw[->, thick] (rear) -- (cell2.south);
    \end{tikzpicture}
\end{center}

\end{document}
