\documentclass[a4paper, 12pt]{article}
\usepackage{geometry}
\geometry{margin=2cm}
\usepackage[indonesian]{babel}
\usepackage{setspace}
\onehalfspacing{}
\usepackage{hyperref}
\hypersetup{
    colorlinks,
    citecolor=black,
    filecolor=black,
    linkcolor=blue,
    urlcolor=blue
}

\usepackage{graphicx}
\graphicspath{./images/}
\title{\textbf{Tugas Perorangan}\linebreak
\textbf{Tugas 4 Menganalisa Tugas}\linebreak}
\date{}

\usepackage{subfiles}
% \usepackage{indentfirst}
\setlength{\parindent}{20pt}
\begin{document}
\subfile{./subfiles/cover.tex}

% \tableofcontents
% \thispagestyle{empty}
% \pagebreak
% \clearpage
\setcounter{page}{1}
\section{Analisis Tugas}
analisislah salah satu tugas pegawai dalam sebuah perusahaan dengan tujuan untuk dijadikan dasar dalam merancang antarmuka yang sesuai\dots
\begin{enumerate}
  \item Nama Perusahaan:
  \item Jabatan:
  \item Tugas Utama:
  \item Sub-Sub Tugas:
\end{enumerate}
apa yang dapat anda simpulkan dari analisis tugas tersebut:
\begin{enumerate}
  \item Aplikasi apa yang sesuai dengan pekerjaan itu?
  \item Modul-Modul apa yang dibutuhkan untuk menyelesaikan tugasnya?
\end{enumerate}
\section{Penyelesaian}
Nama Perusahaan: Perusahaan tambang batu bara\newline
Jabatan: Sekretaris\newline
Tugas Utama:
\begin{itemize}
  \item Mengatur jadwal dan agenda pimpinan unit tambang\dots
  \item Mempersiapkan jadwal rapat dan pertemuan rapat yang berkaitan dengan operasional\dots
  \item Mendokumentasikan catatan dari rapat dan mendistribusikannya ke pihak terkait\dots
  \item Membantu pimpinan dalam menyiapkan laporan, presentasi dan dokumentasi lainnya terkait dengan operasional tambang\dots
  \item Mengelola komunikasi masuk (telepon, email, surat) yang berkaitan dengan operasional tambang\dots
  \item Memberikan dukungan administratif kepada pimpinan tambang\dots
  \item Berkoordinasi dengan departemen lain (divisi transportasi, produksi dan lainnya) terkait kebutuhan pimpinan\dots
\end{itemize}
Sub-Sub Tugas:
\begin{itemize}
  \item Menyusun dan memperbarui kalender dan jadwal pimpinan unit tambang\dots
  \item Mengatur undangan dan mengonfirmasi kehadiran peserta rapat terkait tambang\dots
  \item Menyiapkan materi presentasi, notulen dan dokumen pendukung rapat tambang\dots
  \item Mencata dan mendistribusikan notulen rapat operasional tambang\dots
  \item Mengelola surat masuk dan keluar terkait kegiatan tambang\dots
  \item Membantu pimpinan dalam pembuatan laporan operasional tambang\dots
  \item Berkoordinasi dengan departemen lain untuk mendukung kegiatan tambang\dots
\end{itemize}
berdasarkan dari tugas tugas yang dilaksanan oleh sekretaris pada perusahaan tambang batu bara ini dapat disimpulkan bahwa diperlukan aplikasi yang dapat menunjang dalam menyelesaikan tugasnya antara lain:
\begin{itemize}
  \item Aplikasi Manajemen kalender dan jadwal, aplikasi ini digunakan untuk membantu dalam mengatur dan mengelola jadwal dan agenda pimpinan unit tambang dengan mudah dan efisien\dots
  \item Aplikasi komunikasi dan kolaborasi antar karyawan, berguna untuk membantu dalam mengatur rapat dan mengkoordinasikan kepada departemen yang terkait\dots
  \item Aplikasi manajemen dokumen, untuk mengatur dan menyimpan dokumen penting dan mendistribusikannya kepada pihak terkait\dots
  \item Aplikasi produktivitas kantor, seperti pengolah kata (Microsoft word), Spreadsheet(Excel), dan presentasi(Power Point) untuk membuat dokumen\dots
  \item Aplikasi pengelolaan surat, untuk mengelola surat masuk dan keluar terkait kegiatan tambang\dots
\end{itemize}
Modul-Modul yang dibutuhkan antara lain:
\begin{itemize}
  \item Modul manajemen kalender dan jadwal, untuk membantu dalam mengatur dan mengelola jadwal dan agenda pimpinan unit tambang\dots
  \item modul manajemen rapat, untuk membantu dalam mengatur dan menyiapkan rapat dan pertemuan rapat yang berkaitan dengan operasional tambang\dots
  \item modul manajemen dokumen, untuk membantu dalam mengatur dan menyimpan dokumen penting untuk kegiatan tambang\dots
  \item modul komunikasi, untuk mengelola surat masuk dan keluar serta berkoordinasi dengan departemen lain\dots
  \item modul produktivitas, untuk membantu pimpinan dalam membuat laporan, presentasi dan dokumen lainnya yang membantu dalam kegiatan tambang\dots
  \item modul koordinasi, untuk membantu mengkoordinasikan kegiatan-kegiatan di unit tambang\dots
\end{itemize}
\end{document}
