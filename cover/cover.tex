\documentclass[a4paper,12 pt]{article}
\usepackage{color}
\usepackage[indonesian]{babel}
\usepackage{graphicx}
\graphicspath{ {./images/} }
\usepackage{blindtext}

\pagenumbering{arabic}

%membuat url bisa diklik
\usepackage{hyperref}
\hypersetup{
    colorlinks,
    citecolor=black,
    filecolor=black,
    linkcolor=blue,
    urlcolor=blue
}

\title{\textbf{Manual Book }\linebreak 
\textbf{Company Profile STMIK Bandung}\linebreak}
\date{}


\begin{document}

\maketitle
\thispagestyle{empty}
\begin{center}
\includegraphics[width=8cm,height=6cm]{logo}
\end{center}


\vspace{0.5 cm}
\begin{center}
\begin{tabular}{ll}
Nama & : Rifqi Fadil Fahrial \\
NIM & : 1222646\\
\end{tabular}
\newline
\newline
\newline
Mata Kuliah: Teknik Komputer \\
Dosen Pengampu: Apri Junaidi, S. Kom., M. Kom., MCS \linebreak
\newline
\newline
\textbf {TEKNIK INFORMATIKA} \\
\textbf {FAKULTAS INFORMATIKA} \\
\textbf {STMIK BANDUNG}
\linebreak
\textbf {2024} \linebreak
\end{center}

\pagebreak

\section{Kata Pengantar}
Segala Puji dan Syukur kami panjatkan selalu kepada Tuhan Yang Maha Esa atas Rahmat, Taufiq, dan Hidayah yang diberikan secara khusus dirancang untuk membantu Anda menjelajahi dan memanfaatkan sepenuhnya Manual Book Penggunaan Website Company Profile. Dalam era digital yang terus berkembang, kehadiran sebuah inovasi ini telah menjadi elemen penting bagi organisasi, perusahaan, dan individu untuk mempermudah pekerjaan. 
Buku panduan ini bertujuan untuk memberikan panduan yang jelas dan komprehensif mengenai cara menggunakan Website company profile ini secara efektif. Baik Anda seorang pengguna baru yang ingin memahami fungsi dan fitur dari Website ini, maupun seorang pengguna berpengalaman yang ingin memperdalam pengetahuan, buku ini hadir untuk memberikan petunjuk langkah demi langkah. 
Di dalam buku ini, Anda akan menemukan penjelasan terperinci tentang berbagai aspek, termasuk navigasi, fitur utama, proses registrasi, dan cara mengelola data berita yang akan di terbitkan pada halaman website. Selain itu, Anda juga akan dipandu melalui berbagai tugas umum, seperti menyediakan laporan yang otomatis terbuat pada aplikasi ini. 
Kami berkomitmen untuk memberikan pengalaman pengguna yang optimal, yang memungkinkan Anda untuk memperoleh manfaat penuh dari aplikasi ini. Oleh karena itu, buku panduan ini disusun dengan tujuan memberikan penjelasan yang jelas dan langkah-langkah yang mudah diikuti. Kami berharap, setelah membaca buku ini, Anda akan merasa lebih mampu menggunakan website ini secara maksimal. 
Terima kasih telah memilih buku panduan ini sebagai mitra Anda dalam menjelajahi website ini. Semoga buku ini memberikan kontribusi yang berharga dalam perjalanan Anda dan mempermudah pengalaman penggunaan website ini. 
Selamat membaca dan selamat menikmati petualangan digital Anda!
\pagenumbering{roman}
\clearpage

\tableofcontents
\pagebreak
\pagenumbering{arabic}

\section{PENDAHULUAN}
\subsection{Tujuan Pembuatan buku panduan}
Pembuatan buku panduan petunjuk penggunaan Penggunaan Website ini bertujuan untuk memberikan panduan yang jelas dan komprehensif kepada pengguna dalam menggunakan website ini dengan efektif. Tujuan utamanya adalah untuk memastikan bahwa pengguna, baik yang baru maupun yang berpengalaman, memiliki pemahaman yang mendalam tentang berbagai fitur, fungsi, dan proses yang terkait dengan penggunaannya. Buku panduan ini juga bertujuan untuk memudahkan pengguna dalam menjelajahi mengelola data penjualan dengan tepat. 
Dengan memberikan penjelasan yang terperinci dan langkah-langkah yang mudah diikuti, buku panduan ini diharapkan dapat meningkatkan kepercayaan diri pengguna dan memastikan mereka dapat memanfaatkan website ini dengan maksimal\dots
\subsection{Deskripsi Umum}
Deskripsi umum tentang Website Company Profile STMIK bandung  ini akan memberikan gambaran tentang karakteristik, fitur, tata cara pada sistem tersebut. Sistem yang kami bahas dalam buku panduan ini adalah sebuah sistem yang dirancang khusus untuk memfasilitasi Website ini. Tujuan utama dari websiste ini adalah untuk menyediakan Company Profile kampus STMIK Bandung dengan baik, rapi, praktis, dan efisien bagi pengguna, 
Dengan demikian, buku panduan ini bertujuan untuk memberikan panduan yang komprehensif dan terperinci kepada pengguna website ini\dots

\section{Perangkat yang digunakan}
\subsection{Perangkat Lunak}
Perangkat lunak atau peranti lunak adalah istilah khusus untuk data yang diformat 
dan disimpan secara digital, termasuk program komputer, dokumentasinya, dan berbagai informasi yang bisa dibaca, dan ditulis oleh komputer. Dengan kata lain, bagian sistem komputer yang tidak berwujud. 
Beberapa perangkat lunak yang digunakan adalah sebagai berikut:
\begin{enumerate}
  \item Sistem Operasi(Operating System) : Microsoft Windows/Android 
  \item tools penjelajah seperti chrome, Firefox, dll.
  \item Visual Studio Code
\end{enumerate}

\subsection{Perangkat Keras}
Perangkat keras (hardware) mengacu pada komponen fisik yang membentuk sebuah sistem komputer atau perangkat elektronik. Perangkat keras berfungsi untuk menerima, memproses, dan menyimpan data, serta menjalankan perintah dari perangkat lunak (software). 
Berikut adalah beberapa contoh perangkat keras yang umum digunakan :
\begin{enumerate}
  \item komputer
  \item Monitor
  \item Keyboard
  \item Mouse
  \item Central Processing Unit (CPU)
  \item Random Acces Memory (RAM)
  \item Hard Disk Drive (HDD)
  \item Power Supply
\end{enumerate}
\section{Penggunaan}
Berikut ini adalah langkah-langkah yang perlu\linebreak 
diikuti untuk memulai mengakses website company profile ini:
\subsection{Tampilan Website}
biji keras
\section{Menampilkan Gambar di \LaTeX}
\blindtext\\
\linebreak
Ini adalah contoh untuk  menyisipkan gambar dalam dokumen seperti terlihat pada Gambar \ref{fig:logo}.
\begin{figure}[htp]
	\centering
	\includegraphics[width=4cm,height=5cm]{logo}
	\caption{Logo Institut Teknologi Telkom Purwokerto}
	\label{fig:logo}
\end{figure}

\section{Menampilkan script Program di \LaTeX}
\blindtext\\
\linebreak
Berikut adalah script program python untuk menghitung perkalian :

\begin{verbatim}
ulang = int(input ("Jumlah Looping : "))
kali = int(input ("Perkalian berapa :")) 

for i in range (1, ulang + 1) :
  print(i, "x", kali, "=", i * kali )
print ("Looping sudah dilakukan")
\end{verbatim}


\end{document}
