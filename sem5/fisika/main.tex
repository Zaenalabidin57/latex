\documentclass{article}
\usepackage{amsmath}
\begin{document}

\section*{Soal Fisika: Gerak Jatuh Bebas}
\section*{Solusi Persamaan Gas Ideal}

\subsection*{Data yang Diketahui:}
\begin{itemize}
\item Persamaan gas ideal: $PV = nRT$
\item $P = 1$ atm
\item $V = 22,4$ L
\item $n = 6,023 \times 10^{23}$ molekul (bilangan Avogadro)
\item $T = 273,15$ K (suhu ruang dalam Kelvin)
\end{itemize}

\subsection*{a) Mencari Konstanta Boltzmann (k)}
Hubungan konstanta gas ideal ($R$) dengan konstanta Boltzmann ($k$):
\[R = k \times N_A\]

Dari persamaan gas ideal:
\[k = \frac{PV}{nT}\]

Substitusi nilai:
\[k = \frac{1 \text{ atm} \times 22,4 \text{ L}}{(6,023 \times 10^{23}) \times 273,15 \text{ K}}\]

Konversi satuan:
\begin{itemize}
\item 1 atm = 101325 Pa
\item 1 L = $10^{-3}$ m³
\end{itemize}

\[k = \frac{101325 \text{ Pa} \times 22,4 \times 10^{-3} \text{ m}^3}{(6,023 \times 10^{23}) \times 273,15 \text{ K}}\]
\[k = 1,38 \times 10^{-23} \text{ J/K}\]

\subsection*{b) Satuan yang Tepat}
Analisis dimensi:
\begin{itemize}
\item $[P] = \text{N}/\text{m}^2 = \text{kg}/(\text{m} \cdot \text{s}^2)$
\item $[V] = \text{m}^3$
\item $[n] = \text{molekul}$
\item $[T] = \text{K}$
\end{itemize}

Maka satuan konstanta Boltzmann:
\[[k] = \frac{[P][V]}{[n][T]} = \frac{\text{J}}{\text{K}} = \frac{\text{kg} \cdot \text{m}^2}{\text{s}^2 \cdot \text{K}}\]

\section*{Kesimpulan:}
\begin{enumerate}
\item Konstanta Boltzmann ($k$) = $1,38 \times 10^{-23}$ J/K
\item Satuan yang tepat adalah Joule per Kelvin (J/K)
\end{enumerate}

\section*{Solusi Soal Gerak Partikel}
Diberikan persamaan posisi:
\[s = t^3 - 15t^2 + 63t - 30 \text{ (meter)}\]

\section {soal 2}
\section*{Solusi Soal Kesetimbangan Tangga}

\subsection*{Data yang Diketahui:}
\begin{itemize}
\item Massa tangga ($m$) = 50 kg
\item Panjang tangga ($L$) = 8 m
\item Sudut dengan tanah ($\phi$) = 61,81°
\item Berat ($W$) = $mg$ = 50 kg × 9,8 m/s² = 490 N
\item Pusat berat di tengah tangga
\item Dinding licin (tidak ada gaya gesek)
\end{itemize}

\subsection*{Analisis Gaya:}
\begin{itemize}
\item $F_1$ = gaya normal dari dinding (horizontal)
\item $F_2$ = gaya reaksi lantai (membentuk sudut $\theta$)
\item $W$ = gaya berat (vertikal ke bawah)
\end{itemize}

\subsection*{Persamaan Kesetimbangan:}
1. Kesetimbangan gaya horizontal:\\
   $\sum F_x = 0$: $F_1 - F_2 \cos \theta = 0$ ... (1)

2. Kesetimbangan gaya vertikal:\\
   $\sum F_y = 0$: $F_2 \sin \theta - W = 0$ ... (2)

3. Kesetimbangan momen terhadap titik kontak dengan lantai:\\
   $\sum M = 0$: $F_1(4 \sin 61,81°) - W(4 \cos 61,81°) = 0$ ... (3)

\subsection*{a) Mencari $F_1$:}
Dari persamaan (3):
\[F_1(4 \sin 61,81°) = W(4 \cos 61,81°)\]
\[F_1 = \frac{490 \times \cos 61,81°}{\sin 61,81°}\]
\[F_1 = 490 \times \frac{0,472}{0,882}\]
\[F_1 = 262,3 \text{ N}\]

\subsection*{b) Mencari $F_2$:}
Dari persamaan (2):
\[F_2 \sin \theta = W = 490 \text{ N}\]

Dari persamaan (1):
\[F_2 \cos \theta = F_1 = 262,3 \text{ N}\]

Menggunakan Pythagoras:
\[F_2 = \sqrt{490^2 + 262,3^2}\]
\[F_2 = 555,8 \text{ N}\]

\subsection*{c) Mencari sudut $\theta$:}
\[\theta = \arctan(\frac{490}{262,3})\]
\[\theta = 61,81°\]

\section*{Kesimpulan:}
\begin{enumerate}
\item $F_1 = 262,3$ N
\item $F_2 = 555,8$ N
\item $\theta = 61,81°$
\end{enumerate}

\section*{Catatan:}
Sudut $\theta$ sama dengan sudut $\phi$ karena:
\begin{itemize}
\item Dinding licin menyebabkan $F_1$ horizontal
\item Syarat kesetimbangan momen terpenuhi
\item Distribusi gaya membentuk segitiga yang similar
\end{itemize}
\section{ soal 3}
\section*{Solusi Soal Gerak Partikel}
\subsection*{a) Persamaan Kecepatan}
Kecepatan adalah turunan pertama dari posisi:
$$[v = \frac{ds}{dt} = 3t^2 - 30t + 63 \text{ (meter/detik)}]$$
\subsection*{b) Persamaan Percepatan}
Percepatan adalah turunan dari kecepatan:
[a = \frac{dv}{dt} = 6t - 30 \text{ (meter/detik²)}]
\subsection*{c) Titik Berhenti (v = 0)}
Mencari titik dimana $v = 0$:
[3t^2 - 30t + 63 = 0]
Menggunakan rumus kuadrat:
[t = \frac{-b \pm \sqrt{b^2 - 4ac}}{2a}]
Dimana:
\begin{itemize}
\item $a = 3$
\item $b = -30$
\item $c = 63$
\end{itemize}
[t = \frac{30 \pm \sqrt{900 - 756}}{6}]
[t = \frac{30 \pm \sqrt{144}}{6}]
[t = \frac{30 \pm 12}{6}]
[t = 7 \text{ atau } t = 3 \text{ detik}]
\subsection*{d) Analisis Gerak Mundur}
Partikel bergerak mundur ketika $v < 0$
Dari bentuk parabola $v = 3t^2 - 30t + 63$:
\begin{itemize}
\item Saat $t < 3$: $v > 0$ (maju)
\item Saat $3 < t < 7$: $v < 0$ (mundur)
\item Saat $t > 7$: $v > 0$ (maju)
\end{itemize}
\subsection*{e) Analisis Percepatan Positif}
Percepatan menjadi positif ketika:
[6t - 30 = 0]
[t = 5 \text{ detik}]
\section*{Kesimpulan:}
\begin{enumerate}
\item $v(t) = 3t^2 - 30t + 63$ m/s
\item $a(t) = 6t - 30$ m/s²
\item Titik berhenti: $t = 3$ dan $t = 7$ detik
\item Bergerak mundur: $3 < t < 7$ detik
\item Percepatan positif: $t > 5$ detik
\end{enumerate}
\end{document}
