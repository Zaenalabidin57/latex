\documentclass[a4paper,12pt]{article}
\usepackage{color}
\usepackage{amsmath}
\usepackage[indonesian]{babel}
\usepackage{graphicx}
\graphicspath{ {./images/} }

\author{RifqiFadil Fahrial}


\begin{document}
\section{Himpunan penyelesaian dari sistem persamaan 2x+y=4 dan 3x+y=6}
\begin{align*}
  3x+y=6\\
  y=6-3x\\
  2x+2(6-3x)=4\\
  2x+12-6x=4\\
  -4x=4-12\\
  -4x=-8\\
  x=2\\
\end{align*}
\begin{align*}
  y=6-3(2)
  y=6-6
  y=0
\end{align*}
Himpunan penyelesaiannya adalah 
\begin{align*}
  x=2
  y=0
\end{align*}

\begin{align*}
  \bar{x}=(\sigma(\[
    t_i
  \]))
  /\sigma f
\end{align*}

 $ S = \int_x \left\{ \frac{1}{2} \sum_a \partial^\mu X_a \partial_\mu X_a + V(\rho) \right\} $
  
\end{document}


