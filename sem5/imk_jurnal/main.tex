\documentclass[a4paper, 12pt]{article}
\usepackage{geometry}
\geometry{margin=2cm}
\usepackage[indonesian]{babel}
\usepackage{setspace}
\onehalfspacing{}
\usepackage{hyperref}
\hypersetup{
    colorlinks,
    citecolor=black,
    filecolor=black,
    linkcolor=blue,
    urlcolor=blue
}

\usepackage{graphicx}
\graphicspath{./images/}
\title{\textbf{Tugas Perorangan}\linebreak
\textbf{Review 5 Jurnal IMK}\linebreak}
\date{}

\usepackage{subfiles}
% \usepackage{indentfirst}
\setlength{\parindent}{20pt}
\begin{document}
\subfile{./subfiles/cover.tex}

\tableofcontents
\thispagestyle{empty}
\pagebreak
\clearpage
\setcounter{page}{1}

\section{PENERAPAN INTERAKSI MANUSIA DAN KOMPUTER PADA ANTARMUKA SISTEM INFORMASI AKADEMIK}
\url{https://doi.org/10.51401/jinteks.v5i1.2467}\newline
Jurnal: Jurnal Informatika Teknologi Dan Sains Vol 5, No. 1 (2023) Edisi 15\newline
Ditulis oleh : Isnaeni Hamidah, Bangkit Indarmawan Nugroho, Sarif Surorejo\newline
Artikel ini menjelaskan bagaimana penulis meneliti bagaimana hasil dari implementasi dari gagasan \emph{Human Computer Interaction} (HCI) pada sistem informasi akademik apakah dapat mendapatkan hasil yang efektif ataupun hasil yang berbalik dari yang diigninkan berdasarkan sistem penelitian sistematik dengan mencari jurnal jurnal mengenai interaksi manusia dan komputer yang kemudian sang penulis dapat mengambil kesimpulan bahwa Implementasi dari gagasan HCI dapat membantu dalam menambah ke efektifan dari sistem informasi akademik namun jarang sekali implementasi yang baiknya dikarenakan dibutuhkan penelitian dan pemahaman lebih lanjut dalam ilmu Interaksi manusia dan komputer yang sesuai dengan implementasi yang dibutuhkan agar efektif yang dalam hal ini sangatlah sulit dikarenakan manusia adalah mahluk yang memiliki banyak variabel yang tidak dapat tercakup semuanya dalam implementasi sistem informasi yang ada di akademik\dots
\section{ANALISA WEBSITE PRODI SISTEM INFORMASI UNSIKA BERDASARKAN PRINSIP DAN PARADIGMA INTERAKSI MANUSIA DAN KOMPUTER}
\url{https://doi.org/10.33557/jurnalmatrik.v25i2.2381}\newline
Jurnal: Jurnal Ilmiah Matrik Vol.25 Nomor 2 (2023)\newline
Penulis: Ahzka Nabbilah Tuzzahrah, Apriade Voutama, Azhari Ali Ridha\newline
Dalam artikel ini menjelaskan bagaimana penulis menganalisa dari fungsi sebuah sistem berdasarkan implementasi dari prinsip dan paradigma dari interaksi manusia dan komputer(IMK), yang hasilnya menunjukan bahwa dari sistem yang di analisa yakni Website dari prodi sistem informatika telah memenuhi aspek interaksi manusia dan komputer yang kemudian menghasilkan sebuah website yang mudah digunakan oleh penggunanya dan mendapatkan umpan baik dari pengguna. namun dalam websitenya tidak memiliki akses yang baik untuk pengguna disabilitas yang dapat menghalangi pengguna yang memiliki disabilitas seperti tuna netra yang tidak dapat mendapatkan informasi dengan mudah karena belum mengimplementasikan sehingga hal ini dapat ditingkatkan lagi untuk kemudahan pengguna untuk mengakses websitenya\dots
\section{AUGMENTED REALITY SEBAGAI METAFORA BARU DALAM TEKNOLOGI INTERAKSI MANUSIA DAN KOMPUTER}
\url{http://eprints.undip.ac.id/40503/}\newline
Jurnal: JURNAL SISTEM KOMPUTER , Vol.1 (No.2)\newline
Penulis: Kurniawan Teguh Martono\newline
\emph{Augmented Reality} atau disingkat AR adalah suatu cara baru untuk menghubungkan dunia nyata dengan dunia ciptaan yaki dunia maya, dengan teknologi ini dapat dengan mudah memprojeksikan sebuah gambar kedalam dunia nyata menggunakan layar baik secara 2 dimensi ataupun secara 3 dimensi yang nantinya dapat membantu untuk memberikan informasi kepada pengguna. salah satu contoh penggunaannya adalah untuk seorang arsitek yang sedang membuat prototype dari rumah yang akan dibangun, sang arsitek dapat menggunakan AR ini untuk memproyeksikan gambar dari bangunan yang akan dibangun ke tanah dari lokasi bangunan yang akan dibanun, hal ini dapat membantu sang arsitek karena dapat memberikan informasi bagaimana rumah itu akan dibangun, apakah ada lokasi yang tidak memungkinkan membangun bagian dari rumah yang kemudian AR ini dapat membantu sang pemilik rumah untuk melihat bagaimana rumah yang akan dibuat terbentuk secara maya agar sang pemilik rumah dapat memberikan masukan agar bagaimana rumah yang akan dibangun tersebut sesuai dengan selera dari sang pemilik rumah\dots\newline
hal ini memungkinkan AR menjadi sebuah sarana baru dari bagaimana informasi ditampilkan namun meski begitu sistem ini masih dalam masa pengembangan dikarenakan pengoperasiannya yang sulit dan sangat tidak mungkin untuk dioperasikan oleh yang tidak memiliki pengalaman dalam menggunakan sistem ini, sehingga sistem ini masih kurang untuk digunakan untuk keperluan ekonomis namun hanya terbatas pada sektor akademis yang semoga saja dalam beberapa tahun lagi sistem ini dapat digunakan dengan mudah oleh semua orang\dots
\section{ANALISIS USABILITAS PADA PERMAINAN 'THE ZOO' BERBASIS KINECT}
\url{https://doi.org/10.14710/jtsiskom.1.4.2013.104-112}\newline
Jurnal: Jurnal Teknologi dan Sistem Komputer, vol. 1, no. 4, pp. 104-112, Oct. 2013\newline
Penulis: Feby Adhi Putra, Kodrat Imam Satoto, Kurniawan Teguh Martono\newline
Permainan 'The Zoo' adalah permainan edukasi pengenalan kosakata hewan dalam Bahasa Inggris untuk anak usia empat sampai tujuh tahun yang berbasis Kinect\dots Kinect adalah sebuah perangkat keras yang dikembangkan oleh microsoft untuk perangkat Xbox 360 nya  yang dapat membaca gerakan manusia menggunakan sensor inframerah yang memetakan gerakan pengguna yang nantinya dapat digunakan oleh aplikasi seperti 'The Zoo' ini.Sang penulis menganalisa interaksi ini menggunakan Teori Jacob Nielsen yakni usabilitas harus mencakup lima komponen penting yakni \emph{Learnability, Efficiency, Memorability, Errors and Satisfaction} yang mana nantinya dapat menggambarkan bagaimana sistem ini dapat dengan mudah digunakan ataupun tidak layak digunakan\dots dalam pengujiannya ditemukan jika kinect ini dapat digunakan secara mudah untuk penggunanya namun masih jarang digunakan dikarenakan ketersediaan dari perangkat lunak dan jarangnya aplikasi yang mengimplementasikan perangkat keras ini kedalam aplikasinya\dots
\section{ANALISIS KONSEP INTERAKSI MANUSIA DAN KOMPUTER
PADA ANTARMUKA SISTEM INFORMASI AKADEMIK
UNIVERSITAS MUHAMMADIYAH PONOROGO}
\url{https://dx.doi.org/10.24269/mtkind.v11i1.525}\newline
Jurnal: Multitek Indonesia Vol 11, No1 (2017): Juni\newline
Penulis: Desriyanti\newline
Dalam Artikel ini penulis mengemukakan bagaimana evaluasi dari konsep interaksi manusia dan komputer pada antarmuka dari "SIPRUS" yang ada di perpustakaan di SMA Muhammadiyah 2 yogyakarta. dari hasil penelitiannya ditemukan bahwa sistem informasi dari "SIPRUS" ini sudah cukup baik yang ditopang oleh hasil survey yang dilakukan oleh penulis ini yang menggunakan metode \emph{Purposive Sampling} yang sangat membantu untuk mendapatkan sampel data yang berdasarkan adanya tujuan tertentu yakni \emph{Human Computer Interaction} (HCI) pada sistem informasi AKADEMIK\dots dengan pertanyaan yang berbobot yang digunakan dalam survey ini yang berdasarkan kriteria yang dikemukakan oleh Jacob Nielsen yakni \emph{Learnability, Efficiency, Memorability, Errors and Satisfaction} menjadikan variabel yang didapatkan sangatlah cukup sebagai patokan dalam memahami bagaimana sistem informasi ini memiliki \emph{Usability} yang mumpuni sebagai sistem penunjang dari perpustakaan di SMA Muhammadiyah 2 yogyakarta\dots
\end{document}

