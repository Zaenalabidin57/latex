\documentclass[a4paper, 12pt]{article}
\usepackage{geometry}
\geometry{margin=2cm}
\usepackage[indonesian]{babel}
\usepackage{setspace}
\onehalfspacing{}
\usepackage{hyperref}
\usepackage{float}
\hypersetup{
    colorlinks,
    citecolor=black,
    filecolor=black,
    linkcolor=blue,
    urlcolor=blue
}

\usepackage{graphicx}
\graphicspath{./images/}
\title{\textbf{Tugas UAS}\linebreak
\textbf{Konfigurasi Acces point Achmad}\linebreak}
\date{}

\usepackage{subfiles}
% \usepackage{indentfirst}
\setlength{\parindent}{20pt}
\begin{document}
\subfile{./subfiles/cover.tex}

% \tableofcontents
% \thispagestyle{empty}
% \pagebreak
\clearpage
\section{Tugas}
Pada sebuah kantor seorang manajer bernama Achmad menginginkan ruanganny memiliki jaringan hotspot tersendiri, berbeda dengan bawahannya. Achmad menginginkan hanya ada 6 clients saja yang dapat terkoneksi dengan access point tersebut. meskipun sudah dibatasi dengan penggunaan \textit{mac filtering} pada 6 clients, Achmad masih tidak puas dan menginginkan hotspotnya memiliki \textit{password security}\dots Achmad mengetahui bahwa ada bawahannya yang mengerti tentang \textit{hacking} pada jaringan \textit{wireless}\dots Pada suatu waktu Achmad menyadari bahwa hotspotnya memiliki pengunjung gelap. Achmad mengetahui \textit{mac address} pengunjung gelap tersebut dan menginginkan pengunjung gelap tersebut tidak dapat terkoneksi ke hotspot milikinya. Koneksi internet hotspot tersebut didapat dari \textit{access point} di ruang IT. kedua \textit{acces point} (AP IT dan AP ruangan Achmad) harus di konfigurasi terlebih dahulu agar ruangannya memiliki koneksi internet. Berikanlah solusi untuk hal tersebut. diatas
\section{Penyelesaian}
\subsection{Penjabaran}
Berdasarkan informasi diatas maka harus mengkonfigurasi:
\begin{enumerate}
  \item Mengkonfigurasi \textit{Access Point} agar terkoneksi dengan internet\dots
  \item membatasi pengguna maksimal 6 pengguna dengan \textit{Mac filtering}\dots
  \item menambahkan \textit{password security} pada \textit{access point}\dots
  \item mengatasi \textit{hacker} pada jaringan di \textit{acces point} Ruang IT agar tidak terhubung dengan hotspot\dots
\end{enumerate}
\subsection{Mengkonfigurasi \textit{Access Point} agar terhubung dengan internet}
Untuk menghubungkan jaringan ini dengan internet maka harus ditentukan dahulu Host dari jaringan ini dan Acces point lainnya sebagai \textit{repeater} agar memudahkan proses konfigurasi\dots oleh sebab itu maka Ruangan Achmad menjadi Hostnya dan Ruangan IT menjadi Repeaternyakemudian \textit{access point} Achmad terhubung dengan koneksi fiber dengan provider internet untuk mendapatkan koneksi internet dan menerima \textit{IP Public} untuk menjelajah internet\dots
\begin{figure}[H]
  \begin{center}
    \includegraphics[width=0.95\textwidth]{images/gambar1.png}
  \end{center}
  \caption{Gambar Koneksi antara \textit{Access Point} ruang Acmhad dan Ruang IT agar terhubung dengan internet}\label{fig:accespoint}
\end{figure}
berdasarkan gambar \ref{fig:accespoint} dapat terlihat bahwa \textit{access point} ruang Acmhad dan Ruang IT terhubung dengan internet\dots
\subsection{Membatasi pengguna maksimal 6 pengguna dengan \textit{Mac filtering}}
Kemudian untuk Pengguna dari jaringan ini maka diberlakukan sistem \textit{Mac Filtering} yang bekerja sebagai penjaga dari koneksi dari perangkat yang tidak dikenal\dots hal ini dapat dikonfigurasi pada \textit{Acces Point} pada Ruang it jika 6 perangkat itu ada pada ruang IT dan tidak ada pada ruang Acmhad\dots \newline
untuk membatasi pengguna menggunakan \textit{Mac Filtering} maka diberlakukan \textit{Whitelist} dan \textit{Blacklist} dari \textit{Mac Address} yang terhubung dengan \textit{access point}, jika \textit{Mac Adress} dari perangkat berada pada \textit{whitelist} maka perangkat tersebut dapat terhubung ke \textit{access point} dan sebaliknya dengan \textit{Blacklist}\dots
\begin{figure}[H]
  \begin{center}
    \includegraphics[width=0.95\textwidth]{images/gambar2.png}
  \end{center}
  \caption{Gambaran Mac Filtering}\label{fig:macfiltering}
\end{figure}
misalkan ada perangkat baru yang dimiliki oleh Abidin ingin terhubung dengan \textit{Acces Point} di ruang IT maka abidin meminta IP Address kepada \textit{Acces Point} yang kemudian membaca \textit{Mac Address} dari perangkat abidin, Jika \textit{Mac Address} nya berada pada \textit{Blacklist} maka perangkat Abidin Tidak dapat terhubung dengan \textit{Access Point} dan menerima koneksi internet, namun jika berada dalam \textit{Whitelist} maka dapat terhubung dengan \textit{Acces Point} dan menerima Koneksi Internet\dots
Misalkan ada 6 perangkat yang diizinkan maka dapat dibuatkan \textit{Whitelist} pada \textit{Acces Point} seperti Berikut:
\begin{table}[H]
\begin{tabular}{|l|l|l|}
\hline
Perangkat   & Mac Addres        & Status    \\ \hline
Perangkat 1 & 00:00:00:00:01:01 & Diizinkan \\ \hline
Perangkat 2 & 00:00:00:00:01:02 & Diizinkan \\ \hline
Perangkat 3 & 00:00:00:00:01:03 & Diizinkan \\ \hline
Perangkat 4 & 00:00:00:00:01:04 & Diizinkan \\ \hline
Perangkat 5 & 00:00:00:00:01:05 & Diizinkan \\ \hline
Perangkat 6 & 00:00:00:00:01:06 & Diizinkan \\ \hline
\end{tabular}
\end{table}
\subsection{Menambahkan \textit{password security} pada \textit{Access Point}}
Kemudian Untuk menambahkan Keamanan dari \textit{Acces Point} maka diberlakukannya sistem Password untuk Perangkat baru yang ingin terhubung dengan \textit{Acces Point} untuk mencegah dari perangkat yang tidak dikenal terhubung dan melakukan kejahatan pada Jaringan ini\dots Untuk menambahkan \textit{password security} pada \textit{Acces Point} pada ruang IT dan ruang Acmhad maka diberlakukannya \textit{Password Security} seperti berikut:
\begin{table}[H]
  \begin{tabular}{|l|l|l|}
    \hline
    Nama Acces Point     & Sistem password & Password    \\ \hline
    Access Point Achmad  & WPA2-PSK        & Admin\#1234 \\ \hline
    Acces Point Ruang IT & WPA2-PSK        & password    \\ \hline
  \end{tabular}
\end{table}
yang nantinya jika ada perangkat baru yang ingin melakukan koneksi dengan \textit{Acces Point} maka akan diminta password untuk terhubung dengan \textit{Acces Point}, jika benar memasukan password yang benar maka perangkat tersebut akan menerima koneksi internet dan sebaliknya\dots
\subsection{Mengatasi \textit{hacker} pada \textit{Access Point}}
Kemudian untuk mengatasi hacker yang terhubung pada jaringan maka cara mengatasinya adalah dengan mengidentifikasi dari perangkat mana yang terindikasi Hacker kemudian carilah \textit{Mac Address} dari hacker tersebut dan kemudian masukan pada \textit{Blacklist} pada \textit{Access Point} sehingga hacker tersebut tidak dapat terhubung dengan \textit{Access Point} meskipun mengetahui password dari \textit{Access Point} yang kemudian dapat diketahui siapa yang menjadi hacker tersebut dari perangkatnya yang tidak terhubung dengan \textit{Access Point}\dots
untuk melakukan \textit{Blacklist} maka perlu dilakukan pengisian \textit{Blacklist} pada \textit{Access Point} yang kemudian diisi oleh \textit{Mac Address} dari perangkat hacker yang terindikasi, contohnya:
\begin{table}[H]
  \begin{tabular}{|l|l|l|}
    \hline
    Nama Pengguna & Mac Address       & Status   \\ \hline
    Pengguna 4    & 00:00:00:00:01:04 & Diblokir \\ \hline
  \end{tabular}
\end{table}
\begin{figure}[H]
  \begin{center}
    \includegraphics[width=0.95\textwidth]{images/Gambar3.png}
  \end{center}
  \caption{Simulasi \textit{Blacklist}}\label{fig:hacker}
\end{figure}
berdasarkan gambar \ref{fig:hacker} dapat terlihat jika perangkat yang terindikasi hacker telah terindentifikasi dan \textit{Mac Address} dari perangkat hacker telah ditemukan kemudian dimasukan ke dalam \textit{Blacklist} pada \textit{Access Point}\dots
\end{document}
